Ik volg de opleiding netwerkbeheerder bij het centrum voor deeltijds onderwijs Don Bosco te Wilrijk. Dit eindwerk is daarbij een onderdeel van mijn opdrachten om het certificaat netwerkbeheerder te behalen.
Mijn opleiding bestaat uit drie grote onderdelen, namelijk: Praktijk, Theorie en ASPV.
Bij praktijk is ons leerdoel om zoveel mogelijk praktische kennis te verzamelen over het beheren van netwerken, zoals het opzetten van een server waarmee je verschillende services kan aanbieden.
Denk hierbij aan bijvoorbeeld een dhcp/dns-server, webserver en fileserver om er even een paar te noemen.

Bij Theorie is ons leerdoel om zoveel mogelijk theoretische informatie te vergaren, zoals de achterliggende werking van het TCP/IP-protocol, de componenten van een pc en de elektrische eigenschappen die erin gebruikt worden.

Bij ASPV is ons leerdoel voornamelijk om algemene kennis te verzamelen, denk hierbij vooral aan dagelijkse dingen zoals belastingen, verzekeringen, etc.
Althans dat was het enige doel in het vorige jaar, dit jaar kijken we ook naar web-ontwikkeling, zoals het maken van statische, dynamische en database-gestuurde websites.

Bij deze zou ik graag van de gelegenheid gebruik maken om enkele mensen te bedanken:

Eerst en vooral: Robert Keersse: omdat hij een geweldige leerkracht is en mij op het goede (lees: open-source) pad heeft gezet. Zonder zijn steun was me dat nooit gelukt.
Luc jennes: Voor het bijdragen aan mijn theoretische kennis over personal computers.
Peter Peeters: Voor het bijbrengen van enkele belangrijke levenslessen, het bouwen van websites en voor het algemeen beheer in ons klaslokaal.
Kris buytaert en Inuits in zijn geheel: Om mij een kans te geven om een deel uit te maken van hun geweldige bedrijf, het is voor mij een geweldige kans om bij te leren en ik hoop me in de toekomst nog nuttig te bewijzen voor hen.

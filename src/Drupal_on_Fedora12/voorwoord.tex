Allereerst wil ik mijn dank uitspreken over mijn begeleiders, Marc Decroos, Bram
en Dirk bij Terbeke-Pluma. Niet alleen heb ik van hen ongeloofelijk veel technische kennis opgedaan, maar
ook leren werken, en zeker, werk leren zien.
\\\\
Graag wil ik mijn leerkrachten bedanken voor hun volharding en steun. De heer
Peeters, waar ik van heb ``leren lezen en schrijven'' en in bijzonder dat ik mocht
proeven van zijn kennis over PHP en SQL. Wat me het meest bijblijft van de
theorie lessen gegeven door de heer Jennes, is dat deze meestal ver boven mijn
petje gingen, nochthans begrijp ik nu wel dat je de lat nooit hoog genoeg kan
leggen.
\\\\
Mijn eindwerk over Drupal is anders van structuur dan gevraagd, dit komt
hoofdzakelijk omdat het gebruikelijke ``theorie - praktijk'' hoofdstuk voor mij
niet direct duidelijk was. Na voldoende argumenten stemde de heer Keersse ermee in de
structuur te wijzigen. Deze bestaat uit ``inhoud aanmaken'' en ``beheren''. De
praktische installatie kan je terugvinden in de bijlage. 
\\\\
Alle hoofdstukken, titels en ondertitels, zijn hetzelfde als gebruikt in de
Nederlandse vertaling van Drupal, dit maakt het eenvoudig te zoeken in de help
functie van de website. Het is niet mijn bedoeling dat dit eindwerkje een
volledige beschrijving geeft wat je allemaal met Drupal kan, het kan misschien dienen als startpunt voor
nieuwe gebruikers.

\begin{flushright}
Mitchell Huyssen\\juni 2010
\end{flushright}


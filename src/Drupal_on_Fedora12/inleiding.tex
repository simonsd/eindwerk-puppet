\chapter{CMS}
Het contentmanagementsysteem \index{contentmanagementsysteem}(CMS) waarover mijn eindwerk gaat is
Drupal, dit maakt gebruikt van de scripttaal 'PHP' met een bijbehorende SQL database, ik maak hierbij gebruik van
MySQL. Het is eveneens mogelijk gebruik te maken van andere databases. Het door
mij gekozen besturingssysteem is Linux Fedora 12.
\section{Contentmanagementsysteem}
Een content-beheersysteem of contentmanagementsysteem is een softwaretoepassing,
meestal een web-applicatie, die het mogelijk maakt dat mensen eenvoudig, zonder 
veel technische kennis, documenten en gegevens op internet kunnen publiceren. 
Als afkorting wordt ook wel CMS gebruikt, naar het Engelse content management system. 
Een functionaliteit van een CMS is dat gegevens zonder lay-out (als platte tekst) 
kunnen worden ingevoerd, terwijl de gegevens worden gepresenteerd aan bezoekers met 
een lay-out door toepassing van sjablonen. Een CMS is vooral van belang voor websites 
waarvan de inhoud regelmatig aanpassing behoeft, en de inhoud in een vaste lay-out 
wordt gepresenteerd aan bezoekers. De meeste grote bedrijven gebruiken voor hun website 
tegenwoordig een CMS.
\subsection{Onderdelen}
Een CMS bestaat ten minste uit de volgende onderdelen:
\begin{itemize}
  \item een (bijna altijd afgeschermde) administratiemodule, waar gegevens kunnen worden ingevoerd, verwijderd of aangepast.
  \item een database of een andere vorm van opslag van de gegevens.
  \item een presentatiemodule, waar de ingevoerde gegevens door bezoekers
  kunnen worden bekeken.
\end{itemize}
Daarnaast kunnen er andere onderdelen zijn:
\begin{itemize}
  \item een zoekmodule
  \item een inlogmodule voor bezoekers, als het niet gewenst is dat anonieme bezoekers toegang hebben tot de inhoud
  \item een beheersmodule voor de gegevens van geautoriseerde bezoekers (en beheerders)
  \item een beheersmodule voor de presentatiesjablonen
  \item een module om persoonlijke informatie aan de bezoeker te tonen (personalisaties)
  \item een module om centraal artikelen aan te kunnen maken die op verschillende pagina's getoond kunnen worden
  \item \ldots
\end{itemize}

\section{Waarom Drupal}
\begin{itemize}
  \item Vele functies en uitbreidingsmogelijkheden.
  \item Effici\"ent en effectief: met relatief weinig moeite kan u veel
  bereiken. Drupal gebruiken bespaart tijd. Er hoeft niet steeds vanaf nul te worden ontwikkeld.
  \item Het uiterlijk is flexibel aan te passen. Hiervoor zijn vele 'Themes'
  beschikbaar. 'Themes' zijn eenvoudig en eindeloos aan te passen aan specifieke wensen.
  \item Grote actieve community, plus een groot netwerk van service-aanbieders
  geven vertrouwen in de continu\"\i teit van Drupal.
\end{itemize}

\subsection{Voor managers}
Door de vele functies en uitbreidingsmogelijkheden is Drupal bruikbaar voor
uiteenlopende doeleinden. Van marketing-inspanningen tot intern communicatiegebruik. 
Voor informerende websites, maar ook voor communicatie tussen verschillende groepen. 
Drupal is makkelijk uitbreidbaar. Als bepaalde functionaliteit nog niet beschikbaar is, 
dan zijn er veel mogelijkheden om ontwikkelaars in te huren om een of meerdere extra 
modules te bouwen. Er is een grote en gedreven community met diverse ervaren leden 
die u graag van dienst zijn. De actieve community en een groot netwerk van 
service-aanbieders geven de vele gebruikers bovendien vertrouwen in de
continu\"\i teit van Drupal. Drupal is ook vlot te beheren. Het
toegangscontrolesysteem kan eenvoudig worden aangepast aan uw organisatiestructuur. 
Zodat iedereen precies de nodige toegang heeft.

\subsection{Voor beheerders}
Het staat bekend als effici\"ent en effectief: met relatief weinig moeite kan
u veel bereiken. Drupal biedt gebruikers een eenvoudig te bedienen
gebruikersinterface, die ook flexibel en uitbreidbaar. Zo kan u
verschillende typen inhoud beheren: tekst, afbeeldingen, video, kalenders, enqu\^etes, forums,
\ldots De uitbreidingsmogelijkheden maken het eenvoudig om snel nieuwe inhoudstypen toe
te voegen en daarmee de functionaliteit te verbeteren.

\subsection{Voor ontwikkelaars}
Website-ontwikkelaars waarderen Drupal omdat het hen toelaat effici\"ent te
werken. Drupal gebruiken bespaart tijd. Er hoeft niet steeds vanaf nul te worden ontwikkeld. 
Omdat zoveel mensen aan Drupal bijdragen, bevat het goede API's. De grote verzameling van 
modules wordt voortdurend uitgebreid en bijgewerkt. De Drupal-community is een grote groep 
van vrijwilligers die met plezier informatie uitwisselen om een klus te klaren.

\subsection{Voor grafische vormgevers}
Het uiterlijk van een Drupal-website is flexibel aan te passen. Hiervoor zijn
vele 'themes' beschikbaar. 'Themes' zijn eenvoudig en eindeloos aan te passen aan specifieke wensen.


\section{PHP}
PHP \index{server-side scriptingtaal - PHP}is een server-side scriptingtaal, die
hoofdzakelijk wordt gebruikt om op de webserver dynamische webpagina's te cre\"eren, is voor de gebruiker
onzichtbaar. Alleen het resultaat is zichtbaar en dat ziet eruit als een gewone HTML-pagina. 
De voordelen van PHP zijn: het is open source, het is zeer populair en makkelijk 
te leren en er is een brede ondersteuning mogelijk op het internet.
\subsection{Geschiedenis}
PHP werd in 1994 ontwikkeld door Rasmus Lerdorf. De eerste publieke versie werd uitgegeven in 1995, 
alsook versie 2. Zeev Suraski en Andi Gutmans, twee Isra\"elische ontwikkelaars
aan de Technion IIT, herschreven de parser in 1997 en vormden de basis voor PHP 3 en veranderde hiermee de naam in PHP: 
Hypertext Preprocessor. Het ontwikkelteam bracht PHP/FI 2 officieel in November 1997 uit, na maanden van beta-tests. 
Hierna begon de publieke test van PHP 3 en in juni 1998 werd PHP 3 officieel uitgebracht. Suraski en Gutmans begonnen 
hierna met het herschrijven van de PHP parser, met de Zend Engine in 1999 als resultaat. Hiermee werd Zend Technologies 
opgericht in Ramat Gan, Isra\"el.
Op 22 mei 2000 werd PHP 4, aangedreven door Zend Engine 1.0, uitgebracht. Op 13 juli 2004 werd PHP 5 uitgebracht, 
aangedreven door de nieuwe Zend Engine II.
Ondanks dat PHP 5 al sinds 2004 beschikbaar is, gebruiken veel webservers pas sinds begin 2007 PHP5, omdat eerdere versies 
niet stabiel genoeg waren. De meest recente stabiele versie is 5.3.1 (19 november 2009). In deze versie zijn er ook veel 
bug-fixes gedaan. De belangrijkste kenmerken van PHP 5 zijn het verbeterde objectgeori\"enteerd programmeren, de hogere 
snelheid, de mogelijkheid om SQLite aan te spreken en de vernieuwde XML-bibliotheek.
\subsection{Gebruik}
PHP wordt veel gebruikt om op webservers dynamische webpagina's te
cre\"eren. Andere bekende server-side scripttalen zijn Java Server Pages (JSP), Coldfusion en Active Server Pages (ASP). De code van de pagina wordt op de server uitgevoerd, 
en het resultaat wordt naar de computer van de bezoeker gestuurd en in de webbrowser getoond. Dit in tegenstelling tot 
client-side scripting (zoals Javascript), waarbij de webbrowser eerst de pagina van de webserver downloadt en vervolgens 
zelf (op de computer van de bezoeker) code uitvoert.
\\
Bij het oproepen van een PHP-document op de server wordt (op de server) eerst de in het document opgenomen PHP-code uitgevoerd. 
Dit gebeurt door de PHP-parser (de PHP-engine). Het resultaat (meestal HTML) wordt door de webserver naar de browser gestuurd. 
PHP kan echter ook andere documenttypen versturen. PHP-documenten hebben meestal de extensie .php, maar ook de oudere extensies 
worden nog (weliswaar sporadisch) gebruikt.
\\
PHP ondersteunt ook diverse extensies die (in de Windows-versie) als een simpele DLL kunnen worden geactiveerd, om daarna het 
php.ini aan te passen. Alle documentatie is in de PHP-handleiding te vinden. Onder andere door de gemakkelijk bereikbare 
documentatie (centraal op een locatie) is PHP populair geworden onder webprogrammeurs.
\\
PHP wordt zeer veel gebruikt in combinatie met Linux, Apache en MySQL, afgekort tot LAMP. De LAMP-architectuur is zeer succesvol 
op het internet. Het komt ook wel eens voor dat men Windows gebruikt in plaats van Linux. WAMP is hierbij de afkorting voor 
systemen die Windows gebruiken en er wordt wel eens de afkorting MAMP gebruikt voor de Macintosh. Ook zijn er kant en klare 
programma's die een volledige WAMP omgeving installeren. Voorbeelden hiervan zijn WAMP en XAMPP.

\subsection{Voorbeeld}
\begin{verbatim}
<?php
   $drupaltekst = "Dit is GEEN Drupal website";
   echo $drupaltekst;
  // hier zal dan uiteindelijk "Dit is GEEN Drupal website" op de site komen te staan
?>
\end{verbatim}

\subsection{OOP \index{Object Oriented programming}}
PHP wordt vanwege het lage instapniveau gezien als een van de makkelijkste webtalen en voorziet tegelijk in grote 
doorgroeimogelijkheden. Zo is het met PHP ook mogelijk objectgeori\"enteerd (OO,Object Oriented) te programmeren. 
Bij OO-programmeren (OOP) maakt men klassen van waaruit weer objecten gemaakt kunnen worden. De klassen zijn als het ware 
een recept, een beschrijving van het object. Een bouwplattegrond van een fiets is vergelijkbaar met een klasse en de fiets 
zelf is vergelijkbaar met een object. In de klasse zijn de onderdelen van de fiets beschreven (properties, bijv. wielen, 
trappers, etc.) en de mogelijkheden van een fiets (methods, bijv. fietsen, remmen, bellen, licht aandoen, op slot doen). 
Van een klasse kunnen dus verscheidene objecten (zij het met verschillende parameters) worden gemaakt. Zo zou je met 
dezelfde onderdelen bijvoorbeeld ook een ligfiets of een driewieler kunnen maken. Of tien soortgelijke fietsen met 
allemaal een verschillende kleur.

\section{SQL}
SQL \index{Structured Query Language} of Structured Query Language is een ANSI/ISO-standaardtaal voor
een relationeel 'database management systeem' (DBMS). Het is een gestandaardiseerde taal die gebruikt kan worden 
voor taken zoals het bevragen en het aanpassen van informatie in een relationele databank. SQL kan met vrijwel 
alle moderne relationele databankproducten worden gebruikt.

\subsection{geschiedenis}
SQL is gebaseerd op de relationele algebra en werd in de loop van de jaren
zeventig ontwikkeld door IBM (San Jos\'e). Sinds het ontstaan van SQL hebben
reeds vele verschillende SQL-versies het levenslicht gezien. Pas in de loop van de jaren 80 werd SQL gestandaardiseerd. Tegenwoordig gebruiken de meeste 
'Database Management Systems' SQL-92.
\\
In eerste instantie werd SQL ontwikkeld als een vraagtaal voor de eindgebruiker.
Het idee was dat businessmanagers SQL zouden gaan gebruiken om bedrijfgegevens te analyseren. 
Achteraf is gebleken dat SQL te complex is om door eindgebruikers toegepast te worden. 
Het gebruik van SQL impliceert immers een volledige kennis van de structuur van de te ondervragen databank. 
Tegenwoordig wordt SQL vrijwel uitsluitend door tussenkomst van een applicatie gebruikt. De programmeur van 
de applicatie benadert de database met SQL via een Application Programming Interface (API), zoals ODBC of 
ADO (MS Windows), JDBC (Java) of een productspecifieke API. SQL is dus in essentie omgevormd van een taal voor 
eindgebruikers tot een brug tussen applicaties en databanken.
\subsection{Werking}
SQL maakt voor de communicatie met het DBMS gebruik van zogenaamde query's. Een
query \index{query}is een ASCII-tekenreeks en bevat telkens een opdracht die
naar het databasemanagementsysteem (DBMS) \index{databasemanagementsysteem}
wordt verzonden. Het DBMS zal op zijn beurt die opdracht interpreteren en uitvoeren en stuurt, indien nodig, een aantal gegevens terug naar de opdrachtgever. \\Een SQL-query ziet er bijvoorbeeld als volgt uit:
\begin{verbatim}
SELECT *
  FROM tblleerlingen 
  WHERE tblleerlingen.tebetalen > 0;
\end{verbatim}
De betekenis van bovenstaande query is als volgt:\\
- SELECT: hierachter wordt geplaatst welke velden worden
geselecteerd; * is allevelden. \\ - FROM: hierachter komt de naam
van de tabel, in dit geval tblleerlingen. \\ - WHERE: hierachter komen veldnamen met waarden waaraan de velden
moeten voldoen. \\ In dit geval: alle records waarvan het veld tebetalen in de tabel tblleerlingen groter is dan 0.\\
Dit is een van de simpelste vormen die een query kan aannemen. Met SQL is het
mogelijk om tabellen aan te maken, te wijzigen, te vullen en te verwijderen. 
\subsection{MySQL}
MySQL \index{MySQL}is een open source relationele databasemanagementsysteem
(RDBMS), dat gebruikmaakt van SQL, en word vooral gebruikt voor toepassingen zoals fora,
blogs,cms, enz. en dit meestal in combinatie met PHP. Tegenwoordig is het
de basis van een breed scala aan internettoepassingen.
MySQL-software bestaat onder meer uit een serverprogramma, doorgaans
mysqld genoemd. Verder bestaat het uit een verzameling clientprogramma's, zoals
mysql en mysqldump waarmee automatisch of interactief met de server gecommuniceerd kan worden.
Een bekend MySQL-frontend is phpMyAdmin, een webgebaseerd MySQL-beheerprogramma
geschreven in PHP.


\section{Fedora}
Fedora is een op Linux gebaseerd besturingssysteem dat het laatste in vrije en
open software naar voren wil brengen. Fedora is altijd vrij voor iedereen om te gebruiken, 
aan te passen en te herdistribueren. Het wordt wereldwijd ontwikkeld door een grote gemeenschap 
van mensen: het Fedora Project. Het Fedora Project is open en iedereen is welkom om eraan deel te nemen.
\subsection{Waarom Fedora 12} 
Fedora: proeftuin van Red Hat \footnote{Interview met Paul Frields, CEO
Fedora, 2 nov 2009}, is in feite het R\&D-lab van Red Hat, maar het heeft die
functie ook voor de hele Linux-gemeenschap. Op een gegeven moment kijkt Red Hat naar al die Fedora-releases en besluit om de 
aantrekkelijke features in de volgende versie van Linux te stoppen. Fedora 12 is bijvoorbeeld een goede indicator van hoe RHEL 6 eruit zal zien. 

\section{Drupal druppelt het Witte Huis binnen } 
\footnote {artikel van Stefan Grommen, datanews}
Drupal, het opensource contentmanagementsysteem met de Belg Dries Buytaert
\index{Dries Buytaert} als geestelijke vader, is voortaan het cms voor
WhiteHouse.gov, de offici\"ele website van de administratie van de Amerikaanse president Barack Obama.
\\
WhiteHouse.gov draait voortaan op het opensource cms Drupal, zo meldt Drupal-bezieler 
Dries Buytaert op z'n weblog. Sinds de periode van Bush werd een (niet nader genoemd) 
commercieel cms gebruikt. De overstap komt er omdat de administratie van Obama naar 
eigen zeggen een eenvoudiger te hanteren omgeving nodig had voor de webactiviteiten 
van het Witte Huis. Dat om Obama's visie van 'interactieve overheid' sterker te kunnen uitstralen.
\\
Het Amerikaanse bedrijf General Dynamics Information Technology ging 
op zoek naar een alternatief systeem en kwam uit bij Drupal. Acquia, 
het bedrijfje van Dries Buytaert dat diensten levert rond Drupal, 
is een van de onderaannemers, naast ook Phase2, Terremark Federal Group en Akamai.
\\
Op z'n blog zegt Buytaert dat het een duidelijk teken is dat overheden 
zich realiseren dat open source geen bijkomende risico's stelt in 
vergelijking met commerci\"ele software. En dat ze voorts door commerci\"ele 
software links te laten liggen niet ingesloten zijn door een bepaalde 
technologie. En dat ze kunnen profiteren van innovatie die het resultaat 
is van duizenden ontwikkelaars die samenwerken op Drupal.
\\
Hoewel een van de belangrijkste tot dusver, is het niet het eerste 
overheidsdepartement van de VS dat voor Drupal kiest. Zo zijn onder meer 
het Department of Defense, Commerce en Education en de General Service 
Administration Drupal-gebruikers.

\section{GNU General Public License} \index{GNU General Public Licence}
\footnote {Voor de juridisch geldige teksten van de GPL-licentie kan men terecht op de site van GNU}
De GNU General Public License of kortweg de GPL is een copyleftlicentie voor software, 
bedacht door Richard M. Stallman, die (in het kort) stelt dat je met de software mag doen 
wat je wil (inclusief aanpassen en verkopen), mits je dat recht ook doorgeeft aan anderen 
en de auteur(s) van de software vermeldt. Concreet komt dat er op neer dat als je 
software die onder de GPL is gepubliceerd wilt verspreiden, je daar de broncode bij 
zult moeten doen. Deze broncode mag dan weer verder worden verspreid onder de GPL. 
Iedereen kan ervoor kiezen zijn of haar programma onder de voorwaarden van deze licentie te publiceren.
\\
Software die onder deze licentie wordt uitgegeven is vrij. Vaak wordt dit
verkeerd ge\"interpreteerd als gratis software, aangezien het Engelse woord voor
vrij (free) ook gratis betekent. Met prijzen heeft de licentie echter weinig te maken: 
het gaat over rechten. Wel is het zo dat praktisch alle vrije software gratis te 
downloaden is en als men er toch voor moet betalen, men het recht heeft om de software 
zelf weg te geven of zelfs door te verkopen.
\\
De GNU Lesser General Public License (of kortweg de LGPL) \index{LGPL} is een
afgezwakte versie van de GPL die soepeler omgaat met het gebruik van software in software met een andere licentie. 
Deze verschilt met de GPL op het punt dat software die gebruik maakt (als bibliotheek bijvoorbeeld) 
van LGPL-gelicenseerde software, zelf niet onder de LGPL hoeft te worden vrijgegeven.

\section{De Drupal Association}
 De Drupal Association werd in 2006 opgericht om te helpen met het beheer van de
 infrastructuur, fondsenwerving en promotie van het Drupal-project. Drupal wordt gedragen door een 
 gedreven community van ontwikkelaars, gebruikers, designers,
 documentatieschrijvers, enz.\\
\\
Om de verdere groei van het project te ondersteunen, werd de Drupal Association
opgericht. De Drupal Association is een non-profit-organisatie, geregistreerd in
Belgi\"e (waar Drupal oprichter Dries Buytaert woont). De Drupal Association is
er niet om te bepalen hoe de volgende versie van Drupal eruit zal zien: de ontwikkeling van Drupal is en blijft uitsluitend 
in handen van de community. De Drupal Association kan wel giften en werkbeurzen ontvangen, 
events organiseren en/of sponsoren, infrastructuur (bv. servers) aankopen en beheren ter ondersteuning 
van het Drupal-project, persberichten publiceren, enz.\\
\\
Meer informatie over de Drupal Association vind je op:
http://association.drupal.org.
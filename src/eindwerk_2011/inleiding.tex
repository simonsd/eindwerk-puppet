\chapter{Lenovo ThinkPad T410}

The ThinkPad T410 is the latest revision of the popular T-series ThinkPad from
Lenovo. This model brings a new line of dedicated and integrated graphics cards,
the Intel Core-series processor line, and a completely redesigned chassis.
We took an in-depth look at the highly anticipated T410 to see how well it stacks
up against all the prior T-series ThinkPads. Does it live up to our expectations?

\section{Lenovo ThinkPad T410 with NVIDIA Graphics Specifications:}
the ThinkPad T410 Laptop Computer with discrete graphics
\begin{itemize}
  \item Processor: Intel Core i7-620M Processor (2.66GHz, 4MB L3, 1066MHz FSB)
  \item Operating System: Scientific Linux 6 (64bit)
  \item Display type:	14.1 WXGA+ TFT, w/ LED Backlight (WWAN antenna)
  \item System graphics:	NVIDIA NVS 3100m Optimus Graphics 512MB DDR3 with AMT
  \item Modem:	56K v.92 Designed Modem
  \item Total memory:	8 GB PC3-10600 DDR3 SDRAM 1333MHz SODIMM Memory (2 DIMM)
  \item Keyboards:	Keyboard BE
  \item Pointing device:	UltraNav (TrackPoint and TouchPad)
  \item Hard drive:	128 GB Solid State Drive, Serial ATA
  \item Optical device:	DVD Recordable 8x Max Dual Layer
  \item System expansion slot:	Smart Card
  \item Battery:	9 cell 2.8Ah Li-Ion Battery
  \item Bluetooth:	Bluetooth w/ antenna
  \item Integrated WiFi:	Intel Centrino Ultimate-N 6300 (3x3 AGN)
\end{itemize}
Accessories and options:
\begin{itemize}
  \item 3YR Onsite
  \item Lenovo 90W Ultraslim AC/DC Combo Adapter
  \item ThinkPad Bluetooth Laser Mouse
  \item ThinkPad Business Backpack
  \item ThinkPad 14W Sleeve Case
  \item Kensington MicroSaver Security Cable Lock
  \item Direct connect adapter - Hi-Speed USB - USB
  \item USB 2.0 Portable 500GB Hard Drive
\end{itemize}

\section{Why Thinkpad}
For business users whose work involves intense data manipulation, the ThinkPad
T410 laptop delivers powerful performance in a portable form.

\section{Why Scientific Linux}
SL is a Linux release put together by Fermilab, CERN, and various other labs and universities around the world. Its primary purpose is to reduce duplicated effort
of the labs, and to have a common install base for the various experimenters.The base SL distribution is basically Enterprise Linux, recompiled from source.

Our main goal for the base distribution is to have everything compatible with Enterprise, with only a few minor additions or changes. An example of of items that
 were added are Alpine, and OpenAFS.

Our secondary goal is to allow easy customization for a site, without disturbing the Scientific Linux base. The various labs are able to add their own modifications
to their own site areas. By the magic of scripts, and the anaconda installer, each site is to be able to create their own distributions with minimal effort. Or, if a
 users wishes, they can simply install the base SL release.
\documentclass{beamer}
\usepackage[latin1]{inputenc}
\usepackage[dutch]{babel}
\usepackage{graphicx}
\usepackage[T1]{fontenc}
\usepackage{textcomp}
\usepackage{enumerate}
\usepackage{tikz}
\mode<presentation>
{
	\usetheme{Warsaw}
	\setbeamercovered{transparent}
}
\begin{document}

%------------------------
\title{Configuration Management met Puppet}
\author[D. Simons]{Dave Simons}
\institute{Don Bosco Wilrijk}
\date{2011}
\maketitle
%\begin{picture}(0,0)
%	\put(20,30){\includegraphics[scale=0.5]{src/puppet_logo.png}}
%\end{picture}
%------------------------


%------------------------
\begin{frame}
\frametitle{Inleiding}
\begin{columns}[c]
\column{.37\textwidth}
\begin{block}{Concept}
\begin{itemize}
	\item centraal beheer
	\item automatisatie
	\item herbruikbaar
	\item schaalbaar
\end{itemize}
\end{block}
\begin{block}{Voorbeelden}
\begin{itemize}
	\item Cfengine
	\item Bcfg2
	\item Puppet
	\item Chef
\end{itemize}
\end{block}
\end{columns}
\end{frame}
%------------------------

%------------------------
\begin{frame}
\frametitle{Puppet}
\begin{columns}[c]
\column{.65\textwidth}
\begin{block}{Ontwerp}
\begin{itemize}
	\item geschreven in Ruby
	\item declaratieve DSL
	\item client-server model
	\item andere implementaties mogelijk (Git)
\end{itemize}
\end{block}
%\column{.4\textwidth}
\begin{block}{Onderdelen}
\begin{itemize}
	\item puppet master
	\item puppet agent
	\item puppet apply
	\item puppet resource
	\item Facter
\end{itemize}
\end{block}
\end{columns}
\end{frame}
%------------------------

%------------------------
\begin{frame}
\frametitle{Resource Types}
\begin{columns}[t]
\column{.5\textwidth}
\begin{block}{}
\begin{itemize}
	\item bouwstenen voor manifests
	\item voorgedefini\"eerd
	\item veel keuze
	\item uitbreidbaar
\end{itemize}
\end{block}
\end{columns}
\begin{block}{Voorbeelden}
\begin{columns}[b]
\column{.3\textwidth}
\begin{itemize}
	\item cron
	\item exec
	\item file
	\item group
\end{itemize}
\column{.3\textwidth}
\begin{itemize}
	\item host
	\item mount
	\item notify
	\item package
\end{itemize}
\column{.3\textwidth}
\begin{itemize}
	\item service
	\item stage
	\item user
	\item yumrepo
\end{itemize}
\end{columns}
\end{block}
\end{frame}
%------------------------

%------------------------
\begin{frame}
\frametitle{Uitgebreide mogelijkheden}
\begin{columns}[c]
\column{.5\textwidth}
\begin{block}{Templates}
\begin{itemize}
	\item ERB templating system
	\item variabelen
	\item flow control
	\item standaardwaardes
\end{itemize}
\end{block}
%\column{.5\textwidth}
\begin{block}{Modules}
\begin{itemize}
	\item naamgeving
	\item structuur
	\item zoeken
	\item autoloading
\end{itemize}
\end{block}
\end{columns}
\end{frame}
%------------------------

%------------------------
\begin{frame}
\frametitle{Installatie}
\begin{columns}[c]
\column{.55\textwidth}
\begin{block}{mogelijkheden}
\begin{itemize}
	\item source
	\item gem
	\item distro packages
\end{itemize}
\end{block}
\begin{block}{configuratie}
\begin{itemize}
	\item gewone tekst-bestanden
	\item centrale configuratie
	\item beheerbaar
	\item command line parameters
\end{itemize}
\end{block}
\end{columns}
\end{frame}
%------------------------

%------------------------
\begin{frame}
\frametitle{Uitvoering}
\begin{columns}[c]
\column{.805\textwidth}
%\column{.33\textwidth}
%\begin{block}{puppet agent}
%\begin{itemize}
	%\item automatisch
	%\item achtergrond
	%\item geheugengebruik
%\end{itemize}
%\end{block}
%\column{.33\textwidth}
%\begin{block}{cron job}
%\begin{itemize}
	%\item automatisch
	%\item geen RAM-gebruik
	%\item snel vergeten
%\end{itemize}
%\end{block}
%\column{.33\textwidth}
%\begin{block}{manueel}
%\begin{itemize}
	%\item niet automatisch
	%\item handig voor testen
	%\item geen daemon nodig
%\end{itemize}
%\end{block}
\begin{block}{Mogelijkheden}
\begin{tabular}{|r|c|c|}\hline
& pro & con \\ \hline
\hline
puppet agent & automatisch & geheugengebruik \\
& daemon & \\
& & \\
\hline
cron job & automatisch & snel vergeten \\
& geheugengebruik & \\
& & \\
\hline
manueel & handig bij testen & niet automatisch \\
& geen daemon & \\
& geheugengebruik & \\
& & \\
\hline
\end{tabular}
\end{block}
\end{columns}
\end{frame}
%------------------------

\end{document}

\chapter*{Voorwoord}

%--------------------------
% old
%--------------------------
%Ik volg de opleiding netwerkbeheerder bij het centrum voor deeltijds onderwijs Don Bosco te Wilrijk. Dit eindwerk is daarbij een onderdeel van mijn opdrachten om het certificaat netwerkbeheerder te behalen.\\
%Mijn opleiding bestaat uit drie grote onderdelen, namelijk: Praktijk, Theorie en ASPV.\\[0.3cm]
%Bij praktijk is ons leerdoel om zoveel mogelijk praktische kennis te verzamelen over het beheren van netwerken, zoals het opzetten van een server waarmee je verschillende services kan aanbieden.\\
%Denk hierbij aan bijvoorbeeld een dhcp/dns-server, webserver en fileserver om er even een paar te noemen.\\
%Bij Theorie is ons leerdoel om zoveel mogelijk theoretische informatie te vergaren, zoals de achterliggende werking van het TCP/IP-protocol, de componenten waaruit een pc word opgebouwd en de elektrische eigenschappen ervan.\\[0.3cm]
%Bij ASPV is ons leerdoel voornamelijk om algemene kennis te verzamelen, denk hierbij vooral aan dagelijkse dingen zoals belastingen, verzekeringen, etc.\\
%Althans dat was het enige doel in het vorige jaar, dit jaar kijken we ook naar web-ontwikkeling, zoals het maken van statische, dynamische en database-gestuurde websites.\\[0.3cm]
%Bij deze zou ik graag van de gelegenheid gebruik maken om enkele mensen te bedanken:\\
%Eerst en vooral: Robert Keersse: omdat hij een geweldige leerkracht is en mij op het goede (lees: open-source) pad heeft gezet. Zonder zijn steun was me dat nooit gelukt.\\[0.3cm]
%Luc jennes: Voor het bijdragen aan mijn theoretische kennis over personal computers.\\[0.3cm]
%Peter Peeters: Voor het bijbrengen van enkele belangrijke levenslessen, het bouwen van websites en voor het algemeen beheer in ons klaslokaal.\\[0.3cm]
%
%
%--------------------------
% mitchell
%--------------------------
% Allereerst wil ik mijn dank uitspreken over mijn begeleiders, Marc Decroos, Bram en Dirk bij Terbeke-Pluma. Niet alleen heb ik van hen ongeloofelijk veel technische kennis opgedaan, maar ook leren werken, en zeker, werk leren zien.
%
% Graag wil ik mijn leerkrachten bedanken voor hun volharding en steun. De heer Peeters, waar ik van heb ``leren lezen en schrijven'' en in bijzonder dat ik mocht proeven van zijn kennis over PHP en SQL. Wat me het meest bijblijft van de theorie lessen gegeven door de heer Jennes, is dat deze meestal ver boven mijn petje gingen, nochtans begrijp ik nu wel dat je de lat nooit hoog genoeg kan leggen.\\\\
%
% Mijn eindwerk over Drupal is anders van structuur dan gevraagd, dit komt hoofdzakelijk omdat het gebruikelijke ``theorie - praktijk'' hoofdstuk voor mij niet direct duidelijk was. Na voldoende argumenten stemde de heer Keersse ermee in de structuur te wijzigen. Deze bestaat uit ``inhoud aanmaken'' en ``beheren''. De praktische installatie kan je terugvinden in de bijlage.\\\\
% Alle hoofdstukken, titels en ondertitels, zijn hetzelfde als gebruikt in de Nederlandse vertaling van Drupal, dit maakt het eenvoudig te zoeken in de help functie van de website.
%Het is niet mijn bedoeling dat dit eindwerkje een volledige beschrijving geeft wat je allemaal met Drupal kan, het kan misschien dienen als startpunt voor nieuwe gebruikers.
%
%
%--------------------------
% new
%--------------------------
Allereerst zou ik graag mijn begeleiders, Kris Buytaert en alle leden van het Inuits team, bedanken. Niet alleen heb ik van hen een kans gekregen om deze leuke job uit te oefenen, ze hebben me ook al ontelbare dingen bijgeleerd, die zowel binnen als buiten de job nog nuttig zullen blijken.\\\\
Graag zou ik ook mijn leerkrachten bedanken voor hun volharding en steun. De heer Peeters, waar ik van heb ``leren lezen en schrijven'' en in bijzonder dat ik mocht proeven van zijn kennis over PHP en SQL. Wat me het meest bijblijft van de lessen theorie aan de hand van de heer Jennes, is dat deze meestal ver boven mijn petje gingen, nochtans begrijp ik nu wel dat je de lat nooit hoog genoeg kan leggen. Als laatste zou ik graag de heer Rober Keersse bedanken voor alle begeleiding, bij het vinden van een job en in het bijzonder om me op het goede (less: open source) pad te brengen.\\\\
Tot slot zou ik u ook graag bedanken om te besluiten dat dit eindwerk de moeite waard is om door te lezen. Ook zou ik er graag even op wijzen dat dit eindwerk geenszins een complete referentie is, het is enkel een begin vanwaar u uw kennis verder kan uitbreiden. 

\begin{flushright}Dave Simons\\juni 2011\end{flushright}

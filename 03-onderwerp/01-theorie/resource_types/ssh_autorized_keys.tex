ssh_authorized_key

Manages SSH authorized keys. Currently only type 2 keys are supported.

Autorequires: If Puppet is managing the user account in which this SSH key should be installed, the ssh_authorized_key resource will autorequire that user.
Parameters
ensure

The basic property that the resource should be in. Valid values are present, absent.
key

The key itself; generally a long string of hex digits.
name

The SSH key comment. This attribute is currently used as a system-wide primary key and therefore has to be unique.
options

Key options, see sshd(8) for possible values. Multiple values should be specified as an array.
provider

The specific backend for provider to use. You will seldom need to specify this — Puppet will usually discover the appropriate provider for your platform. Available providers are:

    parsed: Parse and generate authorized_keys files for SSH.

target

The absolute filename in which to store the SSH key. This property is optional and should only be used in cases where keys are stored in a non-standard location (i.e. not in ~user/.ssh/authorized_keys`).
type

The encryption type used: ssh-dss or ssh-rsa. Valid values are ssh-dss (also called dsa), ssh-rsa (also called rsa).
user

The user account in which the SSH key should be installed. The resource will automatically depend on this user.

\section{Group}

\subsection{Functie}
Het beheren van groepen, zowel op besturingssysteem niveau als bijvoorbeeld in LDAP (Lightweight Directory Access Protocol). Op sommige besturingssystemen zoals Mac OS X, word lidmaatschap van groepen niet geattribueerd aan gebruikers, maar aan groepen zelf, in dit geval moet de provider ondersteuning bieden voor de parameter 'manages\_members'. In de meeste gevallen is dit echter niet nodig en kan groepslidmaatschap beheerd worden op gebruikersbasis en dus via de 'user' resource type.

\subsection{Parameters}
allowdupe:\\
Duplicate GID (Group ID) definities toestaan. Geldige waardes zijn 'true' en 'false'.\\\\
%
auth\_membership:\\
Hiermee bepaal je of de 'provider' voor groepen ook gebruikt kan worden voor groep lidmaatschap.\\\\
%
ensure:\\
Geldige waardes zijn 'present' en 'absent', waarmee je er respectievelijk voor zorgt dat de groep aan- of afwezig is.\\\\
%
gid:\\
Het gewenste groep-ID. Deze moet numeriek opgegeven worden en zal indien niet opgegeven willekeurig gekozen worden. Dit is meestal niet wenselijk indien je een configuratie op meerdere systemen wil laten werken.\\\\
%
members:\\
De gebruikers die lid zijn van deze groep. Dit is enkel voor systemen waar het groep lidmaatschap bij de groeps-objecten bijgehouden wordt in plaats van bij de gebruikers-objecten. Dit vereist dat de 'provider' de functie 'manages\_members' ondersteund.\\\\
%
name:\\
De gewenste groepsnaam. Alhoewel de aanvaardbare namen verschillen per systeem, wordt meestal aangeraden om de namen te vormen aan de hand van volgende criteria:
\begin{itemize}
\item enkel alfanumerieke karakter
\item maximum acht karakters
\item begint met een kleine letter
\end{itemize}
%
provider:\\
Welke 'provider' je wil gebruiken om groepen aan te maken. In de meeste gevallen zal puppet zelf ontdekken welke provider voor jouw systeem aangewezen is. Beschikbare providers zijn:
\begin{itemize}
\item aix: AIX groepsbeheer
\item directoryservice: Mac OS X DirectoryService beheer
\item groupadd: groepsbeheer via groupadd, beschikbaar op de meeste systemen
\item ldap: LDAP beheer
\item pw: groepsbeheer via het pw commando (werkt enkel op FreeBSD)
\end{itemize}
%
system:\\
Of de groep een lagere GID hoort te krijgen, wat aanduid dat hij een onderdeel van het systeem is en dus niet van een gebruiker. Geldige waardes zijn: 'true' of 'false'.

\section{Exec}
\subsection{Functie}
De 'exec' resource type zorgt ervoor dat je externe commando's kan uitvoeren, je kan er dus elk commando mee uitvoeren dat beschikbaar is op je systeem. Het belangrijkste waarbij je hier moet opletten is dat zo'n commando elke keer wordt uitgevoerd wanneer puppet zijn manifests uitvoert, het is dus belangrijk dat men geen commando's specifi\"{e}ert die elkaar later overschrijven. Uiteraard zijn er ook wel manieren om te zorgen dat een commando slechts wordt uitgevoerd indien aan een bepaalde conditie vvoldaan word. Een andere manier om dit in te perken is de refreshonly parameter, die ervoor zorgt dat een commando enkel wordt uitgevoerd als reactie op een andere resource definitie binnen de manifests. Meer daarover later.\\\\
%
Momenteel word de 'exec' resource zeer veel gebruikt om opdrachten uit te voeren die nog geen standaard resource type binnen puppet hebben. Op korte termijn is dit een fenomeen dat niet te verhelpen valt, omdat de ontwikkelaars onmogelijk voor elk programma een individuele resource type kan gaan ontwikkelen. Op lange termijn is dit dan weer wel mogelijk maar nog steeds niet realistisch. Veel logischer is dat de mensen die graag ondersteuning zouden zien voor een bepaald object, zelf iets ontwikkelen en dit publiek maken. Zo kan de community ervan mee genieten en indien de vraag groot genoeg word, kunnen de ontwikkelaars hierop verder bouwen om zo een ingebouwde resource type te voorzien.\\\\
%
Je zal vaak bepaalde software-pakketten nodig hebben om commando's uit te voeren, en daar houd puppet rekening mee: puppet zal indien je in een commando een programma aanroept, kijken of dat programma door jou beheerd wordt en dit automatisch 'requiren'. Dit wil zeggen dat puppet bij het dynamisch opstellen van de volgorde waarin alles uitgevoerd word, rekening zal houden met het feit dat dit programma eerst ge\"installeerd moet worden en dan pas uitgevoerd kan worden. Hetzelfde wordt gedaan voor bijvoorbeeld de gebruiker waarmee men het commando uitvoert: als deze door puppet beheerd wordt en nog niet bestaat, zorgt men ervoor dat deze gecre\"{e}erd wordt prior het commando uit te voeren. Deze functionaliteit noemt men ook wel 'auto-require'.\\\\
%
Men kan ofwel de programma's aanroepen via hun absolute pad of een 'PATH' variable opgeven, die de paden bevat waarin puppet moet zoeken om het commando te vinden.\\
Als alles goed gaat en het commando uitgevoerd wordt zonder fouten zal de output van het commando op het normale log-level gelogd worden, indien er een fout optreed zal puppet echter een melding weergeven en de eventuele output voordat de fout optrad.\\
%
\subsection{Parameters}
command:\\
Het commando dat je wil uitvoeren.\\\\
%
creates:\\
Hiermee kan je het pad naar een bestand opgeven dat aangemaakt moet worden. Indien het bestand bestaat zal puppet niets doen, anders voert hij de opdracht gewoon uit.
%
cwd:\\
De map waarin het commando dient uitgevoerd te worden, als de map niet bestaat zal het commando falen.\\\\
%
environment:\\
Met deze parameter kan je omgevingsvariabelen meegeven aan het commando.\\\\
%
group:\\
De groep die gebruikt dient te worden om het commando uit te voeren.\\\\
%
logoutput:\\
Of je de uitvoer van het opgegeven commando wil loggen. Je kan als waarde 'on\_failure' opgeven om ervoor te zorgen dat er enkel gelogd word als er fouten optreden.\\\\
%
onlyif:\\
Je kan hiermee een test opgeven, enkel als de test slaagt zal het commando dan uitgevoerd worden. Als parameter dien je ook een commando op te geven.\\\\
%
path:\\
Dit is de 'PATH' omgevingsvariable die gebruikt zal worden om te zoeken naar programma's, indien deze niet aanwezig is dient men bij een commando steeds de volledige padnaam naar een programma op te geven.\\\\
%
provider:\\
Met welke provider je commando's wil uitvoeren: rechtstreeks (posix) of via een shell zodat de ingebouwde commando's van de shell beschikbaar zijn.\\\\
%
refresh:\\
Een alternatief commando om uit te voeren indien dit object word uitgevoerd via een ander object.\\\\
%
refreshonly:\\
Hiermee word het commando enkel uitgevoerd als het word uitgevoerd via een ander object.\\\\
%
returns:\\
Geef op welke 'return codes' aanvaardbaar zijn. Standaard is dit enkel '0', indien het commando dan eindigt met een status niet gelijk is aan '0', zal puppet denken dat het commando gefaald heeft. Met dit gedrag kan je bijvoorbeeld bepaalde fouten filteren.\\\\
%
timeout:\\
De maximale tijdsduur die het commando in beslag zou mogen nemen, indien deze tijd overschreden word gaat puppet ervan uit dat het commando gefaalt heeft en zal het het uitvoeren onderbreken.\\\\
%
tries:\\
Het maximaal aantal keren puppet zal proberen het commando opnieuw uit te voeren in geval het faalt, met als standaardwaarde \'e\'en keer. Let er op dat de 'timeout' parameter geld voor elke 'try' apart, niet alle tries samen.\\\\
%
try\_sleep:\\
De pauze tussen opeenvolgende 'tries'.\\\\
%
unless:\\
Het omgekeerde van onlyif, in de zin dat het exec object enkel uitgevoerd zal worden indien dit commando faalt.\\\\
%
user:\\
De gebruiker waarmee je dit commando wil uitvoeren.\\\\

\subsection{Voorbeelden}
Enkele voorbeelden om het gebruik van de 'exec' resource duidelijk te maken.\\\\
%
Dit voorbeeld laat zien hoe je een unless parameter specifi\"{e}ert. Het exec object dat in dit voorbeeld gehanteerd wordt zal root toevoegen aan het '/usr/lib/cron/cron.allow' tenzij het woord 'root' er al in teruggevonden werd.
%
\begin{code}
\begin{lstlisting}
exec \{ "/bin/echo root >> /usr/lib/cron/cron.allow":
	path => "/usr/bin:/usr/sbin:/bin",
	unless => "grep root /usr/lib/cron/cron.allow 2>/dev/null"
\}
\end{lstlisting}
\end{code}
%
In dit voorbeeld wordt het commando 'tar xf /my/tar/file.tar' uitgevoerd. Dit commando pakt een archief uit dat gecompresseerd is volgens het tar-protocol, in de map /var/tmp. De inhoud van dit archief zal een bestand 'myfile' bevatten, en wanneer dit uitgepakt wordt zal dit bestand zich in de map '/var/tmp' bevinden. Doordat puppet bij de volgende run gaat kijken of dit bestand bestaat zal het commando slechts \'e\'en keer uitgevoerd worden, tenzij dit bestand verwijderd of verplaatst word.\\\\
%
\begin{code}
\begin{lstlisting}
exec { "tar xf /my/tar/file.tar":
	cwd => "/var/tmp",
	creates => "/var/tmp/myfile",
	path => ["/usr/bin", "/usr/sbin"]
}
\end{lstlisting}
\end{code}
%
Dit voorbeeld voert het commando logrotate uit, dat hij zal gaan zoeken in '/usr/bin:/usr/sbin:/bin', enkel als 'test `du /var/log/messages|cut -f1` -gt 100000' waar is.
Dit wil zeggen dat dit commando kijkt of de centrale logmap groter is dan 100Mb en op basis daarvan beslissen of hij de logs zal omwisselen of niet.
%
\begin{code}
\begin{lstlisting}
exec \{ "logrotate":
	path => "/usr/bin:/usr/sbin:/bin",
	onlyif => "test `du /var/log/messages | cut -f1` -gt 100000"
\}
\end{lstlisting}
\end{code}
%
Hier zie je hoe je de 'exec' resource enkel zal uitvoeren indien je het object '/etc/aliases' veranderd, door gebruik te maken van de 'subscribe' en 'refreshonly' parameters.\\\\
%
\begin{code}
\begin{lstlisting}
file { "/etc/aliases":
	source => "puppet://server/module/aliases"
}

exec { newaliases:
	path => ["/usr/bin", "/usr/sbin"],
	subscribe => File["/etc/aliases"],
	refreshonly => true
}
\end{lstlisting}
\end{code}

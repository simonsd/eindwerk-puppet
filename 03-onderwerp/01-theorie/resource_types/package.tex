\section{Package}

Het beheren van software-pakketten (voortaan vermeld als 'packages'). Puppet zal automatisch proberen te raden welk type packages je wilt installeren op basis van het gebruikte besturingssysteem. Afhankelijk daarvan zal dan ook een package manager gekozen worden. Ook hier maakt puppet gebruik van zijn 'auto-require magic', zodat configuratiebestanden automatisch ge-'required' worden bij het installeren van de package.

\subsection{Parameters}
%
adminfile:\\
Een bestand dat de standaard-instellingen bevat voor het installeren van packages. Dit wordt momenteel enkel gebruikt op het Solaris platform.\\\\
%
category:\\
Een alleen-lezen parameter die de categorie van het pakket aanduid.\\\\
%
configfiles:\\
Geldige waardes zijn 'keep' of 'replace', welke respectievelijk zorgen dat configuratiebestanden bijgehouden of vervangen worden.\\\\
%
description:\\
Een alleen-lezen parameter voor de beschrijving van een package.\\\\
%
ensure:\\
De status waarin de package zich dient te bevinden. Geldige waardes zijn:
\begin{itemize}
\item latest: zorg ervoor dat de nieuwste versie van een package ge\"installeerd is
\item present or installed: zorg ervoor dat het pakket aanwezig is, ongeacht welke versie
\item purged: zorg ervoor dat het pakket afwezig is en dat de configuratiebestanden verwijdert worden
\item absent: zorg ervoor dat het pakket afwezig is
\end{itemize}
%
name:\\
De naam van het pakket. Deze naam wordt door de package manager intern gebruikt, welke in sommige gevallen (Solaris) niet echt bruikbaar is voor mensen. Meestal wordt ook een alias opgegeven om de namen toegankelijk te houden.
%
platform:\\
Een alleen-lezen parameter waarmee het platform waarvoor het pakket is ontwikkeld aangeduid wordt.\\\\
%
provider:\\
De provider die je wil gebruiken om packages te beheren. Meestal zal puppet dit zelf instellen op basis van het platform. Beschikbare providers zijn:
\begin{itemize}
\item aix: installatie op het AIX besturingssysteem
\item apple: installatie mits de ingebouwde package manager in Mac OS X
\item apt: package management met apt-get
\item aptitude: package management met aptitude
\item aptrpm: package management van RPM packages via apt-get
\item blastwave: blastwave.org's pkg-get package manager voor Solaris
\item dpkg: package management met dpkg
\item fink: package management met fink
\item freebsd: FreeBSD's ingebouwde package manager
\item gem: installatie van Ruby Gems
\item hpux: HP-UX's ingebouwde package manager
\item macports: package management met MacPorts op OS X.
\item openbsd: OpenBSD's ingebouwde package manager
\item pip: Python packages (eggs) via pip
\item pkg: OpenSolaris's ingebouwde package manager
\item pkgdmg: DMG installatie met Apple's Installer.app and DiskUtility.app
\item pkgutil: pkgutil package management voor Solaris
\item portage: Gentoo's portage package manager
\item ports: FreeBSD's ports installatie
\item portupgrade: FreeBDS's portupgrade package manager
\item rpm: RPM package manager; werkt op elk systeem met een werkende RPM binary
\item rug: SuSE's rug package manager
\item sun: Sun's standaard package manager
\item sunfreeware: Sunfreeware.com's pkg-get package manager voor Solaris
\item up2date: Red Hat's up2date package manager
\item urpmi: urpmi wrapper voor rpm op Mandriva
\item yum: yum wrapper voor rpm, vooral gebruikt op Fedora, CentOS en RedHat
\item zypper: SuSE's zypper wrapper voor rpm
\end{itemize}
%
responsefile:\\
Een bestand waarin de antwoorden staan op vragen die de package stelt bij het installeren. Dit wordt momenteel gebruikt op Solaris en Debian. De paden die je kan gebruiken zijn afhankelijk van het systeem, maar algemeen gezien zou dit een volledig gekwalificeerd pad moeten zijn.\\\\
%
root:\\
Een alleen-lezen parameter die aanduid in welke map de package is ge\"installeerd.\\\\
%
source:\\
Waar je de package kan terugvinden. Dit moet een lokaal pad zijn of een URL die de gebruikte package manager snapt, puppet haalt de bestanden niet op van alternatieve bronnen. Je kan natuurlijk wel via een 'file' object een bestand laten ophalen, in het lokale bestandssysteem plaatsen en daarvan installeren.\\\\
%
status:\\
Een alleen-lezen parameter die de status van het pakket weergeeft.\\\\
%
vendor:\\
Een alleen-lezen parameter die weergeeft wie het pakket heeft gemaakt.\\\\

package

Manage packages. There is a basic dichotomy in package support right now: Some package types (e.g., yum and apt) can retrieve their own package files, while others (e.g., rpm and sun) cannot. For those package formats that cannot retrieve their own files, you can use the source parameter to point to the correct file.

Puppet will automatically guess the packaging format that you are using based on the platform you are on, but you can override it using the provider parameter; each provider defines what it requires in order to function, and you must meet those requirements to use a given provider.

Autorequires: If Puppet is managing the files specified as a package’s adminfile, responsefile, or source, the package resource will autorequire those files.
Features

    holdable: The provider is capable of placing packages on hold such that they are not automatically upgraded as a result of other package dependencies unless explicit action is taken by a user or another package. Held is considered a superset of installed.
    installable: The provider can install packages.
    purgeable: The provider can purge packages. This generally means that all traces of the package are removed, including existing configuration files. This feature is thus destructive and should be used with the utmost care.
    uninstallable: The provider can uninstall packages.
    upgradeable: The provider can upgrade to the latest version of a package. This feature is used by specifying latest as the desired value for the package.
    versionable: The provider is capable of interrogating the package database for installed version(s), and can select which out of a set of available versions of a package to install if asked.

Provider 	holdable 	installable 	purgeable 	uninstallable 	upgradeable 	versionable
aix 	  	X 	  	X 	X 	X
appdmg 	  	X 	  	  	  	 
apple 	  	X 	  	  	  	 
apt 	X 	X 	X 	X 	X 	X
aptitude 	X 	X 	X 	X 	X 	X
aptrpm 	  	X 	X 	X 	X 	X
blastwave 	  	X 	  	X 	X 	 
dpkg 	X 	X 	X 	X 	X 	 
fink 	X 	X 	X 	X 	X 	X
freebsd 	  	X 	  	X 	  	 
gem 	  	X 	  	X 	X 	X
hpux 	  	X 	  	X 	  	 
macports 	  	X 	  	X 	X 	X
nim 	  	X 	  	X 	X 	X
openbsd 	  	X 	  	X 	  	X
pip 	  	X 	  	X 	X 	X
pkg 	  	X 	  	X 	X 	 
pkgdmg 	  	X 	  	  	  	 
pkgutil 	  	X 	  	X 	X 	 
portage 	  	X 	  	X 	X 	X
ports 	  	X 	  	X 	X 	 
portupgrade 	  	X 	  	X 	X 	 
rpm 	  	X 	  	X 	X 	X
rug 	  	X 	  	X 	X 	X
sun 	  	X 	  	X 	X 	 
sunfreeware 	  	X 	  	X 	X 	 
up2date 	  	X 	  	X 	X 	 
urpmi 	  	X 	  	X 	X 	X
yum 	  	X 	X 	X 	X 	X
zypper 	  	X 	  	X 	X 	X
Parameters
adminfile

A file containing package defaults for installing packages. This is currently only used on Solaris. The value will be validated according to system rules, which in the case of Solaris means that it should either be a fully qualified path or it should be in /var/sadm/install/admin.
allowcdrom

Tells apt to allow cdrom sources in the sources.list file. Normally apt will bail if you try this. Valid values are true, false.
category

A read-only parameter set by the package.
configfiles

Whether configfiles should be kept or replaced. Most packages types do not support this parameter. Valid values are keep, replace.
description

A read-only parameter set by the package.
ensure

What state the package should be in. latest only makes sense for those packaging formats that can retrieve new packages on their own and will throw an error on those that cannot. For those packaging systems that allow you to specify package versions, specify them here. Similarly, purged is only useful for packaging systems that support the notion of managing configuration files separately from ‘normal’ system files. Valid values are present (also called installed), absent, purged, held, latest. Values can match /./.
flavor

Newer versions of OpenBSD support ‘flavors’, which are further specifications for which type of package you want.
instance

A read-only parameter set by the package.
name

The package name. This is the name that the packaging system uses internally, which is sometimes (especially on Solaris) a name that is basically useless to humans. If you want to abstract package installation, then you can use aliases to provide a common name to packages:

# In the 'openssl' class
$ssl = $operatingsystem ? {
  solaris => SMCossl,
  default => openssl
}

# It is not an error to set an alias to the same value as the
# object name.
package { $ssl:
  ensure => installed,
  alias => openssl
}

. etc. .

$ssh = $operatingsystem ? {
  solaris => SMCossh,
  default => openssh
}

# Use the alias to specify a dependency, rather than
# having another selector to figure it out again.
package { $ssh:
  ensure => installed,
  alias => openssh,
  require => Package[openssl]
}

platform

A read-only parameter set by the package.
provider

The specific backend for provider to use. You will seldom need to specify this — Puppet will usually discover the appropriate provider for your platform. Available providers are:

    aix: Installation from AIX Software directory Required binaries: /usr/bin/lslpp, /usr/sbin/installp. Default for operatingsystem == aix. Supported features: installable, uninstallable, upgradeable, versionable.
    appdmg: Package management which copies application bundles to a target. Required binaries: /usr/bin/hdiutil, /usr/bin/curl, /usr/bin/ditto. Supported features: installable.
    apple: Package management based on OS X’s builtin packaging system. This is essentially the simplest and least functional package system in existence – it only supports installation; no deletion or upgrades. The provider will automatically add the .pkg extension, so leave that off when specifying the package name. Required binaries: /usr/sbin/installer. Supported features: installable.
    apt: Package management via apt-get. Required binaries: /usr/bin/apt-cache, /usr/bin/debconf-set-selections, /usr/bin/apt-get. Default for operatingsystem == debianubuntu. Supported features: holdable, installable, purgeable, uninstallable, upgradeable, versionable.
    aptitude: Package management via aptitude. Required binaries: /usr/bin/apt-cache, /usr/bin/aptitude. Supported features: holdable, installable, purgeable, uninstallable, upgradeable, versionable.
    aptrpm: Package management via apt-get ported to rpm. Required binaries: apt-cache, rpm, apt-get. Supported features: installable, purgeable, uninstallable, upgradeable, versionable.
    blastwave: Package management using Blastwave.org’s pkg-get command on Solaris. Required binaries: pkg-get. Supported features: installable, uninstallable, upgradeable.
    dpkg: Package management via dpkg. Because this only uses dpkg and not apt, you must specify the source of any packages you want to manage. Required binaries: /usr/bin/dpkg-deb, /usr/bin/dpkg-query, /usr/bin/dpkg. Supported features: holdable, installable, purgeable, uninstallable, upgradeable.
    fink: Package management via fink. Required binaries: /sw/bin/apt-cache, /sw/bin/dpkg-query, /sw/bin/fink, /sw/bin/apt-get. Supported features: holdable, installable, purgeable, uninstallable, upgradeable, versionable.
    freebsd: The specific form of package management on FreeBSD. This is an extremely quirky packaging system, in that it freely mixes between ports and packages. Apparently all of the tools are written in Ruby, so there are plans to rewrite this support to directly use those libraries. Required binaries: /usr/sbin/pkg_add, /usr/sbin/pkg_info, /usr/sbin/pkg_delete. Supported features: installable, uninstallable.
    gem: Ruby Gem support. If a URL is passed via source, then that URL is used as the remote gem repository; if a source is present but is not a valid URL, it will be interpreted as the path to a local gem file. If source is not present at all, the gem will be installed from the default gem repositories. Required binaries: gem. Supported features: installable, uninstallable, upgradeable, versionable.
    hpux: HP-UX’s packaging system. Required binaries: /usr/sbin/swinstall, /usr/sbin/swlist, /usr/sbin/swremove. Default for operatingsystem == hp-ux. Supported features: installable, uninstallable.

    macports: Package management using MacPorts on OS X.

    Supports MacPorts versions and revisions, but not variants. Variant preferences may be specified using the MacPorts variants.conf file http://guide.macports.org/chunked/internals.configuration-files.html#internals.configuration-files.variants-conf

    When specifying a version in the Puppet DSL, only specify the version, not the revision Revisions are only used internally for ensuring the latest version/revision of a port. Required binaries: /opt/local/bin/port. Supported features: installable, uninstallable, upgradeable, versionable.
    nim: Installation from NIM LPP source Required binaries: /usr/sbin/nimclient. Supported features: installable, uninstallable, upgradeable, versionable.
    openbsd: OpenBSD’s form of pkg_add support. Required binaries: pkg_add, pkg_info, pkg_delete. Default for operatingsystem == openbsd. Supported features: installable, uninstallable, versionable.
    pip: Python packages via pip. Supported features: installable, uninstallable, upgradeable, versionable.
    pkg: OpenSolaris image packaging system. See pkg(5) for more information Required binaries: /usr/bin/pkg. Supported features: installable, uninstallable, upgradeable.
    pkgdmg: Package management based on Apple’s Installer.app and DiskUtility.app. This package works by checking the contents of a DMG image for Apple pkg or mpkg files. Any number of pkg or mpkg files may exist in the root directory of the DMG file system. Sub directories are not checked for packages. See the wiki docs <http://projects.puppetlabs.com/projects/puppet/wiki/Package_Management_With_Dmg_Patterns> for more detail. Required binaries: /usr/bin/hdiutil, /usr/bin/curl, /usr/sbin/installer. Default for operatingsystem == darwin. Supported features: installable.
    pkgutil: Package management using Peter Bonivart’s pkgutil command on Solaris. Required binaries: pkgutil. Supported features: installable, uninstallable, upgradeable.
    portage: Provides packaging support for Gentoo’s portage system. Required binaries: /usr/bin/eix-update, /usr/bin/emerge, /usr/bin/eix. Default for operatingsystem == gentoo. Supported features: installable, uninstallable, upgradeable, versionable.
    ports: Support for FreeBSD’s ports. Again, this still mixes packages and ports. Required binaries: /usr/local/sbin/portversion, /usr/local/sbin/pkg_deinstall, /usr/sbin/pkg_info, /usr/local/sbin/portupgrade. Default for operatingsystem == freebsd. Supported features: installable, uninstallable, upgradeable.
    portupgrade: Support for FreeBSD’s ports using the portupgrade ports management software. Use the port’s full origin as the resource name. eg (ports-mgmt/portupgrade) for the portupgrade port. Required binaries: /usr/local/sbin/portversion, /usr/local/sbin/pkg_deinstall, /usr/local/sbin/portinstall, /usr/sbin/pkg_info, /usr/local/sbin/portupgrade. Supported features: installable, uninstallable, upgradeable.
    rpm: RPM packaging support; should work anywhere with a working rpm binary. Required binaries: rpm. Supported features: installable, uninstallable, upgradeable, versionable.
    rug: Support for suse rug package manager. Required binaries: /usr/bin/rug, rpm. Default for operatingsystem == susesles. Supported features: installable, uninstallable, upgradeable, versionable.
    sun: Sun’s packaging system. Requires that you specify the source for the packages you’re managing. Required binaries: /usr/sbin/pkgadd, /usr/sbin/pkgrm, /usr/bin/pkginfo. Default for operatingsystem == solaris. Supported features: installable, uninstallable, upgradeable.
    sunfreeware: Package management using sunfreeware.com’s pkg-get command on Solaris. At this point, support is exactly the same as blastwave support and has not actually been tested. Required binaries: pkg-get. Supported features: installable, uninstallable, upgradeable.
    up2date: Support for Red Hat’s proprietary up2date package update mechanism. Required binaries: /usr/sbin/up2date-nox. Default for operatingsystem == redhatoelovm and lsbdistrelease == 2.134. Supported features: installable, uninstallable, upgradeable.
    urpmi: Support via urpmi. Required binaries: rpm, urpmi, urpmq. Default for operatingsystem == mandrivamandrake. Supported features: installable, uninstallable, upgradeable, versionable.
    yum: Support via yum. Required binaries: yum, python, rpm. Default for operatingsystem == fedoracentosredhat. Supported features: installable, purgeable, uninstallable, upgradeable, versionable.
    zypper: Support for SuSE zypper package manager. Found in SLES10sp2+ and SLES11 Required binaries: /usr/bin/zypper, rpm. Supported features: installable, uninstallable, upgradeable, versionable.

responsefile

A file containing any necessary answers to questions asked by the package. This is currently used on Solaris and Debian. The value will be validated according to system rules, but it should generally be a fully qualified path.
root

A read-only parameter set by the package.
source

Where to find the actual package. This must be a local file (or on a network file system) or a URL that your specific packaging type understands; Puppet will not retrieve files for you.
status

A read-only parameter set by the package.
type

Deprecated form of provider.
vendor

A read-only parameter set by the package.

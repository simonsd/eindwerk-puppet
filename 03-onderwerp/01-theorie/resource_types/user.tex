\section{User}
%
Het beheren van gebruikers. Dit is vooral bedoeld voor het beheren van systeemgebruikers en er zullen dus enkele mogelijkheden ontbreken die handig zijn bij het beheren van gewone gebruikers. Deze resource type maakt voornamelijk gebruik van mogelijkheden per besturingssysteem bij het beheren van gebruikers, in plaats van het rechtstreeks aanpassen van het '/etc/passwd' bestand of iets dergelijks. Ook hier is de puppet 'auto-require magic' aanwezig, waardoor je automatisch de opgegeven groep zal 'requiren' indien je deze ook beheert.
%
\subsection{Parameters}
allowdupe:\\
Duplicate UID (User ID) definities toestaan. Geldige waardes zijn 'true' en 'false'.\\\\
%
comment:\\
Een beschrijving van de gebruiker, meestal de volledige naam van de persoon in kwestie.\\\\
%
ensure:\\
De status waarin het object zich dient te bevinden. Geldige waardes zijn 'present' of 'absent'.\\\\
%
expiry:\\
De vervaldatum voor de gebruiker. De datum wordt opgegeven in het YYYY-MM-DD formaat, aangevult met nullen waar nodig.\\\\
%
gid:\\
De primaire groep voor de gebruiker. Geldige waardes zijn de GID of de groepsnaam.\\\\
%
groups:\\
Een lijst met extra groepen waar de gebruiker lid van is. De primaire groep moet hier niet opgegeven worden. Indien je meerdere groepen wil opgeven kan je gebruik maken van een array.\\\\
%
home:\\
De 'home' map van de gebruiker.\\\\
%
managehome:\\
Of je de 'home' map van de gebruiker wil beheren of niet. Geldige waardes zijn hierbij 'true' of 'false'.\\\\
%
membership:\\
Of de groepen dienen gezien te worden als de volledige lijst of slechts een minimum-lijst. Geldige waardes zijn 'inclusive' en 'minimum'.\\\\
%
name:\\
De gewenste naam voor de gebruiker. Alhoewel beperkingen voor de gebruikersnamen verschillen per systeem is het een goed idee om je te houden aan volgende criteria:
\begin{itemize}
\item enkel kleine alfanumerieke karakters
\item maximum acht karakters
\item beginnend met een kleine letter
\end{itemize}
%
password:\\
Het paswoord voor de gebruiker, opgegeven in eender welk geencrypteerd formaat het lokale besturingssysteem vereist. Let erop dat indien een dollarteken (\$) wordt gebruikt, het volledige paswoord ingekapseld moet zijn met enkele quotes (').\\\\
%
password\_max\_age:\\
Het maximum aantal dagen dat een opgegeven paswoord gebruikt mag worden voor het ongeldig wordt verklaard.\\\\
%
password\_min\_age:\\
De minimumleeftijd voor een paswoord voor het veranderd mag worden.\\\\
%
provider:\\
Welke provider je wil gebruiken bij het beheren van  gebruikers. In de meeste gevallen zal puppet dit zelf ontdekken en instellen, dit is enkel nuttig voor het overschrijven van de standaardkeuze. Beschikbare providers zijn:
\begin{itemize}
\item aix: standaard gebruikersbeheer voor AIX
\item directoryservice: gebruikersbeheer met behulp van DirectoryService op OS X
\item hpuxuseradd: standaard gebruikersbeheer voor HP-UX
\item ldap: gebruikersbeheer via ldap
\item pw: gebruikersbeheer met het 'pw' programma op FreeBSD
\item useradd: gebruikersbeheer met useradd, beschikbaar op een veelvoud aan systemen
\end{itemize}
%
shell:\\
De standaard login shell voor de gebruiker. De shell mmoet zowel aanwezig als uitvoerbaar zijn.\\\\
%
system:\\
Bepaal of de gebruiker een systeemgebruiker is, met een UID onder een bepaalde waarde. Geldige waardes zijn: 'true' en 'false'.\\\\
%
uid:\\
De gewenste gebruikers-ID. Deze moet numeriek worden opgegeven en zal indien afwezig willekeurig gekozen worden. Dit is meestal niet wenselijk indien je een configuratie op meerdere systemen wil laten werken.\\\\

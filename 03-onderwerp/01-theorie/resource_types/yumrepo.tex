\section{Yumrepo}

Hiermee kan je extra repositories toevoegen aan je package manager.

\subsection{Parameters}

baseurl:\\
De URL voor deze repository. Door deze op 'absent' te zetten wordt de repository compleet verwijderd uit het configuratiebestand. Geldige waardes zijn 'present' en 'absent'.\\\\
%
descr:\\
Een beschrijving van de repository in kwestie. Het opgeven van 'absent' als argument heeft hier hetzelfde effect als bij de 'baseurl' parameter.\\\\
%
enabled:\\
Boolean die bepaalt of de repo al dan niet gebruikt kan worden.\\\\
%
enablegroups:\\
Bepaalt of yum het gebruik van package groepen in deze repo toestaat.\\\\
%
exclude:\\
Een lijst met packages die nooit in aanmerking zullen komen voor updates of installeren uit deze repo. Dit is vooral handig indien meer dan \'e\'en repo een bepaalde package aanbied.\\\\
%
gpgcheck:\\
Boolean voor het nakijken van de GPG handtekeningen van packages die ge\"installeerd zijn vanuit deze repo.\\\\
%
gpgkey:\\
De URL waarop men de GPG sleutel voor deze repo kan vinden.\\\\
%
include:\\
Hiermee kan je een extra configuratiebestand injecteren in het huidige bestand.\\\\
%
includepkgs:\\
Het omgekeerde van de 'exclude' parameter. Hierbij zullen enkel de packages in de lijst ge\"includeerd worden en de rest genegeerd.\\\\
%
metadata\_expire:\\
Hoe lang metadata geldig blijft, te specifi\"eren in seconden.\\\\
%
mirrorlist:\\
De URL waarop men een lijst met mirrors kan vinden.\\\\
%
name:\\
De naam voor deze repo. Dit komt overeen met de repo-id in yum.conf.\\\\
%
priority:\\
De prioriteit van deze repo, gaande van 1 tot 99. Vereist is dat de yum-priorities plugin ge\"installeerd is.\\\\
%
protect:\\
Boolean om in te stellen of deze repo beveiligd is of niet. Dit vereist de protectbase plugin voor yum.\\\\
%
proxy:\\
URL naar een proxy server. Deze proxy server zal dan gebruikt worden om te verbinden met de repositories.\\\\
%
proxy\_password:\\
Het paswoord voor desbetreffende proxy.\\\\
%
proxy\_username:\\
De gebruikersnaam voor de proxy die je wil gebruiken.\\\\
%
timeout:\\
Het aantal seconden dat je wil wachten voordat je de communicatie verbreekt in het geval er geen verbinding is.\\\\

service

Manage running services. Service support unfortunately varies widely by platform — some platforms have very little if any concept of a running service, and some have a very codified and powerful concept. Puppet’s service support will generally be able to do the right thing regardless (e.g., if there is no ‘status’ command, then Puppet will look in the process table for a command matching the service name), but the more information you can provide, the better behaviour you will get. In particular, any virtual services that don’t have a predictable entry in the process table (for example, network on Red Hat/CentOS systems) will manifest odd behavior on restarts if you don’t specify hasstatus or a status command.

Note that if a service receives an event from another resource, the service will get restarted. The actual command to restart the service depends on the platform. You can provide an explicit command for restarting with the restart attribute, or use the init script’s restart command with the hasrestart attribute; if you do neither, the service’s stop and start commands will be used.
Features

    controllable: The provider uses a control variable.
    enableable: The provider can enable and disable the service
    refreshable: The provider can restart the service.

Provider 	controllable 	enableable 	refreshable
base 	  	  	X
bsd 	  	X 	X
daemontools 	  	X 	X
debian 	  	X 	X
freebsd 	  	X 	X
gentoo 	  	X 	X
init 	  	  	X
launchd 	  	X 	X
redhat 	  	X 	X
runit 	  	X 	X
smf 	  	X 	X
src 	  	  	X
upstart 	  	  	X
Parameters
binary

The path to the daemon. This is only used for systems that do not support init scripts. This binary will be used to start the service if no start parameter is provided.
control

The control variable used to manage services (originally for HP-UX). Defaults to the upcased service name plus START replacing dots with underscores, for those providers that support the controllable feature.
enable

Whether a service should be enabled to start at boot. This property behaves quite differently depending on the platform; wherever possible, it relies on local tools to enable or disable a given service. Valid values are true, false. Requires features enableable.
ensure

Whether a service should be running. Valid values are stopped (also called false), running (also called true).
hasrestart

Specify that an init script has a restart option. Otherwise, the init script’s stop and start methods are used. Valid values are true, false.
hasstatus

Declare the the service’s init script has a functional status command. Based on testing, it was found that a large number of init scripts on different platforms do not support any kind of status command; thus, you must specify manually whether the service you are running has such a command. Alternately, you can provide a specific command using the status attribute.

If you specify neither of these, then Puppet will look for the service name in the process table. Be aware that ‘virtual’ init scripts such as networking will respond poorly to refresh events (via notify and subscribe relationships) if you don’t override this default behavior. Valid values are true, false.
manifest

Specify a command to config a service, or a path to a manifest to do so.
name

The name of the service to run. This name is used to find the service in whatever service subsystem it is in.
path

The search path for finding init scripts. Multiple values should be separated by colons or provided as an array.
pattern

The pattern to search for in the process table. This is used for stopping services on platforms that do not support init scripts, and is also used for determining service status on those service whose init scripts do not include a status command.

If this is left unspecified and is needed to check the status of a service, then the service name will be used instead.

The pattern can be a simple string or any legal Ruby pattern.
provider

The specific backend for provider to use. You will seldom need to specify this — Puppet will usually discover the appropriate provider for your platform. Available providers are:

    base: The simplest form of service support.

    You have to specify enough about your service for this to work; the minimum you can specify is a binary for starting the process, and this same binary will be searched for in the process table to stop the service. It is preferable to specify start, stop, and status commands, akin to how you would do so using init.

    Required binaries: kill. Supported features: refreshable.

    bsd: FreeBSD’s (and probably NetBSD?) form of init-style service management.

    Uses rc.conf.d for service enabling and disabling.

    Supported features: `enableable`, `refreshable`.

    daemontools: Daemontools service management.

    This provider manages daemons running supervised by D.J.Bernstein daemontools. It tries to detect the service directory, with by order of preference:
        /service
        /etc/service
        /var/lib/svscan

    The daemon directory should be placed in a directory that can be by default in:
        /var/lib/service
        /etc

    or this can be overriden in the service resource parameters::

    service { "myservice": provider => "daemontools", path => "/path/to/daemons",
    }

    This provider supports out of the box:
        start/stop (mapped to enable/disable)
        enable/disable
        restart
        status

    If a service has ensure => "running", it will link /path/to/daemon to /path/to/service, which will automatically enable the service.

    If a service has ensure => "stopped", it will only down the service, not remove the /path/to/service link.

    Required binaries: /usr/bin/svc, /usr/bin/svstat. Supported features: enableable, refreshable.

    debian: Debian’s form of init-style management.

    The only difference is that this supports service enabling and disabling via update-rc.d and determines enabled status via invoke-rc.d.

    Required binaries: /usr/sbin/update-rc.d, /usr/sbin/invoke-rc.d. Default for operatingsystem == debianubuntu. Supported features: enableable, refreshable.
    freebsd: Provider for FreeBSD. Makes use of rcvar argument of init scripts and parses/edits rc files. Default for operatingsystem == freebsd. Supported features: enableable, refreshable.

    gentoo: Gentoo’s form of init-style service management.

    Uses rc-update for service enabling and disabling.

    Required binaries: /sbin/rc-update. Default for operatingsystem == gentoo. Supported features: enableable, refreshable.

    init: Standard init service management.

    This provider assumes that the init script has no status command, because so few scripts do, so you need to either provide a status command or specify via hasstatus that one already exists in the init script.

    Supported features: `refreshable`.

    launchd: launchd service management framework.

    This provider manages jobs with launchd, which is the default service framework for Mac OS X and is potentially available for use on other platforms.

    See:
        http://developer.apple.com/macosx/launchd.html
        http://launchd.macosforge.org/

    This provider reads plists out of the following directories:
        /System/Library/LaunchDaemons
        /System/Library/LaunchAgents
        /Library/LaunchDaemons
        /Library/LaunchAgents

    …and builds up a list of services based upon each plist’s “Label” entry.

    This provider supports:
        ensure => running/stopped,
        enable => true/false
        status
        restart

    Here is how the Puppet states correspond to launchd states:
        stopped — job unloaded
        started — job loaded
        enabled — ‘Disable’ removed from job plist file
        disabled — ‘Disable’ added to job plist file

    Note that this allows you to do something launchctl can’t do, which is to be in a state of “stopped/enabled or “running/disabled”.

    Required binaries: /bin/launchctl, /usr/bin/plutil, /usr/bin/sw_vers. Default for operatingsystem == darwin. Supported features: enableable, refreshable.

    redhat: Red Hat’s (and probably many others) form of init-style service management:

    Uses chkconfig for service enabling and disabling.

    Required binaries: /sbin/service, /sbin/chkconfig. Default for operatingsystem == redhatfedorasusecentosslesoelovm. Supported features: enableable, refreshable.

    runit: Runit service management.

    This provider manages daemons running supervised by Runit. It tries to detect the service directory, with by order of preference:
        /service
        /var/service
        /etc/service

    The daemon directory should be placed in a directory that can be by default in:
        /etc/sv

    or this can be overriden in the service resource parameters::

    service { "myservice": provider => "runit", path => "/path/to/daemons",
    }

    This provider supports out of the box:
        start/stop
        enable/disable
        restart
        status

    Required binaries: /usr/bin/sv. Supported features: enableable, refreshable.

    smf: Support for Sun’s new Service Management Framework.

    Starting a service is effectively equivalent to enabling it, so there is only support for starting and stopping services, which also enables and disables them, respectively.

    By specifying manifest => “/path/to/service.xml”, the SMF manifest will be imported if it does not exist.

    Required binaries: /usr/sbin/svcadm, /usr/bin/svcs, /usr/sbin/svccfg. Default for operatingsystem == solaris. Supported features: enableable, refreshable.

    src: Support for AIX’s System Resource controller.

    Services are started/stopped based on the stopsrc and startsrc commands, and some services can be refreshed with refresh command.

        Enabling and disableing services is not supported, as it requires modifications to /etc/inittab.

        Starting and stopping groups of subsystems is not yet supported Required binaries: /usr/bin/stopsrc, /usr/bin/startsrc, /usr/bin/lssrc, /usr/bin/refresh. Default for operatingsystem == aix. Supported features: refreshable.

    upstart: Ubuntu service manager upstart.

    This provider manages upstart jobs which have replaced initd.

    See: * http://upstart.ubuntu.com/ Required binaries: /sbin/restart, /sbin/start, /sbin/status, /sbin/initctl, /sbin/stop. Supported features: refreshable.

restart

Specify a restart command manually. If left unspecified, the service will be stopped and then started.
start

Specify a start command manually. Most service subsystems support a start command, so this will not need to be specified.
status

Specify a status command manually. This command must return 0 if the service is running and a nonzero value otherwise. Ideally, these return codes should conform to the LSB’s specification for init script status actions, but puppet only considers the difference between 0 and nonzero to be relevant.

If left unspecified, the status method will be determined automatically, usually by looking for the service in the process table.
stop

Specify a stop command manually.

\section{Service}

Deze resource type is specifiek ontworpen voor het beheren van services. Er zijn een hele resem providers beschikbaar, waardoor dit bruikbaar is op de meest voorkomende systemen, alsook een heleboel verouderde/minder bekend systemen. Denk eraan dat als een 'service' object een event ontvangt van een ander object, dit zal leiden tot het herstarten van de service. Het gebruikte commando zal daarbij verschillen per platform.

\subsection{Parameters}
binary:\\
Hier kan je manueel het pad naar de service binary opgeven, indien je op een systeem werkt waar geen ondersteuning is voor init scripts. Deze binary zal ook gebruikt worden om de service te starten indien geen 'start' parameter wordt opgegeven.\\\\
%
enable:\\
Of een service bij systeemboot dient opgestart te worden of niet. Geldige waardes zijn 'true' en 'false'.\\\\
%
ensure:\\
Bepaal de status van een service. Geldige waardes zijn 'stopped' ('false') of 'running' ('true').\\\\
%
hasrestart:\\
Specifi\"eer dat een init scipt een herstart mogelijkheid bezit, anders zullen de start en stop commando's gebruikt worden om te herstarten. Geldige waardes zijn: 'true' en 'false'.\\\\
%
hasstatus:\\
Duid aan of de init scripts de mogelijkheid hebben om de status van de service weer te geven. Indien dit niet het geval is, kan je met de 'status' parameter zelf een commando opgeven.\\\\
%
manifest:\\
Geef een pad op naar een commando of manifests dat toestaat de configuratie van een service aan te passen.\\\\
%
name:\\
De naam van de service die je wil beheren. Deze naam wordt gebruikt om de service op te zoeken in het opgegeven subsysteem (provider).\\\\
%
path:\\
Het pad waarop normaal gezien de init scripts te vinden zijn. Meerdere waardes kunnen opgegeven worden door middel van een array of door de waardes te scheiden met een ':' teken.\\\\
%
pattern:\\
Het patroon waarnaar je wil zoeken in de processen tabel indien geen status en/of stop commando's opgegeven zijn. Indien dit leeg wordt gelaten zal de servicenaam gebruikt worden om te zoeken.\\\\
%
provider:\\
Welke provider je wil gebruiken om services op het systeem te beheren. Normaal gezien herkent puppet dit automatisch en hoef je dit dus niet uitdrukkelijk op te geven.
\begin{itemize}
\item base: De simpelste vorm van servicebeheer. Hierbij gebruik je geen specifiek subsysteem en moet je voor alle belangrijke commando's (start,stop,herstart,status) zelf een argument opgeven.
\item bsd: FreeBSD's ingebouwde vorm van servicebeheer via init. Hierbij wordt gebruik gemaakt van rc.conf.d voor de 'enable' en 'disable' parameters.
\item daemontools: servicebeheer aan de hand van D.J. Bernstein's Daemontools.
\item debian: Debian's manier van servicebeheer gebaseerd op init. Het enige verschil hierbij is het gebruik van update-rc.d voor de 'enable' en 'disable' parameters en invoke-rc.d voor de 'status' parameter.
\item freebsd: FreeBSD standaard servicebeheer via de rc scripts.
\item gentoo: Gentoo's init servicebeheer. Dit gebruikt rc-update voor de 'enable' en 'disable' parameters.
item init: Standaard init servicebeheer. Dit argument gaat er van uit dat er geen statuscommando beschikbaar is.
\item launchd: Mac OS X's launchd servicebeheer framework.
\item redhat: Red Hat's chkconfig framework voor init-stijl servicebeheer.
\item runit: Runit servicebeheer.
\item smf: Sun's Service Management Framework. Deze provider biedt geen ondersteuning voor aparte 'enable' en 'disable' commando's, als je een service start wordt deze ook meteen 'enabled'. Indien je de service stopt is deze ook 'disabled'.
\item src: AIX's System Resource controller servicebeheer. Deze provider heeft geen ondersteuning voor de 'enable' en 'disable' parameters.
\item upstart: Ubuntu's servicebeheer via upstart.
\end{itemize}
%
restart:\\
Geef een commando op dat zal gebruikt worden om de service te herstarten. Indien dit niet wordt opgegeven, zal de service gestopt en dan gestart worden.\\\\
%
start:\\
Geef een commando op dat zal gebruikt worden om de service te starten. De meeste providers hebben een ingebouwd start-commando waardoor je dit niet hoeft te specifi\"eren.\\\\
%
status:\\
Geef een commando op dat zal gebruikt worden om de status van de service te bepalen. Een return waarde van 0 duidt aan dat de service draait, een return waarde hoger dan 0 duidt op het niet draaien van de service. Normaal gesproken dienen deze script te voldoen aan de LSB specificatie voor init script, maar voor puppet is enkel het verschil tussen 0 en 1 van belang. Indien geen commando wordt opgegeven zal puppet de status van de service zelf proberen te achterhalen door in de tabel met processen te zoeken.\\\\
%
stop:\\
Geef zelf een commando op dat gebruikt moet worden om de service te stoppen.\\\\

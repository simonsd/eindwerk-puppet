\section{Stage}
Puppet manifests worden steeds uitgevoerd in stages. Standaard wordt er enkel een 'main'  stage aangemaakt, die alle aanwezige classes omvat. Simplistisch bekeken zijn stages voor classes wat classes zijn voor objecten; een soort van groepen. Strict genomen heb je dit niet nodig, omdat je met de 'require' en 'before' parameters ook op classes invloed hebt en zo zelf kan schikken wat wanneer moet uitgevoerd worden. Aangezien classes in combinatie met deze parameters ook al bruikbaar zijn om je objecten netjes om beurt te laten uitvoeren, hangt het gebruik van stages enkel af van hoe flexibel je je setup wil maken. Stages kan je namelijk enkel toepassen op classes, niet op gewone ingebouwde resource types.
%
\begin{code}
\begin{lstlisting}
class { foo:
	stage => pre,
}
\end{lstlisting}
\end{code}
%
De volgorde waarin stages worden uitgevoerd dien je ook compleet zelf te beheren, aangezien je je eigen stages aanmaakt en puppet dus niet weet hoe jij ze graag zou ordenen. Het ordenen is simpel, dat doe je op dezelfde manier als bij classes met 'require' en 'before' parameters.
%
\begin{code}
\begin{lstlisting}
stage { pre:
	before => Stage[main],
}
\end{lstlisting}
\end{code}
%
\subsection{Parameters}
Bij stages heb je slechts een handvol parameters, namelijk:\\\\
name:\\
De naam die je de stage wil geven.\\\\
before:\\
Het object/class/stage waarvoor je deze stage wil uitvoeren.\\\\
require:\\
Het object/class/stage waarna je deze stage wil uitvoeren.\\\\

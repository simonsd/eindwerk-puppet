\section{Schedule}

Hiermee kan je een 'schedule' aanleggen, die bepaalt welke objecten op welke tijdstippen/hoeveel keer dat object gerund mag worden. Bij de start van de puppetrun zal puppet een lijst opstellen met objecten die w\'el gerund mogen worden. Indien bepaalde objecten er niet op staan, zullen deze overgeslagen worden. Langs de andere kan men momenteel nog niet garanderen dat een object, indien gebonden aan de schedule, w\'el gerund word. Dit wil simpelweg zeggen dat men er wel voor kan zorgen dat een object gestart word, maar indien dit object bijvoorbeeld een parameter 'ensure' aanneemt en deze heeft als waarde 'false', zal dit object alsnog overgeslaan worden. Hetzelfde geld voor het runnen van incomplete of foutieve objecten, als de configuratie niet correct is of de uitvoering op de host mislukt is er geen manier om te garanderen dat dit werkt. Als laatste potenti\"ele valkuil is er het feit dat de puppetmaster of client uitgeschakeld zijn op het vooraf ingestelde tijdstip, er is dan wederom geen mogelijkheid om te garanderen dat de manifests uitgevoerd worden op een later tijdstip.

Ten einde dit allemaal een beetje te verhelpen is het aangeraden de runs te spreiden en het schema niet te restrictief in te stellen. Zo is er een veel grotere kans dat de gewenste configuraties zullen toegepast worden.
Thus, it behooves you to use wider scheduling (e.g., over a couple of hours) combined with periods and repetitions. For instance, if you wanted to restrict certain resources to only running once, between the hours of two and 4 AM, then you would use this schedule:

schedule { maint:
  range => "2 - 4",
  period => daily,
  repeat => 1
}

With this schedule, the first time that Puppet runs between 2 and 4 AM, all resources with this schedule will get applied, but they won’t get applied again between 2 and 4 because they will have already run once that day, and they won’t get applied outside that schedule because they will be outside the scheduled range.

Puppet automatically creates a schedule for each valid period with the same name as that period (e.g., hourly and daily). Additionally, a schedule named puppet is created and used as the default, with the following attributes:

schedule { puppet:
  period => hourly,
  repeat => 2
}

This will cause resources to be applied every 30 minutes by default.

\subsection{Parameters}

name:\\
De naam die je wil meegeven aan het schema. Deze naam word gebruikt om het schema op te zoeken wanneer men het toewijst aan een object.

schedule { daily:
  period => daily,
  range => "2 - 4",
}
  
exec { "/usr/bin/apt-get update":
  schedule => daily
}

period:\\
The period of repetition for a resource. Choose from among a fixed list of hourly, daily, weekly, and monthly. The default is for a resource to get applied every time that Puppet runs, whatever that period is.

Note that the period defines how often a given resource will get applied but not when; if you would like to restrict the hours that a given resource can be applied (e.g., only at night during a maintenance window) then use the range attribute.

If the provided periods are not sufficient, you can provide a value to the repeat attribute, which will cause Puppet to schedule the affected resources evenly in the period the specified number of times. Take this schedule:

schedule { veryoften:
  period => hourly,
  repeat => 6
}

This can cause Puppet to apply that resource up to every 10 minutes.

At the moment, Puppet cannot guarantee that level of repetition; that is, it can run up to every 10 minutes, but internal factors might prevent it from actually running that often (e.g., long-running Puppet runs will squash conflictingly scheduled runs).

See the periodmatch attribute for tuning whether to match times by their distance apart or by their specific value. Valid values are hourly, daily, weekly, monthly, never.

periodmatch:\\
Whether periods should be matched by number (e.g., the two times are in the same hour) or by distance (e.g., the two times are 60 minutes apart). Valid values are number, distance.

range:\\
The earliest and latest that a resource can be applied. This is always a range within a 24 hour period, and hours must be specified in numbers between 0 and 23, inclusive. Minutes and seconds can be provided, using the normal colon as a separator. For instance:

schedule { maintenance:
  range => "1:30 - 4:30"
}

This is mostly useful for restricting certain resources to being applied in maintenance windows or during off-peak hours.

repeat:\\
Hoeveel maal het gekozen commando/object dient uitgevoerd te worden binnen een bepaalde tijdsduur. Standaardwaarde is 1 en geldige waardes zijn enkel integers.\\\\

\section{Mount}

Het beheren van gemounte filesystems, inclusief de mount tabel aanpassen. Let erop dat als een mount een event ontvangt van een object, zal het object ge-remount worden.

\subsection{Parameters}

atboot:\\
Boolean die ervoor zorgt dat een filesystem gemount word bij het opstarten van het besturingssysteem. Niet alle platformen ondersteunen dit.\\\\
%
blockdevice:\\
Welk apparaat dient gecontroleerd te worden bij een fsck (filesystem check). Deze parameter is enkel geldig op Solaris en zal in de meeste gevallen automatisch de juiste waarde aannemen.
%
device:\\
Het apparaat dat het filesystem bevat dat gemount dient te worden. Geldige waardes zijn eender wat het platform aanvaard, zij het apparaat paden (/dev/sda1), UUID's (78c656dd-6c2a-48d0-a6c2-e7f3061e89e6), netwerk mount punten (http://192.168.1.1:21/share), etc.\\\\
%
ensure:\\
Hiermee kan je beheren wat je wil doen met deze mount, geldige waardes zijn mounted, defined (synoniem voor present), unmounted en absent. Bij de waarde mounted zal een entry worden toegevoegd in de fstab en het filesystem ook effectief gemount worden. Bij een waarde present of defined zal ervoor gezorgd worden dat het filesystem aanwezig is in de filesystem tabel maar word er verder niets veranderd aan de mount status. Het argument unmounted doet hetzelfde maar zorgt ervoor dat er w\'el naar de status word gekeken en eventueel het filesystem zal unmounten. Als laatste hebben we het argument absent, dat ervoor dat er geen entry in de filesystem tabel word aangemaakt/aanwezig is, en dat de status van het filesystem unmounted is.\\\\
%
fstype:\\
Het type filesystem, geldige waardes zijn afhankelijk van het besturingssysteem (verplichte optie).\\\\
%
name:\\
Het pad waar het filesystem gemount dient te worden.\\\\
%
options:\\
Eventuele extra opties voor de mounts, zoals ze zouden worden beschreven in de fstab.\\\\
%
pass:\\
In welke pass deze mount gecheckt word in geval van een fsck.\\\\
%
provider:\\
Welke backend provider je wil gebruiken om filesystems te mounten. Je zal dit normaal gezien niet moeten specifi\"eren, puppet ontdekt het meestal vanzelf. Geldige backends zijn momenteel enkel: 'parsed', die zowel de programma's 'mount' als 'umount' vereist.\\\\
%
remounts:\\
Boolean die bepaalt of een filesystem al dan niet ge-remount kan worden mits het commando 'mount -o remount [filesystem]'. Indien 'false' zal het filesystem ge-unmount en dan ge-mount worden, wat meer kans heeft op falen.\\\\
%
target:\\
Het bestand dat je wil gebruiken als mount tabel (mtab). Word enkel gebruikt door providers die naar de harde schijf wegschrijven.\\\\

\section{File}

\subsection{Doel}
Het beheren van locale bestanden, inclusief permissies, het cre\"{e}ren van zowel bestanden als mappen, en het downloaden van bestanden van remote servers. Het is de bedoeling dat naarmate Puppet verder ontwikkeld, deze resource minder nodig zal zijn, en alles beheerd zal kunnenn worden met nieuwe resource types. Wanneer Puppet de eigenaar en groep van het bestand ook beheerd, zal deze automatisch 'required' worden.\\\\
%
\subsection{Parameters}
backup:\\
Of er een backup van bestanden dient gemaakt te worden voordat ze vervangen worden. De voorkeur gaat hierbij uit naar het gebruiken van een filebucket.	De filebucket slaat bestanden op, herkent ze via hun md5-som, en zorgt voor een simpele overdraginsgmethode zonder de mappenstructuur in de war te brengen. Daarnaast kan je een waarde meegeven die begint met een ".". in dit geval zal de backup dit als extensie meekrijgen. Puppet cre\"{e}ert automatisch een lokale filebucket waarnaar backups zullen geschreven worden. Als je een andere filebucket wil gebruiken dien je dit in je manifests te specifi\"{e}ren.
\begin{code}
\begin{lstlisting}
filebucket { main:
	server => puppet,
}
\end{lstlisting}
\end{code}
%
De puppetmaster cre\"{e}ert standaard zelf een filebucket, die je dan kan gebruiken om backups te maken naar je server
The puppet master daemon creates a filebucket by default, so you can usually back up to your main server with this configuration. Once you've described the bucket in your configuration, you can use it in any file
%
\begin{code}
\begin{lstlisting}
file { "/my/file":
	source => "/path/in/nfs/or/something",
	backup => main
}
\end{lstlisting}
\end{code}
%
This will back the file up to the central server.
At this point, the benefits of using a filebucket are that you do not have backup files lying around on each of your machines, a given version of a file is only backed up once, and you can restore any given file manually, no matter how old. Eventually, transactional support will be able to automatically restore filebucketed files.
%
checksum:\\
The checksum type to use when checksumming a file.
The default checksum parameter, if checksums are enabled, is md5. Valid values are md5, md5lite, mtime, ctime, none.
%
content:\\
Specify the contents of a file as a string. Newlines, tabs, and spaces can be specified using the escaped syntax (e.g., \\n for a newline). The primary purpose of this parameter is to provide a kind of limited templating:
%
\begin{code}
\begin{lstlisting}
define resolve(nameserver1, nameserver2, domain, search) {
	$str = "search $search
	domain $domain
	nameserver $nameserver1
	nameserver $nameserver2
	"

	file { "/etc/resolv.conf":
		content => $str
	}
}
\end{lstlisting}
\end{code}

	This attribute is especially useful when used with templating.
%
ctime:\\
A read-only state to check the file ctime.\\\\
%
ensure:\\
Whether to create files that don't currently exist. Possible values are absent, present, file, and directory. Specifying present will match any form of file existence, and if the file is missing will create an empty file. Specifying absent will delete the file (and directory if recurse => true).
Anything other than those values will create a symlink. In the interest of readability and clarity, you should use ensure => link and explicitly specify a target; however, if a target attribute isn't provided, the value of the ensure attribute will be used as the symlink target:

(Useful on Solaris)
Less maintainable: 
file { "/etc/inetd.conf":
\begin{code}
\begin{lstlisting}
  ensure => "/etc/inet/inetd.conf",
}
\end{lstlisting}
\end{code}

More maintainable:
\begin{code}
\begin{lstlisting}
file \{ "/etc/inetd.conf":
ensure => link,
target => "/etc/inet/inetd.conf",
}
\end{lstlisting}
\end{code}
%
These two declarations are equivalent. Valid values are absent (also called false), file, present, directory, link. Values can match /./.
force:\\
Force the file operation. Currently only used when replacing directories with links. Valid values are true, false.
%
group:\\
Which group should own the file. Argument can be either group name or group ID.
%
ignore:\\
A parameter which omits action on files matching specified patterns during recursion. Uses Ruby\'s builtin globbing engine, so shell metacharacters are fully supported, e.g. [a-z]*. Matches that would descend into the directory structure are ignored, e.g., */*.
%
links:\\
How to handle links during file actions. During file copying, follow will copy the target file instead of the link, manage will copy the link itself, and ignore will just pass it by. When not copying, manage and ignore behave equivalently (because you cannot really ignore links entirely during local recursion), and follow will manage the file to which the link points. Valid values are follow, manage.
%
mode:\\
Mode the file should be. Currently relatively limited: you must specify the exact mode the file should be.
Note that when you set the mode of a directory, Puppet always sets the search/traverse (1) bit anywhere the read (4) bit is set. This is almost always what you want: read allows you to list the entries in a directory, and search/traverse allows you to access (read/write/execute) those entries.) Because of this feature, you can recursively make a directory and all of the files in it world-readable by setting e.g.:
%
\begin{code}
\begin{lstlisting}
file { '/some/dir':
	mode => 644,
	recurse => true,
}
\end{lstlisting}
\end{code}
%
In this case all of the files underneath /some/dir will have mode 644, and all of the directories will have mode 755.
%
mtime:\\
A read-only state to check the file mtime.\\\\
%
owner:\\
To whom the file should belong. Argument can be user name or user ID.\\\\
%
path:\\
The path to the file to manage. Must be fully qualified.\\\\
%
provider:\\
The specific backend for provider to use. You will seldom need to specify this - Puppet will usually discover the appropriate provider for your platform. Available providers are:
%
microsoft\_windows: Uses Microsoft Windows functionality to manage file's users and rights.
posix: Uses POSIX functionality to manage file's users and rights.

purge:\\
Whether unmanaged files should be purged. If you have a filebucket configured the purged files will be uploaded, but if you do not, this will destroy data. Only use this option for generated files unless you really know what you are doing. This option only makes sense when recursively managing directories.
%
Note that when using purge with source, Puppet will purge any files that are not on the remote system. Valid values are true, false.
%
recurse:\\
Whether and how deeply to do recursive management. Options are:
%
inf,true - Regular style recursion on both remote and local directory structure.
remote - Descends recursively into the remote directory but not the local directory. Allows copying of a few files into a directory containing many unmanaged files without scanning all the local files.
false - Default of no recursion.
[0-9]+ - Same as true, but limit recursion. Warning: this syntax has been deprecated in favor of the recurselimit attribute. Valid values are true, false, inf, remote. Values can match \/\^[0-9]+\$\/\.
%
recurselimit:\\
Hoe diep men mag gaan wanneer de 'recurse' parameter word meegegeven. Waardes mogen /\^[0-9]+\$\/\ bevatten.\\\\
%How deeply to do recursive management. Values can match \/\^[0-9]+\$\/\.\\\\
%
replace:\\
Whether or not to replace a file that is sourced but exists. This is useful for using file sources purely for initialization. Valid values are true (also called yes), false (also called no).
%
selinux\_ignore\_defaults:\\
If this is set then Puppet will not ask SELinux (via matchpathcon) to supply defaults for the SELinux attributes (seluser, selrole, seltype, and selrange). In general, you should leave this set at its default and only set it to true when you need Puppet to not try to fix SELinux labels automatically. Valid values are true, false.
%
selrange:\\
What the SELinux range component of the context of the file should be. Any valid SELinux range component is accepted. For example s0 or SystemHigh. If not specified it defaults to the value returned by matchpathcon for the file, if any exists. Only valid on systems with SELinux support enabled and that have support for MCS (Multi-Category Security).
%
selrole:\\
What the SELinux role component of the context of the file should be. Any valid SELinux role component is accepted. For example role\_r. If not specified it defaults to the value returned by matchpathcon for the file, if any exists. Only valid on systems with SELinux support enabled.
%
seltype:\\
What the SELinux type component of the context of the file should be. Any valid SELinux type component is accepted. For example tmp\_t. If not specified it defaults to the value returned by matchpathcon for the file, if any exists. Only valid on systems with SELinux support enabled.
%
seluser:\\
What the SELinux user component of the context of the file should be. Any valid SELinux user component is accepted. For example user\_u. If not specified it defaults to the value returned by matchpathcon for the file, if any exists. Only valid on systems with SELinux support enabled.
%
source:\\
Copy a file over the current file. Uses checksum to determine when a file should be copied. Valid values are either fully qualified paths to files, or URIs. Currently supported URI types are puppet and file.
%
This is one of the primary mechanisms for getting content into applications that Puppet does not directly support and is very useful for those configuration files that don't change much across sytems. For instance:
%
\begin{code}
\begin{lstlisting}
class sendmail {
	file { "/etc/mail/sendmail.cf":
	source => "puppet://server/modules/module\_name/sendmail.cf"
	}
}
\end{lstlisting}
\end{code}
%
You can also leave out the server name, in which case puppet agent will fill in the name of its configuration server and puppet apply will use the local filesystem. This makes it easy to use the same configuration in both local and centralized forms.
%
Currently, only the puppet scheme is supported for source URL's. Puppet will connect to the file server running on server to retrieve the contents of the file. If the server part is empty, the behavior of the command-line interpreter (puppet apply) and the client demon (puppet agent) differs slightly: apply will look such a file up on the module path on the local host, whereas agent will connect to the puppet server that it received the manifest from.
%
See the fileserver configuration documentation for information on how to configure and use file services within Puppet.
If you specify multiple file sources for a file, then the first source that exists will be used. This allows you to specify what amount to search paths for files:
%
\begin{code}
\begin{lstlisting}
file \{ "/path/to/my/file":
	source => [
		"/modules/nfs/files/file.$host",
		"/modules/nfs/files/file.$operatingsystem",
		"/modules/nfs/files/file"
	]
}
\end{lstlisting}
\end{code}
%
This will use the first found file as the source.
You cannot currently copy links using this mechanism; set links to follow if any remote sources are links.
%
sourceselect:\\
Whether to copy all valid sources, or just the first one. This parameter is only used in recursive copies; by default, the first valid source is the only one used as a recursive source, but if this parameter is set to all, then all valid sources will have all of their contents copied to the local host, and for sources that have the same file, the source earlier in the list will be used. Valid values are first, all.
%
target:\\
Het bestand of de map waarnaar gelinkt word. Momenteel worden enkel symlinks ondersteund.
You can make relative links:
%
(Useful on Solaris)
\begin{code}
\begin{lstlisting}
file { "/etc/inetd.conf":
	ensure => link,
	target => "inet/inetd.conf",
}
\end{lstlisting}
\end{code}
%
You can also make recursive symlinks, which will create a directory structure that maps to the target directory, with directories corresponding to each directory and links corresponding to each file. Valid values are notlink. Values can match /./.
%
type:\\
Een alleen-lezen status om het bestandstype te testen.\\\\

\section{File}

\subsection{Doel}
Het beheren van locale bestanden, inclusief permissies, het cre\"{e}ren van zowel bestanden als mappen, en het downloaden van bestanden van remote servers. Het is de bedoeling dat naarmate Puppet verder ontwikkeld, deze resource minder nodig zal zijn, en alles beheerd zal kunnen worden met nieuwe resource types. Wanneer Puppet de eigenaar en groep van het bestand ook beheerd, zal deze automatisch 'required' worden.\\\\
%
\subsection{Parameters}
backup:\\
Het argument dat je hieraan meegeeft is een pad waarnaar je backups wil maken van dit bestand. Indien je een naam opgeeft die begint met '.', zal alles na het punt-teken meegegeven worden aan het backup-bestand als bestands-extensie. Bestanden worden herkend via MD5 checksums en zullen ook enkel ge-backupt worden als deze checksum verandert.\\\\
%
checksum:\\
Welk soort checksum je wil gebruiken voor bestanden. Standaard is dit MD5 (het argument dient echter wel in kleine letters opgegeven te worden, dus: 'md5'), wat staat voor Message Digest Algorithm 5. Andere geldige waardes zijn: 'md5lite', 'mtime', 'ctime' en 'none'.\\\\
%
content:\\
Geef de inhoud van het bestand mee als string. Spaties, tabs en regeleindes kunnen worden ingegeven aan de hand van de 'escaped' syntax (bv.: '\\n' als regeleinde, '\\t' voor een tab, etc.). Het voornaamste doel van deze parameter is het voorzien van een simpele manier om een bestand inhoud te geven en daarbij een aantal variabelen door te geven die rechtstreeks in de manifest beschikbaar zijn. Als alternatief kan je hier ook een pad opgeven naar een 'template', waarbij je meer en krachtigere mogelijkheden hebt om variabelen in te geven.\\\\
%
ctime:\\
Een parameter waarmee je de datering van een bestand kan achterhalen. Deze parameter is alleen-lezen, je kan er dus geen waarde mee veranderen.\\\\
%
ensure:\\
Afhankelijk van de waarde die je aan deze parameter meegeeft, zal hij ervoor zorgen dat het opgegeven pad een bestand, een map, een link of verwijderd is. De corresponderende waardes hiervoor zijn: 'present' (of 'file'), 'directory', 'link' of 'absent'. Indien het opgegeven pad een bestand is zal bij het argument 'absent', gezorgd worden dat het bestand verwijderd word indien het aanwezig is. In het geval het opgegeven pad een map is, zal de map enkel verwijderd worden als de 'recurse' parameter de waarde 'true' meekrijgt. Als het argument 'link' is, dien je ook de parameter 'target' in te vullen, anders word de waarde van de 'ensure' parameter in de plaats gebruikt.\\\\
%
group:\\
Welke groep de eigenaar van het pad is. Het argument kan zowel een groepnaam als een groep-ID zijn.\\\\
%
ignore:\\
Deze parameter zorgt ervoor dat bij het afdalen in mappen geen actie ondernomen zal worden op bestanden die overeenkomen met het opgegeven patroon. Je kan hierbij meta-karakters opgeven om meerdere bestanden te matchen.\\\\
%
links:\\
Hiermee kies je hoe je met links wil omgaan als er een actie op wordt ondernomen. Bij het kopi\"eeren van bestanden zal het argument 'follow' ervoor zorgen dat het bestand waarnaar wordt gelinkt gekopie\"eerd wordt. Het argument 'manage' zal dan er voor zorgen dat de link zelf gekopi"eerd wordt en 'ignore' zorgt dat de link wordt overgeslagen.\\\\
%
mode:\\
Welke rechten het bestand moet hebben. Je moet hierbij een geldige 'mode' opgeven in de vorm 'drwxrwxrwx', de standaard Linux permissie-structuur.\\\\
%
owner:\\
Wie de eigenaar van het opgegeven pad moet zijn. Het argument kan zowel een gebruikersnaam als een gebruikers-ID zijn.\\\\
%
path:\\
Het pad naar het bestand of map dat je wil beheren. Het opgegeven pad moet volledig gekwalificeerd zijn, wat wil zeggen dat de volledige padnaam opgegeven dient te worden beginnend vanaf de root directory ('/').\\\\
%
provider:\\
Welke 'provider' je wil gebruiken bij het beheren van bestanden. Je hoeft dit meestal niet op te geven, puppet zal dit normaal gezien zelf uitzoeken. Geldige waardes zijn:
\begin{itemize}
\item microsoft\_windows
\item posix
\end{itemize}
%
purge:\\
In het geval de parameter 'recurse' wordt meegegeven zorgt dit ervoor dat onbeheerde bestanden verwijderd worden. Indien je gebruik maakt van de 'backup' zal eerst een backup gemaakt worden van bestanden die verwijderd worden. Het is best deze parameter spaarzaam te gebruiken, dus enkel indien het bestanden zijn waarvan je zeker weet dat je ze wil verwijderen, in andere gevallen gaat de data definitief verloren.\\\\
%
recurse:\\
Of en hoe diep je wil afdalen binnen mappen. Geldige opties zijn:
\begin{itemize}
\item inf,true - complete afdaling op zowel lokale als remote bestandssystemen
\item remote - complete afdaling, enkel op remote bestandssystemen
\item false - geen afdaling
\end{itemize}
%
recurselimit:\\
Geef op hoe diep de 'recurse' parameter mag afdalen.\\\\
%
replace:\\
Of je een bestand wil vervangen waarvoor een 'source' is opgegeven als het al bestaat. Dit is handig indien je de 'source' parameter enkel gebruikt om te zorgen dat een bestand aanwezig is, zonder de inhoud te beheren.\\\\
%
source:\\
Zorg ervoor dat een bepaald bestand op het opgegeven pad staat. Er wordt gebruik gemaakt van checksums om te kijken of het juiste bestand al aanwezig is, indien niet wordt het opnieuw gekopi\"eerd. Het argument voor deze parameter kan zowel een externe URL als een pad naar een lokaal bestand zijn. Dit is ook een van de manieren om bestanden te beheren van applicaties die momenteel nog geen ingebouwde resource type hebben binnen puppet. Het is ook zeer handig voor configuratiebestanden die op veel systemen hetzelfde zijn. Je kan meerdere 'sources' opgeven, in dit geval wordt de eerste beschikbare 'source' gebruikt.\\\\
%
sourceselect:\\
Of je alle bestanden die overeen komen met het opgegeven patroon wil kopi\"eeren, of enkel het eerste. Geldige waardes zijn 'all' en 'first', waarbij 'first' de standaardwaarde is.\\\\
%
target:\\
Het bestand of de map waarnaar gelinkt word. Momenteel worden enkel symlinks ondersteund.\\\\
%
type:\\
Een status-indicator, vooral bruikbaar om het bestandstype te testen.\\\\

\section{Ssh\_authorized\_keys}

Beheert toegestane SSH sleutels. Handig is dat als puppet de gebruikersaccount beheert waarin deze sleutel ge\"installeerd dient te worden, deze gebruiker automatisch ge-required zal worden.

\subsection{Parameters}

ensure:\\
Boolean waarde voor de aan- of afwezigheid van de sleutel. Geldige waardes zijn 'present' en 'absent'.\\\\
%
key:\\
De sleutel zelf; dit is meestal een lange string bestaande uit hexa-decimalen.\\\\
%
name:\\
Een naam voor de sleutel. Namen moeten uniek zijn, aangezien de sleutels voor het hele systeem dienen.\\\\
%
options:\\
Extra opties voor de sleutel. Meerdere waardes kan je specifi\"eren door middel van een array.\\\\
%
target:\\
De absolute bestandsnaam waarin de sleutel opgeslagen moet worden. Deze optie is niet verpicht en dient enkel gebruikt te worden indien je je SSH-sleutels op een niet-standaard plaats wil bewaren (dit wil zeggen: niet in het standaard '~/.ssh/authorized\_keys' bestand).\\\\
%
type:\\
Het type encryptie dat gebruikt wordt voor de sleutel, geldige waardes zijn: ssh-dss (ook wel dsa genoemd) en ssh-rsa (ook wel rsa genoemd).\\\\
%
user:\\
De gebruikersaccount waarin deze sleutel dient te worden ge\"installeerd.\\\\

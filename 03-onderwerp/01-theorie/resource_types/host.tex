\section{Host}
Op *NIX besturingssystemen heb je een zogenaamde hostsfile, meestal terug te vinden op '/etc/hosts', waarin je ip-adressen en namen kan associ\"{e}ren. Dit zorgt ervoor dat je bijvoorbeeld in je web-browser 'localhost' kan intypen en op je eigen machine terechtkomen.

\subsection{Functie}
Het Cre\"{e}ren en beheren van host entries.

\subsection{Parameters}
comment:\\
Een annotatie in verband met de entry\\\\
%
ensure:\\
Boolean die specifi\"{e}ert of deze entry al dan niet aanwezig dient te zijn.\\\\
%
host\_aliases:\\
De namen die u zou willen meegeven aan deze entry, bijvoorbeeld 'localhost' of 'server'. Indien je hier meerdere waardes wil meegeven dien je gebruik te maken van een array.\\\\
%
ip:\\
Het ip-adres van de node waaraan je deze entry wil toewijzen. Zowel ipv4 als ipv6 is reeds toegestaan.\\\\
%
name:\\
De host-naam.\\\\
%
provider:\\
Het programma waarmee je dit zou willen beheren. Momenteel is enkel 'parsed' beschikbaar, maar dit is meer dan voldoende voor het doel we hier wensen te bereiken.\\\\
%
target:\\
Welk bestand beheerd dient te worden. Standaard is dit '/etc/hosts', dus kan dit veilig worden weggelaten voor de meeste systemen.\\\\

\chapter{Puppet}

\section{Inleiding}
Puppet is een configuration management applicatie, geschreven in Ruby. Het doel van puppet is het beheren van objecten aan de hand van 'manifests'. Binnen deze manifests word gebruik gemaakt van een DSL, een Domain-Specific Language, wat wil zeggen dat deze taal specifiek ontworpen is met een bepaalde taak in gedachten. In het geval van puppet is deze taak systeembeheer, wat wil zeggen het beheren van bestanden, applicaties, services, etc. De syntax van de puppet DSL heeft veel weg van Ruby, nogal logisch aangezien het daarop gebaseerd is. Door het schrijven van deze manifests kan je groepen van taken aaneenkoppelen, om zo je dagdagelijkse werk als systeembeheerder te verlichten. Dit is zeker handig met het oog op het onderhouden van systemen of het opzetten van nieuwe systemen: het uitrollen en beheren van systemen kan allemaal gebeuren aan de hand van \'e\'en set manifests, waardoor de hele setup ook zeer reproduceerbaar wordt.

\section{Onderdelen}

\subsection{Puppet master}
De puppetmaster (ook wel puppetmasterd genoemd) is de puppet server daemon. De puppetmaster dient als centrale server, die volledig gecompileerde configuraties, bestanden en templates aanlevert aan clients. In de meeste gevallen zal je dan ook steevast \'e\'en puppetmaster hebben waar een veelvoud aan puppet agents met verbinden.\\\\
Standaard zal de puppetmaster ook dienen als certificate authority (vanaf nu ook wel CA genoemd), wat wil zeggen dat de puppetmaster certificaten gaat uitdelen en beheren voor elk van zijn clients. Je hebt daarnaast ook de mogelijkheid een aparte certificate authority te gebruiken, zoals DogTag, OpenCA, CAcert of X.509 certificaten. Meer info daarover kan je terugvinden op:\\
'http://en.wikipedia.org/wiki/Certificate\_authority'\\\\
Net als de andere puppet applicaties, wordt de puppetmaster geconfigureerd in het '/etc/puppet/puppet.conf' bestand, in de '[puppetmaster]' sectie.

\subsection{Puppet agent}
De puppet agent, meestal 'puppetd' genoemd, is het programma dat in de achtergrond op elke host draait. Deze applicatie zal elke 30 minuten (afhankelijk van de instellingen) wakker worden, verbinding maken met de puppetmaster, informatie versturen naar de puppetmaster in verband met het hostsysteem (facts) en op basis daarvan een 'gecompileerde catalogus' toegestuurd krijgen, waarin de gewenste systeemconfiguratie beschreven staat. De puppet agent is dan verantwoordelijk voor het aanbrengen van veranderingen op het hostsysteem, zodat het systeem conform is met de toegestuurde catalogus. De puppetmaster is in dit geval verantwoordelijk voor het toesturen van de correcte catalogus, met andere woorden: een catalogus die enkel geschikt is voor het betreffende systeem, waarin geen informatie wordt vrijgegeven over andere hosts.\\\\
Configuratie van de puppet agent vind wederom plaats in het '/etc/puppet/puppet.conf' bestand, onder de sectie '[puppetd]'.

\subsection{Puppet apply}
Het simpelste deel van puppet is 'puppet apply', beter bekend als simpelweg puppet. Dit is een commando waarmee je op geheel afgeschermde wijze manifests kan uitvoeren, zonder nood te hebben aan een puppetmaster of zelfs een internetverbinding. Dit is vooral handig indien je een manifest wil uittesten of voor gevallen waar geen netwerkverbinding beschikbaar is. Aangezien het puppet programma geen verbinding maakt met een server om zijn configuratie op te halen, dien je wel als argument het pad naar een manifest mee te geven. Er is verder geen verschil tussen de puppet agent en puppet apply.

\subsection{Puppet resource}
Het puppet resource commando, beter bekend als 'ralsh' (Resource Abstraction Layer SHell), biedt een interactieve manier om met puppet aan de slag te gaan. Met deze 'shell' heb je toegang tot alle ingebouwde puppet resource types, vanaf de command line. Zo kan je in je favoriete shell (op de meeste distributies is dit standaard BASH, de Bourne Again SHell), 'ralsh <resource type> <resource naam>' uitvoeren en als antwoord een overzicht krijgen in de stijl van een puppet manifest over de huidige status van het opgegeven object. Je kan er natuurlijk ook wijzigingen mee doorvoeren, door achter het commando parameters en argumenten mee te geven. Enkele voorbeelden:
\begin{code}
\begin{lstlisting}
> ralsh User 'xyz'

user { 'xyz':
	ensure => 'present',
	home => '/home/xyz',
	shell => '/bin/bash',
	uid => '1000',
	gid => '1000',
	groups => ['cdrom', 'sudo', 'admin', 'sambashare'],
	comment => 'abc xyz',
}
\end{lstlisting}
\end{code}
Dit is de output als we de informatie opvragen van het object 'User[xyz]'. Op dezelfde manier kunnen we veranderingen aanbrengen:
\begin{code}
\begin{lstlisting}
> ralsh User 'xyz' comment='een andere commentaar'

user { 'xyz':
	ensure => 'present',
	home => '/home/xyz',
	shell => '/bin/bash',
	uid => '1000',
	gid => '1000',
	groups => ['cdrom', 'sudo', 'admin', 'sambashare'],
	comment => 'een andere commentaar',
}
\end{lstlisting}
\end{code}
Hier hebben we het 'comment' attribuut van de gebruiker gewijzigd.

\subsection{Facter}
Facter is een onafhankelijke cross-platform Ruby library, ontworpen om informatie te verzamelen over nodes en is voorhanden voor elk besturingssysteem dat Ruby ondersteunt. Facter is een lichtgewicht programma dat informatie verzamelt over een host, dingen zoals de hostnaam, IP-adres, MAC-adres, SSH keys, hardware informatie, etc. Facter is ook zo ontworpen dat het compleet modulair is, zodat je zelf in een handomdraai je eigen "facts" kan schrijven. Je kan het in Puppet gebruiken om objecten of variabelen aan te passen op basis van een aantal criteria. De integratie gebeurt door simpelweg in de manifests een variabele op te roepen en op basis daarvan een keuze te maken, bijvoorbeeld:
\begin{code}
\begin{lstlisting}
package { 'apache':
	ensure => present,
	name => $operatingsystem ? {
		Debian => 'apache2',
		Centos => 'httpd',
	},
}
\end{lstlisting}
\end{code}
Hier zie je een declaratie van een package genaamd 'apache', waarbij afhankelijk van het besturingssysteem (operatingsystem), gekozen wordt voor een verschillende packagenaam. Je zal deze in de komende voorbeelden regelmatig zien terugkomen, waarbij meestal op basis van het besturingssysteem een keuze gemaakt wordt. Er zijn echter veel meer mogelijkheden, zoals bijvoorbeeld op basis van de hostnaam (naam van de machine) een andere configuratie toepassen, op basis van het IP-adres een template invullen, etc. Indien je met de ingebouwde 'facts' niet toekomt, kan je steeds zelf een 'fact' schrijven en deze gebruiken in je manifests. Een volledige uitleg over het ontwikkelen van facts valt buiten het doel van dit document en hier gaan we dus niet dieper op in, maar je kan meer informatie terugvinden op de website:\\
'http://projects.puppetlabs.com/projects/1/wiki/Adding\_Facts'.

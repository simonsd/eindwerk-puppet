\chapter{Puppet}

\section{Concept}
Het concept van configuration management is in weze simpel: het centraal beheren van configuraties zodat deze herbruikt kunnen worden op meerdere clients en zodat het gemakkelijker word dit allemaal te beheren. Stel het je maar eens voor, je bent een operator en je job bestaat uit het beheren van een netwerk of zelfs meerdere netwerken van machines. Dit kan gaan over een simpel thuisnetwerk ( 5-10 pc's ), over een iets geavanceerder kmo-netwerk ( 50-100 pc's ), tot zelfs een bedrijfsnetwerk van een multinational ( +1000 pc's ). Stel je even voor dat jij de enige beheerder bent van laatstgenoemde. Er komt een nieuwe software-update uit voor een kritieke applicatie of een kwetsbaarheid in je besturingssysteem, en het is jouw job om te verzekeren dat elke computer binnen het netwerk deze update krijgt. Met wat geluk kan je dit automatisch laten doen, ervan uitgaande dat dit altijd goed verloopt. Indien dit niet het geval is is de enige resterende optie manuele installatie van de update op elke computer. Tegen de tijd dat je rond bent kan je opnieuw beginnen en dan hebben we het enkel over onderhoud, de rest van je job dient dan ook nog gedaan te worden.\\\\
Een configuration management tool kan je dan helpen om dit immense aantal servers en clients te beheren op een min of meer geautomatiseerde manier. Bij puppet is het de bedoeling dat je "manifests" schrijft. Dit zijn configuratie-files waarin je bijvoorbeeld kan specifieren welke services, software-paketten en bestanden aanwezig of juist afwezig moeten zijn op een bepaald systeem en hoe deze ingesteld moeten worden. via een speciale node-definitie kan je dan aangeven welke classes een bepaald systeem moet meekrijgen. Een simpel voorbeeld is een webserver: hierop moet een applicatie draaien die het serven van web-pagina's mogelijk maakt, zoals apache. Met puppet kunnen we niet alleen verzekeren dat dit pakket aanwezig en geinstalleerd is op het doelsysteem, maar ook dat apache automatisch mee opstart en dat de configuratie ervan volgens een bepaalde template verloopt of van een andere server word gedownload.\\\\
Om dit alles te automatiseren is op zich niet zo'n groot probleem, het grootste probleem erbij is de beveiliging. Je kan uiteraard zonder beveiliging de configuraties ook beheren maar dan kunnen derden ook jouw configuraties bekijken en zien waar er eventuele zwakke punten zijn, om nog maar te zwijgen over eventuele reverse engineering tactieken waarmee ze via je configuration management systeem op je centrale server binnen geraken. Om dit alles te voorkomen maakt puppet gebruik van ssl-certificaten. SSL staat voor Secure Socket Layer, en zorgt ervoor dat er een beveligde verbinding gemaakt kan worden tussen jouw server en de clients die je wilt bedienen. De authenticatie verloopt de eerste keer (standaard) manueel. Vanaf dan gebeurt de gehele authenticatie-fase op de achtergrond. De authenticatie gebeurt via certificaten, namelijk een privaat en een publiek certificaat. Het private certificaat is enkel bekend door jouw server. Het publieke certificaat is zowel gekend door de client als de server. Bij de initiele configuratie van de beveiliging, stuurt de client zijn publieke certificaat naar de server, de server tekent dat certificaat met zijn private sleutel en stuurt het dan terug naar de client. Vanaf nu kan de client configuraties downloaden van de server op basis van zijn hostnaam, en de bijgeleverde publieke sleutel. De server kan dan het publieke certificaat met zijn private sleutel ontcijferen en kijken of deze geldig is of niet.

\subsection{Puppet}
Puppet is de naam van het commando dat men gebruikt om puppetruns te maken. Dit wil zeggen dat deze applicatie geen verbinding maakt met een puppetmaster om zijn configuratie op te halen. Omdat het puppet programma geen verbinding maakt met een server om zijn configuratie op te halen, dien je wel als argument het pad naar een manifest mee te geven. Deze manifest heeft dezelfde opmaak als een gewone manifest, en word dan ook hetzelfde geïnterpreteerd.

\subsection{Puppetd}
Puppetd is de puppet daemon. daemon is de UNIX naam voor een service, en zoals verwacht is deze puppetd dan ook een service die op geregelde tijdstippen zijn configuratie binnenhaalt van een puppetmaster en uitvoert.

\subsection{Puppetmaster}
puppetmaster is de naam van de puppet-server daemon. Dit wil zeggen dat we hier te maken hebben met een service die als server een dienst aanbied. In dit geval is die dienst het aanbieden van manifests aan clients, en hierbij gebruik te maken van een beveiligd kanaal. Dit beveiligd kanaal word toegepast door een ssl-verbinding aan te leggen, meer info hierover later.

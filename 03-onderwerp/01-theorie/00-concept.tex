\chapter{Puppet}

\section{Inleiding}
Bij puppet is het de bedoeling dat je "manifests" schrijft. Dit zijn configuratie-files waarin je bijvoorbeeld kan specifieren welke services, software-paketten en bestanden aanwezig of juist afwezig moeten zijn op een bepaald systeem en hoe deze ingesteld moeten worden. Via een speciale node-definitie kan je dan aangeven welke classes een bepaald systeem moet meekrijgen. Een simpel voorbeeld is een webserver: hierop moet een applicatie draaien die het serven van web-pagina's mogelijk maakt, zoals apache. Met puppet kunnen we niet alleen verzekeren dat dit pakket aanwezig en ge\"installeerd is op het doelsysteem, maar ook dat apache automatisch mee opstart en dat de configuratie ervan volgens een bepaalde template verloopt of van een andere server word gedownload.\\\\
%
Puppet is een configuration management applicatie, geschreven in Ruby. Het doel van puppet is het beheren van objecten aan de hand van 'manifests'. Binnen deze manifests word gebruik gemaakt van een DSL, een Domain-Specific Language, wat wil zeggen dat deze taal specifiek ontworpen is met een bepaalde taak in gedachten. In het geval van puppet is deze taak systeembeheer, wat wil zeggen het beheren van bestanden, applicaties, services, etc. De syntax van de puppet DSL heeft veel weg van Ruby, nogal logisch aangezien het daarop gebaseerd is. Door het schrijven van deze manifests kan je groepen van taken aaneenkoppelen, om zo je dagdagelijkse werk als systeembeheerder te verlichten. Dit is zeker handig met het oog op het onderhouden van systemen of het opzetten van nieuwe systemen: je kan namelijk met \'e\'en manifest meerdere malen dezelfde job uitvoeren op een veelvoud aan systemen. Enkele dingen die ik nu graag al zou verduidelijken is de benaming: 'puppet' is de naam van de applicatie als geheel, maar ook van het command-line programma 'puppet', waarmee je individuele manifests kan uitvoeren. De naam is opzettelijk met kleine letters geschreven, naar de richtlijnen van de UNIX filosofie.
%Indien je graag meer wil weten over UNIX, Linux, GNU of andere open-source toestanden, zit er een handvol bijlagen achteraan deze bundel die je daar zeker bij kunnen helpen.

\section{Onderdelen}

\subsection{Puppetmaster}
De puppetmaster (ook wel puppetmasterd genoemd) is de puppet server daemon. De puppetmaster dient als centrale server, die volledig gecompileerde configuraties, bestanden en templates aanlevert aan clients. In de meeste gevallen zal je dan ook steevast \'e\'en puppetmaster hebben waar een veelvoud aan puppet agents met verbinden.\\\\
Standaard zal de puppetmaster ook dienen als certificate authority (vanaf nu ook wel CA genoemd), wat wil zeggen dat de puppetmaster certificaten gaat uitdelen en beheren voor elk van zijn clients. Je hebt daarnaast ook de mogelijkheid een aparte certificate authority te gebruiken, zoals DogTag, OpenCA, CAcert of X.509 certificaten. Meer info daarover kan je terugvinden op 'http://en.wikipedia.org/wiki/Certificate\_authority\#Open\_source\_implementations'.\\\\
Net als de andere puppet applicaties, wordt de puppetmaster geconfigureerd in het '/etc/puppet/puppet.conf' bestand, in de '[puppetmaster]' sectie.

\subsection{Puppetd}
De puppet agent, meestal puppetd genoemd, is het programma dat in de achtergrond op elke host draait. Deze applicatie zal elke 30 minuten (afhankelijk van de instellingen) wakker worden, verbinding maken met de puppetmaster, informatie versturen naar de puppetmaster in verband met het hostsysteem (facts) en op basis daarvan een 'gecompileerde catalogus' toegestuurd krijgen, waarin de gewenste systeemconfiguratie beschreven staat. De puppet agent is dan verantwoordelijk voor het aanbrengen van veranderingen op het hostsysteem, zodat het systeem conform is met de toegestuurde catalogus. De puppetmaster is in dit geval verantwoordelijk voor het toesturen van de correcte catalogus, met andere woorden: een catalogus die enkel geschikt is voor het betreffende systeem, waarin geen informatie wordt vrijgegeven over andere hosts.\\\\
Configuratie van de puppet agent vind wederom plaats in het '/etc/puppet/puppet.conf' bestand, onder de sectie '[puppetd]'.

\subsection{Puppet}
Het simpelste deel van puppet is 'puppet apply', beter bekend als simpelweg puppet. Dit is een commando waarmee je op geheel afgeschermde wijze manifests kan uitvoeren, zonder nood te hebben aan een puppetmaster of zelfs een internetverbinding. Dit is vooral handig indien je een manifest wil uittesten of voor gevallen waar geen netwerkverbinding beschikbaar is. Aangezien het puppet programma geen verbinding maakt met een server om zijn configuratie op te halen, dien je wel als argument het pad naar een manifest mee te geven. Er is verder geen verschil tussen de puppet agent en puppet apply.

\subsection{Ralsh}
Ralsh is de afkorting van Resource Abstraction Layer SHell en biedt een interactieve manier om met puppet aan de slag te gaan. Met deze 'shell' heb je toegang tot alle ingebouwde puppet resource types, vanaf de command line. Zo kan je in je favoriete shell (op de meeste distribiuties is dit standaard BASH, de Bourne Again SHell), 'ralsh <resource type> <resource naam>' uitvoeren en als antwoord een overzicht krijgen in de stijl van een puppet manifest over de huidige status van het opgegeven object.

\subsection{SSL}
Het volgende onderdeel van puppet is SSL. SSL is het acroniem voor Secure Socket Layer en zorgt ervoor dat je een beveiligde verbinding kan maken naar een andere computer. Dit is ook het protocol dat gebruikt wordt bij de beveiliging van HTTP, het HyperText Transfer Protocol, waardoor het HTTPS genoemd word: HyperText Transfer Protocol Secure. In het geval van puppet wordt het gebruikt om de verbinding tussen de puppetmaster en de puppet client te beveiligen, zodat niemand kan zien wat je aan het uitvoeren bent of het kan aanpassen.
%
Om dit alles te automatiseren is op zich niet zo'n groot probleem, het grootste probleem erbij is de beveiliging. Je kan uiteraard zonder beveiliging de configuraties ook beheren maar dan kunnen derden ook jouw configuraties bekijken en zien waar er eventuele zwakke punten zijn, om nog maar te zwijgen over eventuele reverse engineering tactieken waarmee ze via je configuration management systeem op je centrale server binnen geraken. Om dit alles te voorkomen maakt puppet gebruik van SSL-certificaten. SSL staat voor Secure Socket Layer en zorgt ervoor dat er een beveiligde verbinding gemaakt kan worden tussen jouw server en de clients die je wilt bedienen. De authenticatie verloopt de eerste keer (standaard) manueel. Vanaf dan gebeurt de gehele authenticatie-fase op de achtergrond. De authenticatie gebeurt via certificaten, namelijk een privaat en een publiek certificaat. Het private certificaat is enkel bekend door jouw server. Het publieke certificaat is zowel gekend door de client als de server. Bij de initi\"ele configuratie van de beveiliging, stuurt de client zijn publieke certificaat naar de server, de server tekent dat certificaat met zijn private sleutel en stuurt het dan terug naar de client. Vanaf nu kan de client configuraties downloaden van de server op basis van zijn hostnaam en de bijbehorende publieke sleutel. De server kan dan het publieke certificaat met zijn private sleutel ontcijferen en kijken of dit geldig is of niet.

\subsection{Facter}
Facter is een onafhankelijke cross-platform Ruby library, ontworpen om informatie te verzamelen over nodes en is voorhanden voor elk besturingssysteem dat Ruby ondersteunt. Facter is een lichtgewicht programma dat informatie verzamelt over een host, dingen zoals de hostnaam, IP-adres, MAC-adres, SSH keys, hardware informatie, etc. Facter is ook zo ontworpen dat het compleet modulair is, zodat je zelf in een handomdraai je eigen "facts" kan schrijven. Je kan het in Puppet gebruiken om objecten of variabelen aan te passen op basis van een aantal criteria. De integratie gebeurt door simpelweg in de manifests een variabele op te roepen en op basis daarvan een keuze te maken, bijvoorbeeld:
\begin{code}
\begin{lstlisting}
package { 'apache':
	ensure => present,
	name => $operatingsystem ? {
		Debian => 'apache2',
		Centos => 'httpd',
	},
}
\end{lstlisting}
\end{code}
Hier zie je een declaratie van een package genaamd 'apache', waarbij afhankelijk van het besturingssysteem (operatingsystem), gekozen wordt voor een verschillende packagenaam. Je zal deze in de komende voorbeelden regelmatig zien terugkomen, waarbij meestal op basis van het besturingssysteem een keuze gemaakt wordt. Er zijn echter veel meer mogelijkheden, zoals bijvoorbeeld op basis van de hostnaam (naam van de machine) een andere configuratie toepassen, op basis van het IP-adres een template invullen, etc. Indien je met de ingebouwde 'facts' niet toekomt, kan je steeds zelf een 'fact' schrijven en deze gebruiken in je manifests. Een volledige uitleg over het ontwikkelen van facts valt buiten het doel van dit document en hier gaan we dus niet dieper op in, maar je kan meer informatie terugvinden op de website: 'http://projects.puppetlabs.com/projects/1/wiki/Adding\_Facts'.

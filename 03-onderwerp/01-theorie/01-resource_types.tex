% +++++ resource types +++++
Er zijn een aantal resource types die man kan gebruiken om dingen te definieren in puppet manifests.
Deze resource types variëren van het beheren van een bestand of map, tot het beheren van software-pakketten en services.


% +++++ cron +++++

	Deze resource type beheert cron jobs, de manier om op *NIX systemen een taak op bepaalde tijdstippen te laten uitvoeren.
	Hiermee heb je de mogelijkheid om bepaalde taken automatisch te laten uitvoeren op vooraf gedefiniëerde tijdstippen.
	Alle parameter met uizondering van de gebruiker en het uit te voeren commando zijn optioneel.

	De naam die aan een cron job meegeeft heeft enkel als identificatie als doel, voor de rest is deze naam compleet zonder mening.
	Als je bijvoorbeeld een tweede cron job specifieert die exact hetzelfde is als de eerste, dan zal puppet dit herkennen en deze twee namen simpelweg aan elkaar gelijkstellen zonder een tweede job toe te voegen.

	Enkele voorbeelden:

	cron { logrotate:
		command => "/usr/sbin/logrotate",
		user => root,
		hour => 2,
		minute => 0
	}

	Dit voorbeeld zou ervoor zorgen dat de gebruiker met naam 'root' om stipt 2u elke dag het commando '/usr/sbin/logrotate' uitvoert.
	Als extraatje kan je ook meerdere waardes doorgeven, dit doe je door middel van een array.

	cron { logrotate:
		command => "/usr/sbin/logrotate",
		user => root,
		hour => [2, 4]
	}

	Dit voorbeeld doet net hetzelfde als het vorige voorbeeld, met als enige verandering dat deze definitie zowel om 2 als om 4u het gespecifiëerde commando zal uitvoeren.
	Meer dan twee waardes zijn ook geen probleem, zolang ze netjes gescheiden zijn door een comma en binnen de vierkante haakjes zitten.

	Een laatste mogelijkheid om waardes mee te geven is via ranges, dit wil zeggen dat je alle waardes tussen waarde1 en waarde2 wil meegeven.

	cron { logrotate:
		command => "/usr/sbin/logrotate",
		user => root,
		hour => ['2-4'],
	}

	Met dit laatste voorbeeld voer je wederom het commando '/usr/sbin/logrotate' uit, maar nu zowel om 2, 3 als 4u.


	Parameters:

		command:
		Deze parameter specifiëert het uit te voeren commando, let wel dat je ofwel de absolute bestandsnaam moet opgeven, of een waarde aan 'path' moet meegeven zodat puppet het programma vind.
		De 'PATH' variabele word niet vanzelf overgedragen van de gebruiker dus als je een bepaalde 'path' wilt specifiëren dien je dit manueel te doen.

		ensure:
		Met deze parameter kan je specifiëren of iets aanwezig moet zijn, mag zijn, of juist niet mag zijn.
		Aanvaardbare waardes zijn bv.: present of absent.
		Als je present specifiëert zal de cron job aanwezig zijn, als je absent ingeeft zorgt puppet ervoor dat de cron job zeker niet aanwezig is.

		environment:
		Met deze parameter kan je omgevingswaarden meegeven. Let wel dat puppet deze niet automatisch reset, dus zullen deze omgevingsvariablen behouden blijven totdat puppet afsluit.

		hour:
		Het uur waarop de cron job dient te lopen.
		Aanvaardbare waardes: 1-23

		minute:
		De minuut waarop puppet de job zal starten.
		Aanvaardbare waardes: 1-59

		weekday:
		De dag van de week waarop men het commando uitvoert.
		Aanvaardbare waardes: 0-7 of de dagnaam ( Tuesday, Friday ).

		month:
		De maand waarop men het commando uitvoert.
		Aanvaardbare waardes: 1-12 of de maandnaam ( December, Januari )

		monthday:
		De dag van de maand waarop het commando uitgevoerd zal worden.
		Aanvaardbare waardes: 1-31

		name:
		De symbolische naam voor de cron job.
		word enkel gebruikt zodat mensen snel kunnen herkennen om welke job het juist gaat.
		
		provider:
		Het gewenste programma dat gebruikt word.
		Dit dient zelden gespecifiëerd te worden, puppet zal dit normaal gezien zelf herkennen.

		target:
		Waar de cron job opgeslagen dient te worden.
		Standaard is dit de crontab entry van de gebruiker zelf.

		user:
		De gebruiker waarmee men het commando dient uit te voeren.
		Deze valt terug op de huidige gebruiker indien geen gebruiker word meegegeven.

% +++++ exec +++++

	Exec is een resource type die toestaat externe commando's te gebruiken.
	Het belangrijkste waarbij je hier moet opletten is dat zo'n commando elke keer word uitgevoerd wanneer puppet zijn manifests uitvoert, het is dus belangrijk dat men geen commando's specifiëert die elkaar later overschrijven.
	Uiteraard zijn er ook wel manieren om te zorgen dat een commando slechts word uitgevoerd indien aan een bepaalde conditie vvoldaan word.
	Een andere manier om dit in te perken is de refreshonly parameter, die ervoor zorgt dat een commando enkel word uitgevoerd als reactie op een andere resource definitie binnen de manifests. Meer daarover later.

	Zoals te verwachten is word de exec resource veelal gebruikt om dingen te doen die (nog) niet in puppet ingebakken zijn.
	Alhoewel dit op korte termijn een vaak onoverkomelijk fenomeen zal zijn, word er toch sterk aangeraden om geëngageerd te geraken in de community rondom puppet en te helpen pushen voor een native resource type die de functies die jij nodig hebt kan uitvoeren.
	Natuurlijk is een eigen contributie van code ook altijd welkom.

	Je zal vaak bepaalde software-pakketten nodig hebben om commando's uit te voeren, en daar houd puppet rekening mee:
	puppet zal indien je in een commando een programma aanroept, kijken of dat programma door jou beheerd word en dit automatisch 'requiren'.
	Dit wil zeggen dat puppet bij het dynamisch opstellen van de volgorde waarin alles uitgevoerd word, rekening zal houden met het feit dat dit programma eerst geïnstalleerd moet worden en dan pas uitgevoerd kan worden.
	Hetzelfde word gedaan voor bijvoorbeeld de gebruiker waarmee men het commando uitvoert:
	Als deze door puppet beheerd word en nog niet bestaat, zorgt men ervoor dat deze gecreëerd word prior het commando uit te voeren.
	Deze functionaliteit noemt men ook wel 'auto-require'.

	Men kan ofwel de programma's aanroepen via hun absolute pad of een 'PATH' variable opgeven, die de paden bevat waarin puppet moet zoeken om het commando te vinden.
	Als alles goed gaat en het commando uitgevoerd word zonder fouten zal de output van het commando op het normale log-level gelogd worden, indien er een fout optreed zal puppet echter een melding weergeven en de eventuele output voordat de fout optrad.


	Parameters:
		command:
		Het commando dat uitgevoerd moet worden.

		creates:
		Een bestand dat dit commando creëert.
		Als hier een waarde opgegeven word zal het commando enkel uitgevoerd worden als dit bestand niet bestaat.

			exec { "tar xf /my/tar/file.tar":
				cwd => "/var/tmp",
				creates => "/var/tmp/myfile",
				path => ["/usr/bin", "/usr/sbin"]
			}

		In dit voorbeeld word het commando 'tar xf /my/tar/file.tar' uitgevoerd.
		Dit commando pakt een archief uit dat gecompresseerd is volgens het tar-protocol, in de map /var/tmp.
		De inhoud van dit archief zal een bestand 'myfile' bevatten, en wanneer dit uitgepakt word zal dit bestand zich in de map '/var/tmp' bevinden.
		Doordat puppet bij de volgende run gaat kijken of dit bestand bestaat zal het commando slechts één keer uitgevoerd worden, tenzij dit bestand verwijderd of verplaatst word.

		cwd:
		De map waarin het commando dient uitgevoerd te worden, als de map niet bestaat zal het commando falen.

		environment:
		Met deze parameter kan je omgevingsvariabelen meegeven aan het commando.

		group:
		De groep die gebruikt dient te worden om het commando uit te voeren.

		logoutput:
		Deze waarde specifiëert of men de output dient te loggen, gebruik hier de waarde on_failure om ervoor te zorgen dat er enkel gelogd word als het commando faalt.

		onlyif:
		Deze parameter kan men beschouwen als een test, indien de test slaagt zal het commando uitgevoerd worden, anders niet.
		De waarde van deze parameter is ook steeds een commando, wat zorgt voor een grote keuze uit testmogelijkheden.

			exec { "logrotate":
				path => "/usr/bin:/usr/sbin:/bin",
				onlyif => "test `du /var/log/messages | cut -f1` -gt 100000"
			}

		Dit voorbeeld voert het commando logrotate uit, dat hij zal gaan zoeken in '/usr/bin:/usr/sbin:/bin', enkel als 'test `du /vara/log/messages|cut -f1` -gt 100000' waar is.
		Dit wil zeggen dat dit commando kijkt of de centrale logmap groter is dan 100Mb en zal dan op basis daarvan beslissen of hij de logs gaat omwisselen of niet.

		path:
		Dit is de 'PATH' omgevingsvariable die gebruikt zal worden om te zoeken naar programma's, indien deze niet aanwezig is dient men bij een commando steeds de volledige padnaam naar een programma op te geven.

		provider:
		hiermee kan je specifiëren hoe je commando's wil uitvoeren: rechtstreeks (posix) of via een shell zodat de ingebouwde commando's van de shell beschikbaar zijn.

		refresh:
		Deze parameter zorgt ervoor dat als de exec aangeroepen word vanuit een andere resource, terwijl de parameter refreshonly ook is toegevoegd, er eventueel een ander commando in de plaats uitgevoerd kan worden.
		Normaal gezien zal het reguliere commando worden uitgevoerd, maar als hier een alternatief commando word opgegeven zal dit in de plaats worden gebruikt.

		refreshonly:
		Deze parameter zorgt ervoor dat het commando enkel uitgevoerd zal worden als een ander object waarvan dit object afhankelijk is verandert of uitgevoerd word.

			file { "/etc/aliases":
				source => "puppet://server/module/aliases"
			}

			exec { newaliases:
				path => ["/usr/bin", "/usr/sbin"],
				subscribe => File["/etc/aliases"],
				refreshonly => true
			}

		In dit voorbeeld zie je dat door middel van een subscribe parameter (meer info hier over later) dit commando enkel zal uitvoeren wanneer het object '/etc/aliases' gewijzigd word.

		returns:
		Hier kan je opgeven wat de verwachte return codes van het commando zijn, waardoor je bijvoorbeeld bepaalde fouten kan filteren.
		De standaardwaarde is enkel 0, dus enkel als een commando compleet foutloos uitgevoerd word.

		timeout:
		De maximum tijd die dit exec object in beslag zou mogen nemen.
		Als het commando langer duurt word er vanuit gegaan dat het commando gefaald heeft en zal dus gestopt worden.

		tries:
		Het aantal keren dat men dient te proberen het commando uit te voeren, met als standaardwaarde 1.
		Let wel dat de timeout parameter geld voor elk van deze tries apart, en dus niet als geheel.

		try_sleep:
		De pauze tussen opvolgende 'tries'.

		unless:
		Het omgekeerde van onlyif, namelijk dat het exec object enkel uitgevoerd zal worden indien dit commando wel faalt.

			exec { "/bin/echo root >> /usr/lib/cron/cron.allow":
				path => "/usr/bin:/usr/sbin:/bin",
				unless => "grep root /usr/lib/cron/cron.allow 2>/dev/null"
			}

		Dit voorbeeld laat zien hoe je een unless parameter specifiëert.
		Het exec object dat in dit voorbeeld gehanteerd word zal root toevoegen aan het '/usr/lib/cron/cron.allow' tenzij het woord 'root' er al in teruggevonden werd.

		user:
		De gebruiker die dit commando dient uit te voeren.

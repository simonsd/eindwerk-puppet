\chapter{Modules}

Een puppet module is een verzameling van resource types, classes, bestanden, defines en templates. Je kan bijvoorbeeld een module maken die een Ruby on Rails applicatie opzet, een Apache webserver of MySQL database. Modules zijn gemaakt om simpel te kunnen herdistribueren, zodat je je werk met andere mensen kan delen.
%
\section{Naamgeving}
Modulenamen zijn beperkt tot kleine letters, cijfers en liggende streepjes en moeten beginnen met een letter. Dezelfde beperkingen gelden voor class-namen, met de bijkomende beperking dat modulenamen geen scheidingsteken (::) mogen bevatten aangezien modules niet genest kunnen worden. Alhoewel modulenamen die niet conform deze regels ontworpen zijn kunnen werken, is het gebruik ervan niet aan te raden.
%
\section{Structuur}
Een puppet module bevat manifests, templates, plugins en bijbehorende bestanden, allemaal netjes geschikt in een voorgedefini\"eerde mappenstructuur:
%
\begin{code}
\begin{lstlisting}
MODULE\_PATH/
+--modulenaam/
   +--bestanden
   +--manifests
   |  +--init.pp
   |  +--foo.pp
   |
   +--lib
   |  +--puppet
   |  |  +--parser
   |  |  |  +--functions
   |  |  |
   |  |  +--provider
   |  |  +--type
   |  |
   |  +--facter
   |
   +--templates
   +--tests
   |  +--init.pp
   |  +--foo.pp
   |
   +--README
\end{lstlisting}
\end{code}
%
Elke module moet een 'init.pp' manifest bevatten in de manifests folder. Dit manifest kan alle classes bevatten die gebruikt worden in de module, of er kan per class een apart bestand aangemaakt worden in de manifests folder. Indien je gebruik maakt van meerdere bestanden om alle classes in onder te brengen, kan je ervoor zorgen dat ze automatisch herkend worden door de bestanden dezelfde namen mee te geven als de classes die ze herbergen. Dit noemt men 'auto lookup magic' (meer uitleg daarover in de sectie 'Zoeken').\\\\
%
Een van de belangrijkste principes waarop deze modules gebaseerd zijn is het delen van code. In het beste geval zou een module autonoom moeten zijn en dus niet afhankelijk van andere modules. Op die manier kan je een module downloaden, ze op de juiste plaats zetten (bijvoorbeeld ergens in je '\$MODULE\_PATH'), en ze gebruiken zonder verder nog iets te moeten aanpassen (met andere woorden: 'plug and play').\\\\
%
Uiteraard zijn er ook gevallen waarbij een module afhankelijk is van bepaalde onderdelen die ook bij andere modules handig kunnen zijn, bijvoorbeeld een web-applicatie die op de achtergrond een database gebruikt om bestanden op te slaan. In dit geval is het logischer een aparte database module te ontwikkelen die op zijn beurt herbruikt kan worden, dan een volledige database setup toe te voegen aan de huidige module. In dit geval kan je dit best documenteren in het README bestand, zodat andere mensen die deze module willen gebruiken weten wat ze mogen verwachten.
%
\section{Zoeken}
Aangezien modules submappen bevatten voor verschillende soorten bestanden, wordt er achter de schermen gezorgd dat het juiste bestand in de juiste context wordt geopend. Alle zoekopdrachten naar bestanden worden gedaan op basis van de '\$MODULEPATH' variabele. In de meeste gevallen zal hierdoor bij een zoekopdracht een manifest, bestand of template geleverd worden dat het dichtste bij de bestandsnaam aansluit en het meeste kans maakt het juiste bestand te zijn. Als je een bestand wil gebruiken kan je dus een modulenaam opgeven met daarachter een bestandsnaam, ook al zit dit bestand in een submap van de module.

Voor het opzoeken van bestanden op de een andere machine, wordt gebruik gemaakt van een URL die aan het begin de naam van de server meedraagt. Even een voorbeeld: 'puppet://\$servernaam/modulenaam/bestandsnaam'.

Om een module bruikbaar te maken voor zowel gebruik op de command line als via een puppetmaster, kan je gebruik maken van een URL in de vorm van: 'puppet:///pad/naar/module', waarbij je geen specifieke servernaam opgeeft. In dit geval zal de URL verschillend ge\"interpreteerd worden door puppet en puppetd, namelijk: puppet zal gaan zoeken naar een lokaal bestand, terwijl puppetd op de puppetmaster zal gaan kijken. Dit zorgt ervoor dat deze module zowel lokaal als op andere locaties gebruikt kan worden, zonder enige veranderingen aan te brengen.

Tot slot worden templates op een veelal vergelijkbare manier gezocht: indien je het argument template('autofs/auto.master.erb') opgeeft, zal puppet eerst zoeken naar '\$TEMPLATEDIR/autofs/auto.master.erb', gevolgd door '\$MODULEPATH/autofs/auto.master.erb'. Dit zorgt ervoor dat je zowel zeer specifieke templates kan aanbieden, als eerder abstracte configuraties.

Module Autoloading
In short, lookup paths within a module's manifest directory are derived by splitting class and definition names on :: separators, then interpreting the first element as the name of the module, the final element as the filename (with a .pp extension appended), and any intermediate elements as subdirectories of the module's manifests directory:

{module name}::{subdirectory}::{...}::{filename (sans extension)}

The one special case is that a one-wordt class or definition name which matches the name of the module will always be found in manifests/init.pp.1

\section{Documentatie}
Indien je besluit je modules beschikbaar te stellen voor andere mensen, wat sterk aangemoedigd wordt, is het handig dat je een beetje documentatie toevoegt. Deze documentatie kan zowel voor jezelf handig zijn om je geheugen even op te frissen over hoe een module net in elkaar zit, of voor iemand anders die deze module werkende probeert te krijgen. Het is vooral belangrijk dat je de andere modules waarvan deze module afhankelijk is duidelijk vermeld.
%
\section{Voorbeelden}
Hier bekijken we even enkele voorbeelden van echte modules. We maken hierbij gebruik van de exacte code die ik in het dagelijkse leven gebruik, maar wel met een beetje meer uitleg ertussen. Let er op dat dit hoogst waarschijnlijk niet de beste modules zijn die je kan vinden, deze zijn ook slechts gebouwd met mijn beperkte kennis en zijn bijlange na niet feature-complete. In de (nabije) toekomst zullen deze verder uitgebreid en verbeterd worden en ondertussen kan u gerust eens gaan kijken op de bovenvermelde website, waar u meer modules zal vinden.
%
\subsection{Apache}
Als eerste voorbeeld een Apache module:
%
\begin{code}
\begin{lstlisting}
Apache
+--manifests
|	+--config.pp
|	+--init.pp
|	+--mod-php.pp
|	+--packages.pp
|	+--passenger.pp
|	+--prefork.pp
|	+--xsendfile.pp
|
+--templates
	+--httpd.conf.erb
\end{lstlisting}
\end{code}
%
Dit is de basis mapstructuur voor deze module. Een aantal van de bestanden in de manifests map zijn niet strict noodzakelijk, maar vormen een mooie uitbreiding op het standaardaanbod aan ingebouwde modules voor Apache. We zullen nu per bestand bekijken wat er juist gebeurd, we doen dit niet op alfabetische volgorde maar op de volgorde die puppet normaal gezien ook zal volgen.
%
\subsubsection{init.pp}
Als eerste manifest hebben we 'init.pp'. Dit is het eerste manifest dat puppet zal uitlezen en waarin je normaal gezien alle onderdelen gerust kan onderbrengen. In dit geval word het echter enkel gebruikt om de overige bestanden te laden, ook al is dit niet strict noodzakelijk. Als alles meezit zal puppet zelf de andere bestanden herkennen en ze includeren, indien puppet dit niet doet door bijvoorbeeld een systeemfout, een foute configuratie of een (zeer) verouderde versie van puppet, gebruiken we dit bestand om de anderen te includeren.
\begin{code}
\begin{lstlisting}
include *.pp
\end{lstlisting}
\end{code}
%
\subsubsection{packages.pp}
Als tweede manifest hebben we packages.pp. zoals je ziet noemt de class die dit manifest bevat 'apache::packages' en dus niet gewoon 'packages' zoals je zou verwachten. Dit verwijst naar de modulenaam (apache) met als subclass daarvan 'packages'. Indien je je classes op deze manier defini\"eert zal puppet met zijn 'auto lookup magic' de classes vinden en automatisch includeren.\\\\
Allereerst zorgen we ervoor dat het pakket 'apache' ge\"installeerd is. De naam die we gebruiken om het pakket te installeren is afhankelijk van een variabele, die een verschillende naam zal doorgeven afhankelijk van de distributie waarop puppet het manifest uitvoert. In dit geval zijn de keuzemogelijkheden 'apache2' of 'httpd' indien we een distributie vinden die overeenkomt met een van de genoemden. Indien dit niet zo is zal puppet terugvallen op de naam die voorzien word door 'Default'. Er is uiteraard geen garantie dat deze van toepassing zal zijn op het huidige systeem, maar het biedt wel de mogelijkheid om bijvoorbeeld een standaardwaarde mee te geven.
\begin{code}
\begin{lstlisting}[tabsize=4]
class apache::packages {
	package { apache:
		ensure => installed,
		name => $operatingsystem ? {
			/Debian|Ubuntu/ => 'apache2',
			/Centos|Fedora/ => 'httpd',
			Default => 'apache',
		},
	}
\end{lstlisting}
\end{code}
%
Hier zorgen we dat de service met als alias 'apache\_daemon' draait en dat deze gestart wordt als de machine opstart. Wederom zien we een aparte naam voor de service per distributie.
\begin{code}
\begin{lstlisting}
	service { apache_daemon:
		ensure => running,
		enable => true,
		name => $operatingsystem ? {
			/Debian|Ubuntu/ => 'apache2',
			/Centos|Fedora/ => 'httpd',
		},
		require => Package['apache'],
	}

	package { 'apache-dev':
		ensure => installed,
		name => $operatingsystem ? {
			/Debian|Ubuntu/ => 'apache2-threaded-dev',
			/Centos|Fedora/ => 'httpd-devel',
		},
	}
}
\end{lstlisting}
\end{code}

\subsection{MySQL}
Een voorbeeld van een MySQL module:
%
\begin{code}
\begin{lstlisting}
mysql
+--manifests
   +--init.pp
   +--packages.pp
   +--config.pp
\end{lstlisting}
\end{code}
%
De mappenstructuur is bijna exact hetzelfde als bij de voorgaande apache module. Ook de onderverdeling van classes zit ongeveer hetzelfde in elkaar.
%
\subsubsection{init.pp}
Zoals je ziet wordt het 'init.pp' bestand wederom enkel gebruikt voor het includeren van de andere .pp bestanden, indien dit niet automatisch zou gebeuren.
%
\begin{code}
\begin{lstlisting}
include *.pp
\end{lstlisting}
\end{code}
%
\subsubsection{packages.pp}
Het installeren van de vereiste packages:
\begin{code}
\begin{lstlisting}
class mysql::packages {
	package { 'mysql-server':
		ensure => installed,
		name => $operatingsystem ? {
			Debian => 'mysql-server',
			Centos => 'mysql-server',
		},
	}

	service { 'mysqld':
		ensure => running,
		enable => true,
		name => $operatingsystem ? {
			Centos => 'mysqld',
			Debian => 'mysql',
		},
		require => Package['mysql-server'],
	}

	package { 'mysql-client':
		ensure => installed,
		name => $operatingsystem ? {
			Debian => 'mysql-client',
			Centos => 'mysql',
		},
		require => Package['mysql-server'],
	}

	package { 'mysql-dev':
		ensure => installed,
		name => $operatingsystem ? {
			Debian => 'libmysqlclient15-dev',
			Centos => 'mysql-devel',
		},
		require => Package['mysql-server'],
	}
}
\end{lstlisting}
\end{code}
%
\subsubsection{config.pp}
En tot slot de configuratie:
\begin{code}
\begin{lstlisting}
class mysql_config {
	exec { 'setup_mysql_pass_root':
		command => '/usr/bin/mysqladmin -uroot -h localhost password "penguins"',
		unless => '/usr/bin/mysql -uroot -h localhost',
		require => Class['mysql_packages'],
	}
}
\end{lstlisting}
\end{code}


Als we kijken naar de bestandsnaam die word doorgegeven aan de 'source' of 'content' parameters zie je dat deze veel veschillende vormen kan aannemen. Bij de source parameter wordt hier een bestandsnaam opgegeven die 'modules' bevat. Dit komt door het feit dat bestanden steeds uit modules worden gehaald. Dit is echter een recente toevoeging, bij vorige versies van puppet moest je dit bijvoorbeeld niet opgeven,
Notice that the file source path includes a modules/ component. In Puppet version 0.25 and later, you must include this component in source paths in order to serve files from modules. Puppet 0.25 will still accept source paths without it, but it will warn you with a deprecation notice about 'Files found in modules without specifying 'modules' in file path'. In versions 0.24 and earlier, source paths should not include the modules/ component.

Ook merk je dat bij de 'source' parameter een hostnaam wordt meegegeven v\'o\'or de bestandsnaam. Dit verwijst naar de server die het bestand aanbied, je kan hier bijvoorbeeld ook een externe 'http://' URL voor opgeven (alhoewel dit best te vermijden is, aangezien bestanden op het web wel eens durven verhuizen). In dit geval verwijst men naar een machine die in de DNS-records bekend staat onder de naam 'puppet'. In de meeste gevallen zou dit de puppetmaster moeten zijn. Bij de 'content' parameter zie je dit niet, alhoewel deze dezelfde regels volgt. Hier wordt gekozen voor een bestandsnaam in de vorm van: [modulenaam]/[bestandsnaam], wat ervoor zorgt dat puppet nu enkel zal gaan kijken in zijn eigen modulepad.
%
\begin{code}
\begin{lstlisting}
file { '/etc/auto.homes':
    source => 'puppet:///modules/autofs/auto.homes'
}
\end{lstlisting}
\end{code}

\subsection{Apache}
Als eerste voorbeeld een Apache module:
%
\begin{code}
\begin{lstlisting}
Apache
+--manifests
|	+--config.pp
|	+--init.pp
|	+--packages.pp
|
+--templates
	+--httpd.conf.erb
\end{lstlisting}
\end{code}
%
Dit is de basis mapstructuur voor deze module. We zullen nu per bestand bekijken wat er juist gebeurt, we doen dit niet op alfabetische volgorde maar op de meest logische volgorde die puppet normaal gezien ook zal volgen.
%
\subsubsection{init.pp}
Als eerste manifest hebben we 'init.pp'. Dit is het eerste manifest dat puppet zal uitlezen en waarin je normaal gezien alle onderdelen gerust kan onderbrengen. In dit geval word het echter enkel gebruikt om de overige bestanden te laden, ook al is dit niet strict noodzakelijk. Als alles meezit zal puppet zelf de andere bestanden herkennen en ze includeren, indien puppet dit niet doet door bijvoorbeeld een systeemfout, een foute configuratie of een (zeer) verouderde versie van puppet, gebruiken we dit bestand om de anderen te includeren.
\begin{code}
\begin{lstlisting}
include *.pp
\end{lstlisting}
\end{code}
%
\subsubsection{packages.pp}
Als tweede manifest hebben we packages.pp. zoals je ziet noemt de class die dit manifest bevat 'apache::packages' en dus niet gewoon 'packages' zoals je zou verwachten. Dit verwijst naar de modulenaam (apache) met als subclass daarvan 'packages'. Indien je je classes op deze manier defini\"eert zal puppet met zijn 'auto lookup magic' de classes vinden en automatisch includeren.\\\\
Allereerst zorgen we ervoor dat het pakket 'apache' ge\"installeerd is. De naam die we gebruiken om het pakket te installeren is afhankelijk van een variabele, die een verschillende naam zal doorgeven afhankelijk van de distributie waarop puppet het manifest uitvoert. In dit geval zijn de keuzemogelijkheden 'apache2' of 'httpd' indien we een distributie vinden die overeenkomt met een van de genoemden. Indien dit niet zo is zal puppet terugvallen op de naam die voorzien word door 'Default'. Er is uiteraard geen garantie dat deze van toepassing zal zijn op het huidige systeem, maar het biedt wel de mogelijkheid om bijvoorbeeld een standaardwaarde mee te geven.
\begin{code}
\begin{lstlisting}
class apache::packages {
	package { apache:
		ensure => installed,
		name => $operatingsystem ? {
			/Debian|Ubuntu/ => 'apache2',
			/Centos|Fedora/ => 'httpd',
			Default => 'apache',
		},
	}
\end{lstlisting}
\end{code}
%
Hier zorgen we dat de service met als alias 'apache\_daemon' draait en dat deze gestart wordt als de machine opstart. Wederom zien we een aparte naam voor de service per distributie. Ditmaal voegen we ook een 'require' toe, waardoor puppet ervoor zorgt dat de apache service pas gestart zal worden nadat het pakket ge\"installeerd is.
%
\begin{code}
\begin{lstlisting}
	service { apache_daemon:
		ensure => running,
		enable => true,
		name => $operatingsystem ? {
			/Debian|Ubuntu/ => 'apache2',
			/Centos|Fedora/ => 'httpd',
		},
		require => Package['apache'],
	}
\end{lstlisting}
\end{code}
%
Als laatste zien we nog een 'package' object staan. Deze package is niet strict noodzakelijk maar wordt veelal gebruikt voor het integreren van andere packages met apache. Aangezien dit vaak voorkomt voegen we deze ook toe in de basisinstallatie.
%
\begin{code}
\begin{lstlisting}
	package { 'apache-dev':
		ensure => installed,
		name => $operatingsystem ? {
			/Debian|Ubuntu/ => 'apache2-threaded-dev',
			/Centos|Fedora/ => 'httpd-devel',
		},
	}
}
\end{lstlisting}
\end{code}
%
\subsubsection{config.pp}
Deze manifest wordt gebruikt voor de configuratie van het apache pakket. In weze is dit slechts een voorbeeld om te laten zien hoe de configuratie verloopt, de echte configuratie zal normaal gezien gebeuren per project, eventueel op basis van dit bestand.
%
\begin{code}
\begin{lstlisting}
class apache::config {
	file { 'apache.conf':
		ensure => present,
		owner => root,
		group => root,
		mode => 0644,
		name => $operatingsystem ? {
			/Debian|Ubuntu/ => '/etc/apache2/apache2.conf',
			/Centos|Fedora/ => '/etc/httpd/conf/httpd.conf',
		},
		content => template('apache/httpd.conf'),
		notify => Service['apache'],
	}
}
\end{lstlisting}
\end{code}
%
Zoals je ziet wordt de configuratie gedaan via \'e\'en enkel bestand, dat afhankelijk van het besturingssysteem op een andere plaats te vinden is. We zorgen met deze definitie dat het bestand aanwezig is, zowel als eigenaar als groep aan 'root' toebehoort en als mode '0644' meekrijgt. Het effectieve bestand halen we uit de template map van de apache module en heet 'httpd.conf'. Tot slot zorgen we ervoor dat de service 'apache' wordt herstart telkens er een wijziging aan dit bestand wordt aangebracht.
%
\subsubsection{httpd.conf}
Dit is een template, geschikt om te dienen als basis apache configuratie bestand. Aangezien dit geen onderdeel is van puppet, maar bij de configuratie van Apache thuishoort, zullen we dit configuratie-bestand hier niet ontleden. Weet alleen dat je met dit bestand een basis-webserver kan opzetten, aan de hand van een aantal veilige standaardwaardes. Aangezien dit bestand standaard 1009 regels bevat, zal ik het hier achterwege laten en het bijvoegen in de bijlages.

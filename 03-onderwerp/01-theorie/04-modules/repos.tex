\subsection{Repos}
Dit is de repos module, een zeer basale module voor het beheren van enkele (voorlopig alleen centos) repositories.
%
\begin{code}
\begin{lstlisting}
repos
+--manifests
   +--centos.pp
   +--init.pp
\end{lstlisting}
\end{code}
%
\subsubsection{init.pp}
Zoals bij de meeste modules zien we hier weer een 'catchall' definitie voor de bijbehorende manifests.
%
\begin{code}
\begin{lstlisting}
include *.pp
\end{lstlisting}
\end{code}
%
\subsubsection{centos.pp}
Die is waar de centos repositories gespecifi\"eerd staan.
\begin{code}
\begin{lstlisting}
class repos::centos {
	yumrepo { 'rpmforge':
		baseurl => 'http://apt.sw.be/redhat/el5/en/i386/rpmforge/RPMS/rpmforge-release-0.3.6-1.el5.rf.i386.rpm',
		descr => 'rpmforge repo',
		enabled => true,
	}

	yumrepo { 'epel':
		baseurl => 'http://be.mirror.eurid.eu/epel',
		descr => 'epel',
		enabled => 1,
	}

	yumrepo { "psql":
		baseurl => "http://yum.pgrpms.org/8.4/redhat/rhel-5-x86_64",
		descr => "postgresql yum repo",
		enabled => 1,
		gpgcheck => 0,
		require => Exec["exclude_pgsql_from_base"],
	}

	yumrepo { "inuits":
		baseurl => "http://repo.inuits.be/centos/5/os",
		descr => "inuits internal repo",	
		enabled => 1,
		gpgcheck => 0,
	}
}
\end{lstlisting}
\end{code}

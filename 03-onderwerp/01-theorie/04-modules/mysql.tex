\subsection{MySQL}
Een voorbeeld van een MySQL module:
%
\begin{code}
\begin{lstlisting}
mysql
+--manifests
   +--init.pp
   +--packages.pp
   +--config.pp
\end{lstlisting}
\end{code}
%
De mappenstructuur is bijna exact hetzelfde als bij de voorgaande apache module. Ook de onderverdeling van classes zit ongeveer hetzelfde in elkaar.
%
\subsubsection{init.pp}
Zoals je ziet wordt het 'init.pp' bestand wederom enkel gebruikt voor het includeren van de andere .pp bestanden, indien dit niet automatisch zou gebeuren.
%
\begin{code}
\begin{lstlisting}
include *.pp
\end{lstlisting}
\end{code}
%
\subsubsection{packages.pp}
Het installeren van de vereiste packages:
\begin{code}
\begin{lstlisting}
class mysql::packages {
	package { 'mysql-server':
		ensure => installed,
		name => $operatingsystem ? {
			Debian => 'mysql-server',
			Centos => 'mysql-server',
		},
	}

	service { 'mysqld':
		ensure => running,
		enable => true,
		name => $operatingsystem ? {
			Centos => 'mysqld',
			Debian => 'mysql',
		},
		require => Package['mysql-server'],
	}

	package { 'mysql-client':
		ensure => installed,
		name => $operatingsystem ? {
			Debian => 'mysql-client',
			Centos => 'mysql',
		},
		require => Package['mysql-server'],
	}

	package { 'mysql-dev':
		ensure => installed,
		name => $operatingsystem ? {
			Debian => 'libmysqlclient15-dev',
			Centos => 'mysql-devel',
		},
		require => Package['mysql-server'],
	}
}
\end{lstlisting}
\end{code}
%
\subsubsection{config.pp}
En tot slot de configuratie:
\begin{code}
\begin{lstlisting}
class mysql_config {
	exec { 'setup_mysql_pass_root':
		command => '/usr/bin/mysqladmin -uroot -h localhost password "penguins"',
		unless => '/usr/bin/mysql -uroot -h localhost',
		require => Class['mysql_packages'],
	}
}
\end{lstlisting}
\end{code}

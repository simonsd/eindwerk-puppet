\chapter{Configuration Management}
\section{Concept}
Het concept van configuration management is in weze simpel: het centraal bewaren en beheren van configuraties zodat deze herbruikt kunnen worden door meerdere clients. Beeld het je even in: jij bent een operator en je job bestaat uit het beheren van \'e\'en of meerdere netwerken. Dit kan gaan over een simpel thuisnetwerk (5 tot 10 machines), over een iets uitgebreider kmo-netwerk (50 tot 100 machines), tot zelfs een bedrijfsnetwerk van een multinational (meer dan 1000 machines). Wat als jij de enige beheerder bent van zulk een netwerk? Er komt een nieuwe software-update uit voor een missie-kritieke applicatie of een kwetsbaarheid in je besturingssysteem en het is jouw job om te verzekeren dat elke computer binnen het netwerk deze update krijgt. Met wat geluk kan je dit automatiseren, ervan uitgaande dat dit altijd goed verloopt. Indien dit niet het geval is is de enige resterende optie manuele installatie van de update op elke computer. Tegen de tijd dat je rond bent kan je opnieuw beginnen en dan hebben we het enkel over onderhoud, los daarvan zal je ook nog nieuwe machines moeten opzetten, nieuwe technologi\"een aanleren en aanwenden en zorgen dat in geval van nood alles snel weer in orde gebracht kan worden. In dit geval is configuration management software een absolute must, aangezien zulke tools je kunnen helpen bij zowel het opzetten van nieuwe machines (deployment), het onderhouden van machines (maintenance), het maken en nakijken van backups en dit allemaal terwijl jij de nieuwste technieken aanleert. Verder kan het ook dienen als abstractielaag voor je hele infrastructuur, aangezien de kans klein is dat alle machines die je beheert exact dezelfde hard- en software bevatten.

\section{Geschiedenis}
De geschiedenis van configuration management kan men traceren naar de jaren '50, toen het ontwikkeld werd door de Amerikaanse luchtmacht als een inventaris systeem, zodat ze konden bijhouden waar bepaalde onderdelen zich bevonden. Mettertijd is dit getransformeerd in iets aanzienlijk groter dan slechts een inventaris systeem, namelijk een systeem om niet alleen bij te houden waar objecten zich bevinden en in welke staat ze verkeren, maar om dit ook centraal te beheren.\\\\
Sommige mensen durven wel eens te discussi\"{e}ren dat configuration management enkel het inventariseren van objecten omvat en dat het controleren en manipuleren van configuraties een compleet andere discipline is, zoals bv.: change management. Uiteraard zijn dit allemaal slechts een hoop marketing termen die verder niet veel betekenis hebben.

\section{Hedendaags}
De term configuration management omvat tegenwoordig zowel het inventariseren als het aanpassen van configuraties, met name op computersystemen. De dag van vandaag wordt configuration management vooral gebruikt voor het implementeren van nieuwe systemen en het onderhoud ervan. Oorspronkelijk werd configuration management manueel gedaan, later begon men software tools te schrijven om dit meer te kunnen automatiseren.\\
Enkele voorbeelden van zulke tools zijn:
\begin{itemize}
\item cfengine
\item bcfg2
\item chef
\item puppet
\end{itemize}
Uiteraard hebben al deze tools hun eigen voor- en nadelen, we bekijken ze even:
%
\subsection{Cfengine}
Cfengine is veruit de oudste, deze is namelijk ontwikkeld door Mark Burgess in 1993. Cfengine is opgebouwd in de programmeertaal 'C' en is daardoor zeer snel bij het uitvoeren, maar niet zo simpel om te ontwikkelen. De grootste nadelen aan cfengine zijn het gebrek aan high-level structuren, wat wil zeggen dat enkel bestanden beheert kunnen worden. In principe is het mogelijk alles te configureren via bestanden, maar high-level structuren maken dit veel simpeler. Een ander nadeel is de syntax van cfengine, deze is namelijk niet zo eenvoudig.
%
\subsection{Bcfg2}
Bcfg2 is een project dat ontwikkeld is aan de afdeling wiskunde en computerwetenschappen van de Argonne National Laboratory, het alleerste offici\"ele onderzoeksinstituut in Amerika. De tool is geschreven in Python en wordt aangeboden onder een BSD licentie. De allereerste versie kwam uit in 2004 en wordt tot op heden nog steeds actief ontwikkeld. In principe is bcfg2 meer ontwikkeld om configuraties te evalueren dan ze te beheren. Zo zal je bij een standaard bcfg2 run enkel de configuratie van een machine of netwerk van machines bekijken, maar geen wijzigingen doorvoeren. Ook al is bcfg2 geschikt om de configuraties toe te passen, is dit slechts een bijkomende optie en wordt daardoor meestal niet gekozen bij een stand-off tussen twee configuration management tools.
%
\subsection{Chef}
Chef is veruit de jongste, ontwikkeld in 2009 en opgebouwd in Ruby. De applicatie wordt aangeboden onder de Apache licentie, wat het makkelijk en flexibel maakt om op veel verschillende platformen aan te bieden. De ontwikkeling begon vooral uit onvrede over het aanbod aan configuration management software, alhoewel er op dat moment al een aantal zeer respectabele tools op de markt waren. Chef maakt gebruik van een DSL (Domain Specific Language), net zoals puppet, maar staat los daarvan ook programmatie toe in pure Ruby. De mogelijkheid om ook in Ruby zelf configuraties te schrijven is zeer aantrekkelijk voor programmeurs met enige ervaring in de taal, maar voor system administrators zonder enige ervaring is het in de meeste gevallen een te steile leercurve.
%
\subsection{Puppet}
Als laatste voorbeeld nemen we puppet, waar we in dit eindwerk veel dieper op zullen ingaan. Puppet is oorspronkelijk ontwikkeld in 2005 en is eveneens opgebouwd in Ruby. Zoals net al gezegd werdt maakt ook puppet gebruik van een DSL, losjes gebaseerd op de Ruby syntax. De taal zit echter zo simpel in elkaar, dat puppet het mogelijk maakt om op een snelle en simpele manier, grote projecten uit te rollen. Deze simpliciteit en kracht heeft ervoor gezorgd dat we er bij Inuits zeer veel gebruik van maken, waardoor ik ook de kans kreeg puppet onder handen te nemen. In dit eindwerk zullen we puppet bekijken als configuration management systeem, zowel in theorie als in de praktijk.

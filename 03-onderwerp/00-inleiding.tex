\chapter{Configuration Management}

De keuze voor mijn eindwerk is gevallen op puppet, een configuration management tool. Configuration management omvat het centraal beheren van configuraties en deze distribueren van je server naar een of meerdere clients.


\section{Geschiedenis}

De geschiedenis van configuration management kan men traceren naar de jaren '50, toen het ontwikkeld werd door de Amerikaanse luchtmacht als een inventaris systeem, zodat ze konden bijhouden waar bepaalde onderdelen zich bevonden. Mettertijd is dit getransformeerd in iets aanzienlijk groter dan slechts een inventaris systeem, namelijk een systeem om niet alleen bij te houden waar objecten zich bevinden en in welke staat ze verkeren, maar om dit ook centraal te beheren.\\\\
Sommige mensen durven wel eens te discussi\"{e}ren dat configuration management nog steeds enkel het inventariseren van objecten omvat en dat het controleren en manipuleren van configuraties een compleet andere discipline is, zoals bv.: change management. Uiteraard zijn dit allemaal slechts een hoop marketing termen die verder niet veel betekenis hebben. De term configuration management omvat op de dag van vandaag zowel het inventariseren als het aanpassen van configuraties, met name op computersystemen.\\\\
Oorspronkelijk werd configuration management manueel gedaan, later begon men software tools te schrijven om dit meer te kunnen automatiseren.\\
Enkele voorbeelden van zulke tools zijn:
\begin{itemize}
\item cfengine
\item chef
\item puppet
\end{itemize}
\section{Concept}
Het concept van configuration management is in weze simpel: het centraal bewaren en beheren van configuraties zodat deze herbruikt kunnen worden door meerdere clients. Stel het je maar eens voor, je bent een operator en je job bestaat uit het beheren van een netwerk of zelfs meerdere netwerken van machines. Dit kan gaan over een simpel thuisnetwerk ( 5-10 pc's ), over een iets geavanceerder kmo-netwerk ( 50-100 pc's ), tot zelfs een bedrijfsnetwerk van een multinational ( +1000 pc's ). Wat als jij de enige beheerder bent van zulk een netwerk? Er komt een nieuwe software-update uit voor een missie-kritieke applicatie of een kwetsbaarheid in je besturingssysteem en het is jouw job om te verzekeren dat elke computer binnen het netwerk deze update krijgt. Met wat geluk kan je dit automatiseren, ervan uitgaande dat dit altijd goed verloopt. Indien dit niet het geval is is de enige resterende optie manuele installatie van de update op elke computer. Tegen de tijd dat je rond bent kan je opnieuw beginnen en dan hebben we het enkel over onderhoud, de rest van je job dient dan ook nog gedaan te worden.\\\\

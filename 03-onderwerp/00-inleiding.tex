\chapter{Configuration Management}

De keuze voor mijn eindwerk is gevallen op puppet, een configuration management tool. Configuration management omvat het centraal beheren van configuraties en deze distribu\"{e}ren van je server naar een of meerdere clients.


\section{Geschiedenis}

De geschiedenis van configuration management kan men traceren naar de jaren '50, toen het ontwikkeld werd door de Amerikaanse luchtmacht als een inventaris systeem, zodat ze konden bijhouden waar bepaalde onderdelen zich bevinden. Mettertijd is dit getransformeerd in iets aanzienlijk groter dan slechts een inventaris systeem, namelijk een systeem om niet alleen bij te houden waar objecten zich bevinden en in welke staat ze verkeren, maar om dit ook centraal te beheren.\\\\
Sommige mensen durven wel eens te discussi\"{e}ren dat configuration management nog steeds enkel het inventariseren van objecten omvat, en dat het controleren en manipuleren van configuraties een compleet andere discipline is, zoals bv.: change management. Uiteraard zijn dit allemaal slechts een hoop marketing termen die verder niet veel betekenis hebben. De term configuration management omvat op de dag van vandaag zowel het inventariseren als het aanpassen van configuraties, met name op computersystemen.\\\\
Oorspronkelijk werd configuration management manueel gedaan, later begon men software tools te schrijven om dit meer te kunnen automatiseren.\\
Enkele voorbeelden van zulke tools zijn:
\begin{itemize}
\item cfengine
\item chef
\item puppet
\end{itemize}

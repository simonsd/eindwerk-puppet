\chapter{Praktijk}
In dit hoofdstuk zullen we bekijken hoe je een puppet server en client opzet, waarna we aan de hand van enkele simpele voorbeelden een manifest zullen samenstellen om de client en zelfs de server zelf te beheren.\\\\
Allereerst enkele concessies: Deze tutorial is in principe enkel gegarandeerd te werken op Fedora 14, de linux-distributie die ik zelf gebruik. Het meeste ervan zal echter wel overdraagbaar zijn naar toekomstige versies en/of andere distributies. Als je niet weet wat een linux-distributie is raad ik je aan om het even op te zoeken op het internet, er is genoeg informatie voorhanden en het is niet het doel van dit eindwerk om je dat allemaal uit te leggen dus doe ik dat ook niet.\\\\
Ik ga er vanuit dat je voor je hieraan begint een werkend Fedora 14 (of vergelijkbare) distributie al geinstalleerd hebt. Moest dit niet het geval zijn kan je altijd even kijken op de website www.howtoforge.com . Die website is een rijke bron van informatie in verband met installatie van een veelvoud aan paketten op linux en heeft dan ook al een grote hoeveelheid bijgedragen aan mijn kennis.


\section{Installatie}
De eerste stap die we zetten is het installeren van zowel puppet zelf (client) als de puppetmaster (server). Er zijn meerdere manieren om dit te doen, namelijk:
\begin{itemize}
\item source code
\item gem
\item native package
\end{itemize}

\subsection{Source code}
De eerste optie is het installeren vanaf source code. De source code is de onderliggende code-basis van de applicatie, in dit geval geschreven in Ruby. Het installeren van source omvat het bemachtigen van een kopie van de source en deze dan manueel op de juiste plaatsen zetten en eventuele installatieprocedures uitvoeren. Dit is veruit de meest omslachtige manier, maar ook de manier waarbij je jezelf ervan kan verzekeren dat wat je installeert de meest up to date, niet aangepaste code is.
%
\subsection{Gem}
De tweede optie omvat het installeren van een gem, een soort Ruby 'addon'. Dit doe je via het programma 'gem', dat meestal in het standaard Ruby pakket zit of apart verkrijgbaar als het pakket 'ruby-gems'. Je kan gem beschouwen als een soort package manager voor Ruby addons, een programma dat deze addons kan opzoeken, installeren en up-to-date houden. Door hiervan gebruik te maken is de kans ook nog zeer groot dat je een recente versie van de benodigde pakketten zal krijgen, alhoewel hier enkele dagen of zelfs weken verschil op kan zitten. De installatie via het gem programma verloopt al een stuk simpeler in vergelijking met de installatie vanaf source code en bestaat enkel uit het commando 'gem install puppet puppetmaster' uit te voeren als de root-gebruiker (administrator).
%
\subsection{Native package}
De laatste optie bestaat uit het installeren van native packages, wat wil zeggen dat de source code verpakt is in het formaat dat gebruikelijk is voor jouw distributie. Bij Fedora/CentOS/RedHat is dit bijvoorbeeld het RPM-formaat, bij Debian/Ubuntu is dit het DEB-formaat en bij Archlinux dan weer het PKG-formaat. Het voordeel hiervan is dat je zeker kan zijn dat deze packages niet in de weg zullen liggen van andere packages, alle dependencies netjes worden opgelost en de packages veelal vanzelf up-to-date gehouden worden. Het grote nadeel daarentegen is dat je hoogstwaarschijnlijk een tijdje zal moeten wachten voor je de laatste nieuwe versies van de code kan verwachten, doordat men de code moet verpakken in vooraf gedefini\"{e}erde formaten en het dan nog even duurt voordat deze degelijk getest kunnen worden en stabiel verklaard worden. De wachttijd is een beetje afhankelijk van de bereidwilligheid van de distributie; Bij Fedora zal je hoogstwaarschijnlijk enkele weken tot zelfs maanden moeten wachten voor je de versie die vandaag uitgebracht wordt in je package manager zal zien verschijnen. Bij Ubuntu zal dit ook nog ettelijke weken duren, alhoewel dit door de grote community meestal snel wordt aangepakt. Bij Archlinux daarentegen kan je de update al een dag later verwachten maar uiteraard zonder garanties: als het nieuwe pakketje iets kapot maakt op je systeem, is dit enkel jouw verantwoordelijkheid. Als je meer wil weten over package managers en hoe je deze gebruikt, kan je in de bijlages een 'cheat sheet' terugvinden met daarop een overzicht van de meest gebruikelijke opties van grote package managers. Met wat oefening kan je daarmee in geen tijd aan de slag met het installeren van software op eender welk linux systeem.

Als we even de tijd nemen om dit commando te ontleden zien we dat het bestaat uit meerdere onderdelen, namelijk: sudo => sudo is de afkorting van 'switch user do' en doet zoveel als van gebruiker veranderen om dan een bepaald commando uit te voeren. We gebruiken dit omdat je als gewone gebruiker normaal gezien niet zomaar paketten mag installeren, dit is voorbehouden voor de root-gebruiker. de root-gebruiker is de unix-variant van de administrator account onder windows, maar dan echt almachtig. zijn wil is wet en daar is geen weg rond, dus ook niet als je per ongeluk als root de helft van je data weg vaagt, pas dus op wanneer en vooral hoe je de root-account gebruikt. De reactie van dit commando hangt echter ook af van de configuratie ervan, standaard zal deze bij gebruik een paswoord vragen, het paswoord van de root-account. Er is echter ook de mogelijk dat je volledige rechten hebt gekregen op dit programma en dat je zelfs zonder paswoord de kracht van de root-account kan gebruiken. Dit is enorm gevaarlijk en dan ook enkel aan te raden wanneer je weet met wat je bezig bent.

het tweede deel is 'yum': yum is de standaard package manager van Fedora 14 en is de afkorting van 'yellowdog updater, modified'. De naam is afkomstig van de linux-distributie die deze front-end ontwikkeld heeft, namelijk 'yellowdog linux' oftewel 'ydl'. Deze wordt gebruikt om paketten te installeren en deinstalleren en uiteraard om bij te houden welke paketten al geinstalleerd zijn en welke beschikbaar zijn voor jouw systeem. In weze is yum slechts een front-end voor rpm, de 'redhat package manager'. Het grootste probleem met rpm was dat er geen dependency resolving plaatsvond en dat zorgt er dus voor dat als het paketje dat jij wilt installeren afhankelijk is van een of meerdere andere paketten, je eerst manueel deze andere paketten moest gaan installeren. Niet  echt een leuk karweitje als je er even over nadenkt. yum zorgt ervoor dat dit dus wel automatisch gedaan wordt en voegt daarnaast nog enkele handige mogelijkheden toe zoals plugins. door middel van deze plugins kan je extra functionaliteit toevoegen aan yum, denk bijvoorbeeld aan de plugin fastestmirror. Deze plugin zorgt ervoor dat eerst wordt vastgesteld welke mirror op dit moment het snelst aan jouw behoeften kan voorzien en zorgt ervoor dat jij van deze mirror gebruik maakt, zonder dat je daarvoor iets hoeft te doen. Bij fedora 14 is deze plugin trouwens standaard al aanwezig.

Daarna komt de schakeloptie '-y': deze specifieert dat yum op alle vragen die hij normaal gezien zou stellen, simpelweg 'ja' antwoord. Dit is uiteraard handig, maar pas er wel een beetje mee op, zeker bij het verwijderen van paketten. Zoals ik zelf al hardhandig heb mogen ondervinden verwijderd hij dan bijvoorbeeld ook de dependencies van het paketje dat je probeert te verwijderen en uiteraard ook alle paketten die daar ook afhankelijk van zijn. 

Het vierde deel van het commando is 'install': dit is een actie binnenin het programma yum. Deze actie zorgt ervoor, zoals het woord al insinueert, dat er iets geinstalleerd moet worden. yum zorgt zelf voor de rest, zoals de keuze van welke repository gebruikt word, de opties bij het installeren, etc ..

Last but not least hebben we 'puppet puppet-server': Dit zijn simpelweg de namen van de paketten die je wenst te installeren. Je kan er meerdere specifieren, zolang ze gescheiden zijn door een spatie. In het geval jouw pakketje een naam zou hebben met een spatie erin, dien je dus haakjes te gebruiken om de pakketnaam te 'quoten' < ja inderdaad zoals dit.

In zijn geheel zorgt dit commando er dus voor dat de pakketten 'puppet' en 'puppet-server' geinstalleerd worden op jouw systeem zonder al te veel gemor over dependencies en dergelijke. Als het commando de command line terugkeert, zijn de pakketten geinstalleerd en kunnen we beginnen met de configuratie van de puppetmaster en daarna de puppet client.

\section{Configuratie}

\subsection{Configuratiebestanden}
De eerste mogelijkheid is het configureren van puppet via de configuratiebestanden die je terugvind in de configuratiemap. De configuratiemap is afhankelijk van de gebruiker die puppet uitvoert, voor de root gebruiker zal de standaardlocatie '/etc/puppet' zijn, voor een andere gebruiker zal dit '\$HOME/.puppet' zijn. In elk geval wordt er steeds gezocht naar het bestand 'puppet.conf', dat de configuratie van zowel de puppetmaster, puppet en puppetd bevat. Parameters die in de '[main]' sectie worden opgegeven zullen gebruikt worden door alle programma's, terwijl ook afzonderlijke secties beschikbaar zijn voor 'master' en 'agent' programma's. Enkel de puppetmaster zal gebruik maken van de '[master]' sectie, waar de andere applicaties gebruik zullen maken van de '[agent]' sectie. Het bestand zelf wordt opgesteld in init-stijl. Even een voorbeeld van een zeer simplistisch 'puppet.conf' bestand:
\begin{code}
\begin{lstlisting}
[main]
  confdir = /private/puppet
  storeconfigs = true
  vardir = /new/vardir {owner = root, mode = 644}
\end{lstlisting}
\end{code}
%
Indien je puppet voor de eerste keer gebruikt kan het wenselijk zijn om de gebruikte applicatie een configuratie voor je te laten genereren. Dit kan je doen door de '--genconfig' parameter mee te geven aan het programma, waardoor een volledige standaardconfiguratie naar de standard output (je scherm in dit geval) geschreven wordt. Deze kan je dan laten opvangen in een bestand, door na het commando een '>' teken te specifi\'eren met een padnaam. Op het opgegeven pad zal een bestand gecre\"eerd worden met als inhoud de configuratie die daarvoor naar het scherm werd gestuurd. Even een voorbeeld:
\begin{code}
\begin{lstlisting}
puppet agent --genconfig > /etc/puppet/puppet.conf
\end{lstlisting}
\end{code}
Dit commando zorgt ervoor dat een werkende configuratie wordt weggeschreven naar het '/etc/puppet/puppet.conf' bestand, wat in de meeste gevallen als standaard configuratiebestand aanschouwt wordt. Let er wel op dat dit commando de huidige inhoud van het bestand compleet overschrijft, dus maak best even een backup of gebruik '>>' in plaats van '>', om de gegenereerde configuratie te laten toevoegen aan het bestand. Een interessante functie van puppet is de mogelijkheid om zijn eigen gebruiker en bijbehorende groep aan te maken. Je kan het gebruiken als volgt:
\begin{code}
\begin{lstlisting}
puppet master --mkusers
\end{lstlisting}
\end{code}

\subsection{Parameters}
Hier geven we een kort overzicht van enkele van de parameters die je kan gebruiken in de configuratiebestanden. Een volledige lijst kan je terugvinden op: 'http://docs.puppetlabs.com/references/latest/configuration.html'.\\\\
%
autosign:\\
Of je certificate requests automatisch wil laten tekenen of niet. Geldige waardes zijn hierbij 'true' (zorgt ervoor dat eender welke aanvraag goedgekeurd wordt en is meestal een zeer slecht idee), 'false' (weigert alle requests) en een pad naar een bestand. In dit geval zal de inhoud van het bestand bepalen welke requests automatisch goedgekeurd mogen worden en welke niet. De standaardwaarde is hierbij '\$CONFDIR/autosign.conf'.\\\\
%
confdir:\\
Instellen van de '\$CONFDIR' variabele, waarmee je een alternatieve waarde opgeeft voor de map die de configuratie van puppet bevat.\\\\
%
configtimeout:\\
Hoe lang puppet mag wachten op een configuratie. Als er na deze periode nog steeds geen configuratie ontvangen is zal puppet falen.\\\\
%
daemonize:\\
Of je de applicatie op de achtergrond wil draaien of niet.\\\\
%
dbadapter:\\
Het type database dat je wenst te gebruiken.\\\\
%
dbconnections:\\
Het maximum aantal verbindingen naar de database, indien deze niet lokaal is.\\\\
%
dbname:\\
De databasenaam die je wenst te gebruiken.\\\\
%
dbpassword:\\
Het paswoord dat je wil gebruiken voor de database.\\\\
%
dbport:\\
De poortnummer waarop je de database wil benaderen.\\\\
%
dbserver:\\
Met welke database server je wil verbinden.\\\\
%
dbuser:\\
Welke database gebruiker je wil gebruiken voor het uitvoeren van opdrachten.\\\\
%
group:\\
De groepsnaam waaronder je puppet wil uitvoeren.\\\\
%
ignorecache:\\
Zorg ervoor dat nieuwe configuraties worden binnengehaald, vooral bedoeld voor het forceren van updates van de manifests.\\\\
%
logdir:\\
De standaard map waarnaar je wil loggen.\\\\
%
manifestdir:\\
Stel de '\$MANIFESTDIR' variabele in.
%
masterport:\\
Op welke poort de puppetmaster dient te luisteren voor binnenkomende verbindingen.\\\\
%
modulepath:\\
Instelling voor de '\$MODULEPATH' variabele.\\\\
%
postrun\_command:\\
Geef een commando op dat je wil uitvoeren na elke puppet agent run.\\\\
%
prerun\_command:\\
Geef een commando op dat je wil uitvoeren voor elke puppet agent run.\\\\
%
runinterval:\\
Hoe vaak de puppet agent de configuraties toepast, uitgedrukt in seconden.\\\\
%
server:\\
Met welke server puppet agent verbinding dient te maken.\\\\
%
templatedir:\\
Instelling voor de '\$TEMPLATEDIR' variabele.\\\\
%
user:\\
De gebruiker waarmee je het programma wil uitvoeren.\\\\
%
\subsection{Command line}
Elk puppet commando aanvaard ook elk van deze opties op de command line, alhoewel deze niet altijd even bruikbaar zullen zijn voor elk programma.
\begin{code}
\begin{lstlisting}
puppet agent --confdir=/private/puppet
\end{lstlisting}
\end{code}
%
Voor boolean waardes dien je enkel de parameter op te geven, zonder argument. Indien je een negatieve waarde wil meegeven, dien je simpelweg 'no-' voor de optie te zetten.
\begin{code}
\begin{lstlisting}
$ puppet agent --storeconfigs
$ puppet agent --no-storeconfigs
\end{lstlisting}
\end{code}


\section{Mappenstructuur}
Zoals alle andere programma's heeft ook puppet enkele configuratie-bestanden nodig. Deze bestanden vinden we voor puppet op de meeste linux-distributies terug in de map /etc/puppet. Hier worden niet enkel de configuraties van puppet zelf opgeslagen, maar ook de manifests die hij zal aanbieden aan clients. Per definitie zal puppet eerst zoeken naar een bestand genaamd site.pp . Dit bestand is het beginpunt van alle aangeboden manifests. vanuit dit bestand kan je ook andere bestanden aanhalen, zodat je een mooie, overzichtelijke mappenstructuur kan aanleggen waar je zelf nog aan uit kan.

\section{Uitvoering}
Het uitvoeren van puppet manifests kan je op een aantal verschillende manieren aanpakken, afhankelijk van wat het beste werkt in jouw omgeving. Enkele mogelijkheden zijn:
\begin{itemize}
\item puppet agent
\item puppet cron job
\item manuele uitvoering
\end{itemize}
\subsection{Puppet agent}
In deze configuratie laat je de puppet daemon applicatie steevast op de achtergrond slapen, alwaar ze periodiek zal ontwaken en manifests zal ophalen van de opgegeven locatie (lokaal of via een puppetmaster). Dit is een zeer eenvoudige setup aangezien je enkel de puppet daemon hoeft te starten, waarna alles automatisch verloopt.
%
\subsection{Puppet cron job}
In deze opstelling laat je middels het cron programma op geregelde tijdstippen de puppet applicatie uitvoeren. De puppet applicatie zal dan afhankelijk van de configuratie lokaal of via de puppetmaster manifests ophalen. Het voordeel van deze methode ten opzichte van de puppet agent, is dat de puppet agent steeds op de achtergrond aanwezig moet zijn, terwijl deze manier dit niet vereist. Dit zorgt voor minder verspilling van systeembronnen.
% 
\subsection{Manuele uitvoering}
Deze manier is vooral handig voor het uittesten van manifests. Het is geen automatische oplossing, aangezien je elke keer manueel de puppet applicatie moet uitvoeren, maar komt zeer goed van pas bij het ontwikkelen van nieuwe configuraties voor hosts.
%
\section{Voorbeelden}
hier volgen een aantal voorbeelden van puppet manifests die ik heb geschreven en die de mogelijkheid hebben om steeds een applicatie volledig te uit te rollen. Het gaat hier dan om iets uitgebreidere applicaties, die meestal veel configuratie en aanpassingen vereisen alsook veel moeite om ze op te zetten.

\subsection{Redmine}
Redmine is een flexibele webapplicatie met het oog op projectbeheer, geschreven in Ruby on Rails. De applicatie is cross-platform en dus beschikbaar op zowel Linux, Mac als Windows. Daarnaast is Redmine ook cross-database bruikbaar en dus kan zowel MySQL, PostgreSQL als OracleDB, etc. als backend gebruikt worden.\\\\
%
Enkele mogelijkheden van Redmine, volgens hun offici\"ele website:
\begin{itemize}
    \item Multiple projects support
    \item Flexible role based access control
    \item Flexible issue tracking system
    \item Gantt chart and calendar
    \item News, documents \& files management
    \item Feeds \& email notifications
    \item Per project wiki
    \item Per project forums
    \item Time tracking
    \item Custom fields for issues, time-entries, projects and users
    \item SCM integration (SVN, CVS, Git, Mercurial, Bazaar and Darcs)
    \item Issue creation via email
    \item Multiple LDAP authentication support
    \item User self-registration support
    \item Multilanguage support
    \item Multiple databases support
\end{itemize}

\subsubsection{Setup}
Hier zal ik de setup van Redmine bespreken aan de hand van de manifests, stap per stap. Even ter verduidelijking de mappenstructuur waarin de Redmine manifests zich bevinden:
%
\begin{code}
\begin{lstlisting}
Redmine
+--manifests
|  +--site.pp
|
+--modules
   +--apache
   +--mysql
   +--redmine
   +--repos
\end{lstlisting}
\end{code}
%
Zoals je ziet zit het grootste deel van het zware werk in de modules. In de manifests map is slechts \'e\'en enkel bestand aanwezig. Dit is in feite het allereerste bestand dat word nagekeken, het bestand waarin de nodes gedefini\"eerd staan. De inhoud van het bestand in ons geval is:
%
\begin{code}
\begin{lstlisting}
node redmine {
	include apache::packages
	include apache::mod::passenger
	include mysql::packages
	include repos::centos
	include ruby
	include redmine::install
	include redmine::config
}
\end{lstlisting}
\end{code}
%
Aangezien alles in modules georganiseerd is en er een nette onderverdeling is in classes, is dit stukje relatief kaal doch overzichtelijk. Zo zien we hier telkens een 'include' staan; dit zorgt ervoor dat de classes die erbij horen ingeladen en uitgevoerd worden.
%
Een overzicht van de redmine module:
%
\begin{code}
\begin{lstlisting}
+--redmine
   +--files
   |  +--redmine-1.1.3.tar.gz
   |
   +--manifests
   |  +--centos.pp
   |  +--config.pp
   |  +--debian.pp
   |  +--init.pp
   |  +--install.pp
   |
   +--templates
      +--database.yml.erb
\end{lstlisting}
\end{code}
%
\subsubsection{install.pp}
Installatie van de benodigde packages en de groep en gebruiker voor redmine aanmaken:
%
\begin{code}
\begin{lstlisting}
class redmine::install {
	case $operatingsystem {
		Centos: { include redmine_install_centos }
		Debian: { include redmine_install_debian }
	}

	$redmine_id = $operatingsystem ? {
		/Debian|Ubuntu/ => 'www-data',
		/Centos|Fedora/ => 'apache',
	}

	group { redmine:
		ensure => present,
		name => $redmine_id,
	}	
\end{lstlisting}
\end{code}
%
Vervolg van het install.pp bestand:\\*
\begin{code}
\begin{lstlisting}
	user { redmine:
		ensure => present,
		name => $redmine_id,
		gid => $redmine_id,
		require => Group["$redmine_id"],
	}

	package { redmine:
		ensure => installed,
		name => $operatingsystem ? {
			Centos => 'redmine_client',
			Debian => 'redmine',
		},
		provider => $operatingsystem ? {
			Centos => "gem",
			Debian => "apt",
		},
		before => Exec["config_redmine_mysql_bootstrap"],
		require => [ User['redmine'], Class['apache_packages', 'mysql_packages'] ],
	}
}
\end{lstlisting}
\end{code}
%
\subsubsection{debian.pp}
Een stuk specifiek voor Debian-gebaseerde besturingssystemen:
%
\begin{code}
\begin{lstlisting}
class redmine::debian {
	package { 'redmine-mysql':
		ensure => installed,
		require => Package['redmine'],
	}

	file { '/etc/apache2/sites-available/redmine':
		ensure => present,
		owner => root,
		group => root,
		mode => 0644,
		content => 'RailsBaseURI /redmine',
		require => Package['redmine'],
	}

	exec { 'config_redmine_link_apache':
		command => '/usr/sbin/a2ensite redmine',
		require => File['/etc/apache2/sites-available/redmine'],
		unless => '/usr/bin/test -f /etc/apache2/sites-enabled/redmine',
	}
}
\end{lstlisting}
\end{code}
%
\subsubsection{centos.pp}
Een stuk specifiek voor CentOS-gebaseerde systemen:
%
\begin{code}
\begin{lstlisting}
class redmine::centos {
	file { '$HOME/.netrc':
		content => 'machine ftp.ruby-lang.org login anonymous password anonymous\nmacdef init\nprompt\ncd /pub/ruby\nget ruby-1.8.7-p334.tar.gz\nbye',
	}

	file { '/usr/share/redmine-1.1.3.tar.gz':
		ensure => present,
		source => 'puppet:///modules/redmine/redmine.tar.gz',
	}

	exec { 'extract_redmine':
		path => '/bin:/usr/bin',
		command => 'cd /usr/share && tar xzvf redmine-1.1.3.tar.gz redmine && touch /usr/share/redmine/redmine.puppet',
		require => File['/usr/share/redmine-1.1.3.tar.gz'],
		unless => '/usr/bin/test -f /usr/share/redmine/redmine.puppet',
	}
\end{lstlisting}
\end{code}
%
Vervolg van het centos.pp bestand:
%
\begin{code}
\begin{lstlisting}
	file { '/etc/redmine':
		ensure => directory,
		owner => root,
		group => root,
		mode => 0755,
		before => File['/etc/redmine/default'],
	}

	file { '/etc/redmine/default':
		ensure => directory,
		owner => $redmine_id,
		group => $redmine_id,
		mode => 0755,
		before => Class['redmine_config'],
		require => Exec['redmine_centos'],
	}

	package { 'gem_i18n':
		ensure => '0.4.2',
		provider => gem,
		before => Package['gem_rails'],
	}

	package { 'gem_mysql':
		ensure => installed,
		name => mysql,
		provider => gem,
		require => Package['gem_i18n'],
	}

	package { 'gem_rack':
		ensure => '1.0.1',
		name => 'rack',
		provider => gem,
		before => Package['gem_rails'],
	}

	package { 'gem_hoe':
		ensure => installed,
		name => 'hoe',
		provider => gem,
		before => Package['gem_rails'],
	}
\end{lstlisting}
\end{code}
%
Vervolg van het centos.pp bestand:
%
\begin{code}
\begin{lstlisting}
	package { 'gem_rails':
		ensure => installed,
		name => 'rails',
		provider => gem,
		before => Exec['config_redmine_mysql_bootstrap'],
	}

	exec { 'build_passenger_modules':
		path => '/bin:/usr/bin:/opt/ruby/bin',
		command => 'passenger-install-apache2-module -a',
		require => Package['$package_apache_mod_passenger'],
		unless => 'test -f /opt/ruby/lib/ruby/gems/1.8/gems/passenger-3.0.7/ext/apache2/mod_passenger.so',
	}

	exec { 'selinux_disable':
		path => '/bin:/usr/bin',
		command => 'system-config-securitylevel-tui -q --selinux="disabled"',
		unless => 'cat /etc/selinux/config|grep "SELINUX=disabled"',
		before => Service['apache'],
		notify => Service['apache'],
	}

	exec { 'session_store':
		path => '/bin:/usr/bin:/opt/ruby/bin',
		command => '/bin/sh -c "cd /usr/share/redmine/public && rake generate_session_store"',
		require => Package['gem_rails'],
	}

	file { '/etc/httpd/conf.d/redmine.conf':
		ensure => present,
		content => '<VirtualHost *:80>\n\tDocumentRoot /usr/share/redmine/public\n\tErrorLog logs/redmine_error_log\n</VirtualHost>',
		notify => Service['apache'],
	}

	exec { 'apache_modules':
		path => '/bin:/usr/bin',
		command => 'echo -e "LoadModule passenger_module /opt/ruby/lib/ruby/gems/1.8/gems/passenger-3.0.7/ext/apache2/mod_passenger.so\nPassengerRoot /opt/ruby/lib/ruby/gems/1.8/gems/passenger-3.0.7\nPassengerRuby /opt/ruby/bin/ruby" >> /etc/httpd/conf/httpd.conf',
		unless => 'cat /etc/httpd/conf/httpd.conf|grep "LoadModule passenger_module"',
		require => Class['apache_mod_passenger', 'rubygems'],
		notify => Service['apache'],
	}
}
\end{lstlisting}
\end{code}
%
\subsubsection{config.pp}
Configuratie van de redmine applicatie, de mysql backend en de apache instellingen:
%
\begin{code}
\begin{lstlisting}
class redmine::config {
	file { '/var/www/redmine':
		ensure => link,
		target => '/usr/share/redmine/public',
		owner => $redmine_id,
		group => $redmine_id,
		require => Package['redmine'],
	}

	exec { 'config_redmine_mysql_db':
		command =>  '/usr/bin/mysqladmin -uroot create redmine',
		unless => '/bin/echo "show databases"|mysql -uroot|grep redmine',
		require => Class['mysql_packages'],
	}

	exec { 'config_redmine_mysql_user':
		command =>  '/bin/echo "create user \'redmine\'@\'localhost\' identified by \'redmine\'"|mysql -uroot',
		unless => '/bin/echo "select user from mysql.user where user=\'redmine\'"|mysql -uroot|grep redmine',
		require => Exec['config_redmine_mysql_db'],
	}

	exec { 'config_redmine_mysql_permissions':
		command => '/bin/echo "grant all on redmine.* to \'redmine\'@\'localhost\'"|mysql -uroot',
		require =>  Exec["config_redmine_mysql_user"],
	}

	exec { 'config_redmine_mysql_bootstrap':
		environment => 'RAILS_ENV=production',
		path => '/usr:/usr/bin:/opt/ruby/bin',
		command => '/bin/sh -c "cd /usr/share/redmine && sudo /opt/ruby/bin/rake db:migrate"',
		require => Exec['config_redmine_mysql_permissions'],
	}
\end{lstlisting}
\end{code}
%
Vervolg van het config.pp bestand:
\begin{code}
\begin{lstlisting}
	exec { 'config_redmine_reload':
		command => $operatingsystem ? {
			Debian => '/etc/init.d/apache2 reload',
			Centos => '/etc/init.d/httpd reload',
		},
		require => Exec['config_redmine_mysql_bootstrap'],
		notify => Service['apache'],
	}
}
\end{lstlisting}
\end{code}
%
\subsubsection{database.yml.erb}
%
Een template voor de database configuratie van redmine:
%
\begin{code}
\begin{lstlisting}
production:
  adapter: mysql
  database: <%= production_db =%>
  username: <%= redmine_db_user =%>
  password: <%= redmine_db_pass =%>
  host: localhost
  encoding: utf8

development:
  adapter: mysql
  database: <%= development_db =%>
  username: <%= redmine_db_user =%>
  password: <%= redmine_db_pass =%>
  host: localhost
  encoding: utf8
\end{lstlisting}
\end{code}
%
\subsubsection{finish}
Na het uitvoeren van deze manifests, samen met alle modules, zal de redmine applicatie volledig werkende zijn. Zo kan je een reproduceerbare installatie-procedure ontwikkelen die je toestaat tegelijkertijd aan andere projecten voort te werken.

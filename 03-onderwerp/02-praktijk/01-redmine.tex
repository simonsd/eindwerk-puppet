\subsection{Redmine}
Redmine is een flexibele webapplicatie met het oog op projectbeheer, geschreven in Ruby on Rails. De applicatie is cross-platform en dus beschikbaar op zowel Linux, Mac als Windows. Daarnaast is Redmine ook cross-database bruikbaar en dus kan zowel MySQL, PostgreSQL als OracleDB, etc. als backend gebruikt worden.\\\\
%
Enkele mogelijkheden van Redmine:
\begin{itemize}
    \item Multiple projects support
    \item Flexible role based access control
    \item Flexible issue tracking system
    \item Gantt chart and calendar
    \item News, documents \& files management
    \item Feeds \& email notifications
    \item Per project wiki
    \item Per project forums
    \item Time tracking
    \item Custom fields for issues, time-entries, projects and users
    \item SCM integration (SVN, CVS, Git, Mercurial, Bazaar and Darcs)
    \item Issue creation via email
    \item Multiple LDAP authentication support
    \item User self-registration support
    \item Multilanguage support
    \item Multiple databases support
\end{itemize}

\subsubsection{Setup}
Hier zal ik de setup van Redmine bespreken aan de hand van de manifests, stap per stap. Even ter verduidelijking de mappenstructuur waarin de Redmine manifests zich bevinden:
%
\begin{code}
\begin{lstlisting}
Redmine
+--manifests
|  +--site.pp
|
+--modules
   +--apache
   +--mysql
   +--redmine
   +--repos
\end{lstlisting}
\end{code}
%
Zoals je ziet zit het grootste deel van het zware werk in de modules, maar aangezien je deze kan hergebruiken naar hartelust zou dit geen probleem mogen zijn. Je kan hierdoor \'e\'en keer een MySQL module maken en deze honderd keer opnieuw gebruiken, of gewoon even op het internet kijken of er al niet zulk een module bestaat. Als je geluk hebt kan je dan simpelweg een module downloaden en hergebruiken. Dan zien we dat er nog \'e\'en enkel bestand in de manifests map staat. Dit is in feite het allereerste bestand dat word nagekeken, het bestand waarin de nodes gedefini\"eerd staan. De inhoud van het bestand in ons geval is:
%
\begin{code}
\begin{lstlisting}
node redmine {
	include apache::packages
	include apache::mod::passenger
	include mysql::packages
	include repos::centos
	include ruby
	include redmine::install
	include redmine::config
}
\end{lstlisting}
\end{code}
%
Hier zie je om te beginnen 'node redmine {'. Dit wil zeggen dat we hier gebruik maken van de resource type 'node', en als naam 'redmine' meegeven. Het haakje zorgt ervoor dat puppet weet dat alle text tot aan het volgende haakje, toebehoort aan dit object. Het aansluitende haakje vinden we in dit geval dan ook terug aan het einde van dit bestand. Aangezien alles in modules georganiseerd is en er een nette onderverdeling is in classes, is dit stukje relatief kaal doch overzichtelijk. Zo zien we hier telkens een 'include' staan; dit zorgt ervoor dat de classes die erbij horen ingeladen en uitgevoerd worden. We zullen hier even bekijken wat deze classes allemaal inhouden, althans de classes die we daarnet nog niet hebben gezien in het hoofdstuk 'Modules'.

Om te beginnen: de 'apache::packages' class. Deze werd in het hoofdstuk 'modules' reeds besproken en slaan we dus over.
%
\begin{code}
\begin{lstlisting}
+--apache
   +--manifests
   |  +--config.pp
   |  +--init.pp
   |  +--mod-php.pp
   |  +--packages.pp
   |  +--passenger.pp
   |  +--prefork.pp
   |  +--xsendfile.pp
   |
   +--templates
     +--httpd.conf.erb
\end{lstlisting}
\end{code}
%
\begin{code}
\begin{lstlisting}
+--mysql
   +--manifests
      +--init.pp
\end{lstlisting}
\end{code}
%
\begin{code}
\begin{lstlisting}
+--redmine
   +--classes
   |  +--redmine.pp
   |
   +--files
   |  +--redmine-1.1.3.tar.gz
   |
   +--manifests
   |  +--init.pp
   |
   +--templates
      +--database.yml.erb
\end{lstlisting}
\end{code}
%
\begin{code}
\begin{lstlisting}
+--repos
   +--manifests
      +--init.pp
\end{lstlisting}
\end{code}
%
\begin{code}
\begin{lstlisting}
+--ruby
   +--manifests
      +--init.pp
\end{lstlisting}
\end{code}

\subsection{Redmine}
Redmine is een flexibele webapplicatie met het oog op projectbeheer, geschreven in Ruby on Rails. De applicatie is cross-platform en dus beschikbaar op zowel Linux, Mac als Windows. Daarnaast is Redmine ook cross-database bruikbaar en dus kan zowel MySQL, PostgreSQL als OracleDB, etc. als backend gebruikt worden.\\\\
%
Enkele mogelijkheden van Redmine:
\begin{itemize}
    \item Multiple projects support
    \item Flexible role based access control
    \item Flexible issue tracking system
    \item Gantt chart and calendar
    \item News, documents \& files management
    \item Feeds \& email notifications
    \item Per project wiki
    \item Per project forums
    \item Time tracking
    \item Custom fields for issues, time-entries, projects and users
    \item SCM integration (SVN, CVS, Git, Mercurial, Bazaar and Darcs)
    \item Issue creation via email
    \item Multiple LDAP authentication support
    \item User self-registration support
    \item Multilanguage support
    \item Multiple databases support
\end{itemize}

\subsubsection{Setup}
Hier zal ik de setup van Redmine bespreken aan de hand van de manifests, stap per stap. Even ter verduidelijking de mappenstructuur waarin de Redmine manifests zich bevinden:
%
\begin{code}
\begin{lstlisting}
Redmine
+--manifests
  +--site.pp
+--modules
  +--apache
    +--manifests
      +--init.pp
  +--mysql
    +--manifests
      +--init.pp
  +--redmine
    +--classes
      +--redmine.pp
    +--files
      +--redmine-1.1.3.tar.gz
    +--manifests
      +--init.pp
    +--templates
      +--database.yml.erb
  +--repos
    +--manifests
      +--init.pp
  +--ruby
    +--manifests
      +--init.pp
\end{lstlisting}
\end{code}
